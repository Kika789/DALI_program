\input ctustyle3  % The template (in version 3, for OpTeX) is included here.

\worktype [M/CZ] % Type: B = bachelor, M = master, D = Ph.D., O = other
                 % / the language: CZ = Czech, SK = Slovak, EN = English

\faculty    {F3}  % Type your faculty F1, F2, F3, etc. or MUVS
            % use main language of your document here:
\department {Katedra měření}
\title      {Uživatelsky přívětivé ovládání systému osvětlení školní učebny}
\subtitle   {User-friendly control of the classroom lighting system}
            % \subtitle is optional
\author     {Bc. Kamila Sedláková}
\date       {Květen 2024}
\supervisor {Ing. arch. Lenka Maierová, Ph.D}  % One or more supervisors
\studyinfo  {}  % Study programme etc.
%\workname   {Dokumentace} % Used only if \worktype [O/*] (Other)
            % optional more information about the document:
%\workinfo   {\url{http://petr.olsak.net/ctustyle.html}}
            % Title / Subtitle in minor language:
%\titleEN    {CTUstyle -- the user manual}
%\subtitleEN {the \OpTeX/ template for theses at CTU}
            % If minor language is other than English
            % use \titleCZ, \subtitleCZ or \titleSK, \subtitleSK instead it.
\pagetwo    {}  % The text printed on the page 2 at the bottom.

\abstractEN {
   This master's thesis focuses on the analysis of artificial lighting in the biology classroom
   at Gymnázium U Libeňského zámku. The first step involved conducting a spectral and luminance analysis
   of the existing lighting conditions. Based on the obtained results, lighting simulations were carried out in
   the DIALux program, leading to the design of optimal lighting scenes with an emphasis on biological comfort.
   In the final phase, the implementation of intuitive lighting control was performed under laboratory conditions.
}
\abstractCZ {
   Tato diplomová práce se zaměřuje na analýzu umělého osvětlení v~učebně biologie na gymnázium U Libeňského zámku.
   Prvním krokem byla provedena spektrální a~jasová analýza stávajících světelných podmínek.
   Na základě získaných výsledků byla provedena simulace osvětlení v~programu DIALux, která vedla
   k~navržení optimálních světelných scén s~důrazem na biologický komfort. V závěrečné fázi byla provedena
   implementace intuitivního světelného ovládání v laboratorních podmínkách.
}           % If your language is Slovak use \abstractSK instead \abstractCZ

\keywordsEN {
   DALI protocol, lighting, spectrometer, DIALux
}
\keywordsCZ {
   DALI protokol, osvětlení, spektrometr, DIALux
}
\thanks {           % Use main language here
   % Chtěla bych poděkovat...
   Chtěla bych moc poděkovat Ing. arch. Lence Maierové, Ph.D. za skvělé vedení mé diplomové prace.
   Vždy byla ochotná mě navést k správnému řešení, i přes její časové vytížení si našla čas na konzultace a dávala mi rady,
   jak dále postupovat a motivovala mě zpracovávat tuto práci s nadšením.

   Ráda bych také poděkovala profesoru Ing. Janu Havlíkovi, Ph.D. za jeho vedení při zaučení s protokolem DALI
   a za pomoc při sestavování světelného ovládání.

   Další dík patří doktorandce Ing. Martině Liberské,
   která mě svým nadšením pro problematiku osvětlení naučila analyzovat osvětlenost pomocí luxmetru.

   Na závěr patří dík doktorandovi Ing. Arch. Patriku Kučerovi za vysvětlení postupu měření pomocí jasového analyzátoru
   a ukázky zpracování výsledků v~softwaru LUMIDisp.

   Díky Vám jsem si zpravování mé diplomové práce velmi užila.

}
\declaration {      % Use main language here
   Prohlašuji, že jsem předloženou práci vypracovala
   samostatně a že jsem uvedla veškeré použité informační zdroje v~souladu
   s~Metodickým pokynem o~dodržování etických principů při přípravě
   vysokoškolských závěrečných prací.

   V~Praze dne ... 2024 % !!! Attention, you have to change this item.
   \signature % makes dots
}

%%%%% <--   % The place for your own macros is here.

% \draft     % Uncomment this if the version of your document is working only - use \rfc

\ifx\usedraft\undefined
\else
   \draft
\fi

%\linespacing=1.7  % uncomment this if you need more spaces between lines
                   % Warning: this works only when \draft is activated!
%\savetoner        % Turns off the lightBlue backround of tables and
                   % verbatims, only for \draft version.
%\blackwhite       % Use this if you need really Black+White thesis.
%\onesideprinting  % Use this if you really don't use duplex printing.

\parskip=3pt plus 1pt % Some space between paragraphs

\specification {%
\vbox to0pt{\vskip-25mm\centerline{\inspic zadani.pdf }\vss}
}

% cti slovníček - zkratky
\input glosdata

\makefront  % Mandatory command. Makes title page, acknowledgment, contents etc.

