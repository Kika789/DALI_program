\input ctustyle3  % The template (in version 3, for OpTeX) is included here.

\worktype [M/CZ] % Type: B = bachelor, M = master, D = Ph.D., O = other
                 % / the language: CZ = Czech, SK = Slovak, EN = English

\faculty    {F3}  % Type your faculty F1, F2, F3, etc. or MUVS
            % use main language of your document here:
\department {Katedra měření}
\title      {Uživatelsky přívětivé ovládání systému osvětlení školní učebny}
\subtitle   {User-friendly control of the classroom lighting system}
            % \subtitle is optional
\author     {Bc. Kamila Sedláková}
\date       {Květen 2024}
\supervisor {Ing. arch. Lenka Maierová, Ph.D}  % One or more supervisors
\studyinfo  {}  % Study programme etc.
%\workname   {Dokumentace} % Used only if \worktype [O/*] (Other)
            % optional more information about the document:
%\workinfo   {\url{http://petr.olsak.net/ctustyle.html}}
            % Title / Subtitle in minor language:
%\titleEN    {CTUstyle -- the user manual}
%\subtitleEN {the \OpTeX/ template for theses at CTU}
            % If minor language is other than English
            % use \titleCZ, \subtitleCZ or \titleSK, \subtitleSK instead it.
\pagetwo    {}  % The text printed on the page 2 at the bottom.

\abstractEN {
   This master thesis is about...
}
\abstractCZ {
   Tato diplomová práce se zabývá...
}           % If your language is Slovak use \abstractSK instead \abstractCZ

\keywordsEN {
   DALI protocol, lighting, spectrometr
}
\keywordsCZ {
   DALI protokol, osvětlení, spektrometr
}
\thanks {           % Use main language here
   Chtěla bych poděkovat...
}
\declaration {      % Use main language here
   Prohlašuji, že jsem předloženou práci vypracovala
   samostatně a že jsem uvedla veškeré použité informační zdroje v~souladu
   s~Metodickým pokynem o~dodržování etických principů při přípravě
   vysokoškolských závěrečných prací.

   V Praze dne ... 2024 % !!! Attention, you have to change this item.
   \signature % makes dots
}

%%%%% <--   % The place for your own macros is here.

%\draft     % Uncomment this if the version of your document is working only.
%\linespacing=1.7  % uncomment this if you need more spaces between lines
                   % Warning: this works only when \draft is activated!
%\savetoner        % Turns off the lightBlue backround of tables and
                   % verbatims, only for \draft version.
%\blackwhite       % Use this if you need really Black+White thesis.
%\onesideprinting  % Use this if you really don't use duplex printing.

\specification {%
\vbox to0pt{\vskip-25mm\centerline{\inspic zadani.pdf }\vss}
}

\makefront  % Mandatory command. Makes title page, acknowledgment, contents etc.

% Teorie
\input 01_Teorie/010_Uvod
\input 01_Teorie/020_Umele_osvetleni
\input 01_Teorie/030_Oko
\input 01_Teorie/040_Rizeni_osvetleni  % asi nebudu psat, ale vykopiruji si parametry z tabulky
\input 01_Teorie/050_DALI

% Prakticka
\input 02_Prakticka/000_Cela


% Prakticka zatim nepouzivam
% \input 05_Prakticka/01_mereni_osv_jas
% \input 05_Prakticka/02_CIE
% \input 05_Prakticka/03_osvetleni_ucebny

\nocite[*] % This command causes all items in the bibliographic database to be added to the bibliography.

\bibchap{
    \usebib/c (simple) reference
}

%\------------------------------------------------------------------------------------
%\input uvod    % Files where the source of the document is prepared.
%\input popis   % Full name is: uvod.tex, popis.tex, the suffix can be omitted.
% \input prilohy2

\bye
