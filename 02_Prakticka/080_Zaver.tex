\chap Závěr

Cílem mé diplomové práce bylo seznámit se s probelmitikou osvětlení ve školských prosterech. Bylo zapotřebí se seznámit s legisatviními požadackami
pro osvětlenost v školských prostprech.

Po prostudování problematiky jsem provedla důkladnou analýzu školní učebny na gymnázium U Libeňského zámku. Mšřila jsem osvětlenost a jasové kontrasty při
různých světlených scénám. Z měření poté bylo vyhodnocení

V praktické části bylo provedeno

Měření, Simulace, Sestavení světelného ovládání,

Poukazuji na to, že se zapomíná na návrh intuitivního ovládání...

Na tuto práci by šlo navázat přidáním senzoru, kde by se osvětlenost měnila podle denního světla, přidat dotykový panel, kde by se daly hodnoty
jasu mohli měnit. Přidat k zapojení WiFi a udělat k tomu aplikaci, kde by byla také možnost změn paramtrů osvětlení.
Návrh nouzoveho osvětlení, přidat zpětnou komunikaci a~monitorovat světla na dálku

Navrhované optimalizované ovládání světel se zaměřuje na zjednodušení a~zefektivnění obsluhy systému.
Snížení počtu tlačítek, rozdělení funkcí a~intuitivní regulace jasu vedou
k pohodlnějšímu a~uživatelsky přívětivějšímu ovládání.
Implementace tohoto řešení by tak mohla významně přispět
ke zlepšení uživatelského komfortu a~zjednodušení ovládání světel v daném prostoru.

