\nonum\chap Závěr

V této diplomové práci byl navržen a implementován optimální ovládací panel pro učebnu
biologie na gymnáziu U Libeňského zámku.
Cílem práce bylo zjednodušit ovládání osvětlení v učebně a navrhnout světelné scény pro různé aktivity studentů.


V teoretické části se práce zabývala legislativními požadavky na osvětlení ve školních prostorech,
fyziologií oka a protokolem DALI, používaným pro komunikaci mezi ovládacím panelem a světly.

V praktické části proběhla analýza aktuálního osvětlení v učebně a na základě jejích
výsledků byly navrženy tři světelné scény pro činnosti - Výklad, Písemka a Adaptace na denní světlo.
Tyto scény byly implementovány v laboratorních podmínkách a otestovány.
Dále byl navržen a zkonstruován ovládací panel se třemi tlačítky pro zapínání scén, dvěma tlačítky pro regulaci jasu
a jedním tlačítkem pro celkové vypnutí světel.

Navržený ovládací panel a světelné scény splňují všechny požadavky na osvětlení v učebně biologie.
Ovládání panelu je intuitivní a snadné a světelné scény zajišťují optimální podmínky pro různé aktivity studentů.
Implementace navrženého systému v učebně by vedla ke zlepšení komfortu studentů a pedagogů a k úspoře energie.

Do navrženého systému by v budoucnu mohly být integrovány senzory pro automatické přizpůsobení
osvětlení dennímu světlu, dotykový panel pro intuitivní regulaci jasu,
Wi-Fi připojení pro vzdálené ovládání a aplikace pro chytré telefony.
Implementace těchto vylepšení by vedla k ještě většímu pohodlí uživatelů a k efektivnějšímu využití osvětlení.


Diplomová práce prokázala, že inteligentní osvětlení má značný potenciál
pro zlepšení komfortu a produktivity v učebnách.
Zároveň však zdůrazňuje, že nedostatečná pozornost věnovaná
intuitivnímu ovládání může vést k neefektivnímu využití inteligentního osvětlení a marnění jeho potenciálu.



---


% Cílem diplomové práce byla navrhnout optimální ovládací panel pro učebnu biologie na gymnáziu U Libeňského zámku. Aktuální
% ovládání se skládá z dvanácti tlačítek a je velmi obtížné s ním zacházet. Bylo zapotřebí tento počet snížit na 3 tlačítka,
% která by zapínala konkretní světelné scény pro činnosti v učebně.

% V teoretické části proběhlo seznámení se s~legislativními požadavky na osvětlení ve školních prostorech, abychom podle toho mohli
% navrhnout optimální scény. Dále proběhlo seznámení se s fyziologii oka, abychom věděli jak se naše lidské oko chová při kontaktu
% se světlem a dále seznámení se s protokolem DALI, jelikož pomocí něho budeme ovládat nový navržený ovládací panel.

% V praktické části na začátku proběhla aktuální analýza osvětlenosti. Byla změřena osvětlenost a spektrální složení světla
% na horizontální rovině ve výšce pracovní plochy 800 mm a na vertikální rovině ve výšce oší průměrného žáka 1200 mm.
% Dále proběhlo měření jasových kontrastů pomocí jasového analyzátoru. Bylo zjištěno, že kontrasty jsou optimální i osvětlenost.
% Měřili se různé světelné scény při maximální intenzitě jasu světel. Byly změřené různé kombinace.
% Jen přímé osvětlení, nebo nepřímé nebo jejich kombinace a i zapínaní jen jednotlivých řad - řada u dveří, ve středu a u okna.

% V softwaru DIALux proběhla simulace navržených světelných scén pro činnosti - Výklad, Písemka a Adaptace na denní světlo, abychom
% minimalizovali používaní umělého osvětleni. Z výsledků simulace byly navržené optimální scény,
% které se budou v budoucnu realizovat v učebně.
% Nyní realizace v učebně neproběhla z provozních důvodů, stále probíhá školní výuka.

% Proto navržené scény byly implementovány v laboratorních podmínkách.
% Tyto tři sceny se zapínají pomocí tři tlačítek a je tam další tlačítko
% off pro uplne vypnutí světel a dvě tlačítka pro regulaci jasu.
% Světla umí komunikovat přes DALI protokol, proto veškerá logika byla
% napsaná v jazyce C++ a ovládací panel byl realizován v laboratoři.

% Touto práci jsme chtěli poukazat na to, že i když se řeší inteligentni osvětleni,
% často se opomíjí intuitivní ovládaní a bez tohoto
% důležitého prvku může být instalace buď zbytečná, obtížně použitelná nebo dokonce nevyužitelná.

% Do navrženého systému by v budoucnu mohly být integrovány senzory pro automatické
% přizpůsobení osvětlení dennímu světlu,
% dotykový panel pro intuitivní regulaci jasu,
% Wi-Fi připojení pro vzdálené ovládání a aplikace pro chytré telefony.
% Implementace těchto vylepšení by vedla k ještě většímu pohodlí uživatelů

---------------------------------------------
\rfc{v tomto odstavci by mohla byt jedna/dve vety o tom, co analyza ukazala
(dobrou rovnomernost, nadstandardni osvetlenost a spektralni kvalitu)
nechcete zde ty navrzene sceny vyjmenovat nebo strucne popsat?
}






------------------------------------------------------------------------------------------------------------
% Cílem diplomové práce bylo provést komplexní analýzu osvětlení ve školní učebně biologie na gymnáziu
% U Libeňského zámku a~na základě získaných dat navrhnout nový ovládací panel pro regulaci navržených světelných scén
% zjednodušující ovládání a~zlepšující uživatelský komfort.

% V teoretické části práce jsme se seznámili s~legislativními požadavky na osvětlení ve školních prostorech,
% fyziologií oka a protokolem DALI.

% V praktické části práce bylo provedeno měření osvětlenosti v~učebně gymnázia U~Libeňského zámku pomocí spektrometru
% a~jasového analyzátoru. Analýza proběhla na horizontální a~vertikální rovině. Dále byla provedena spektrální analýza,
% aby bylo možné získat informace o spektrálním složení světla a~jeho vnímání z pohledu uživatelů. Na základě těchto měření
% byla identifikována nejsvětlejší a~nejtemnější místa v učebně a~byly vyhodnoceny kontrasty mezi tabulí, lavicemi
% a~okolním prostorem.

% V softwaru DIALux byla provedena simulace navržených světelných scén, které zohledňují různé činnosti probíhající v~učebně.
% Byl sestaven ovládací panel pro jednoduché přepínání mezi světelnými scénami v~laboratorních podmínkách.
% Touto prací jsme chtěli zdůraznit, že i~když se řeší inteligentní osvětlení, často se opomíjí
% intuitivního ovládání. Bez tohoto důležitého prvku může být instalace buď zbytečná, obtížně použitelná nebo dokonce nevyužitelná.

% Do navrženého systému by v budoucnu mohly být integrovány senzory pro automatické přizpůsobení osvětlení dennímu světlu,
% dotykový panel pro intuitivní regulaci jasu, Wi-Fi připojení pro vzdálené ovládání a aplikace pro chytré telefony.
% Implementace těchto vylepšení by vedla k ještě většímu pohodlí uživatelů.














% Cílem mé diplomové práce bylo seznámit se s probelmitikou osvětlení ve školských prosterech. Bylo zapotřebí se seznámit s legisatviními požadackami
% pro osvětlenost v školských prostprech.

% Po prostudování problematiky jsem provedla důkladnou analýzu školní učebny na gymnázium U Libeňského zámku. Mšřila jsem osvětlenost na horizontalni a vertikalnin rovine
% abychom ziskali povedomi kolik luxu se nachazi na pracovni rovine a u vertiklani jak je osvetlenost
% vnimana z pohodledu zaka, dale probehla i spektrlni analyza a jasova analyza, abychom vedeli jake jsou  kontrasty při
% různých světlených scénách. Bylo zvoleno několik světelných scén, kde byla světla nastavena na maximální intenzitě.
% Z měření poté bylo vyhodnoceno, jesvteljeis a nejtmavsi mista a kontrasty mezitabulil, lavicemi,...

% Dale v praktické části bylo provedena simulace v DIALuxu z navrzenych scen a na závěr byl sestaven ovládací světelný panel
% pro ovládání různých světelných scén v laboratorních podmínkách. Touto pracá jsme chtěli poukázat, že se sice řeší inteligentní
% osvětlení, ale zapomíná se na intuitivní ovládání a poté je taková instalace zbytečná a nebo náročná na využití a nebo vůbec použitelná

% Na tuto práci by šlo navázat přidáním senzoru, kde by se osvětlenost měnila podle denního světla, přidat dotykový panel, kde by se daly hodnoty
% jasu mohli měnit. Přidat k zapojení WiFi a udělat k tomu aplikaci, kde by byla také možnost změn paramtrů osvětlení.
% Návrh nouzoveho osvětlení, přidat zpětnou komunikaci a~monitorovat světla na dálku

% Navrhované optimalizované ovládání světel se zaměřuje na zjednodušení a~zefektivnění obsluhy systému.
% Snížení počtu tlačítek, rozdělení funkcí a~intuitivní regulace jasu vedou
% k pohodlnějšímu a~uživatelsky přívětivějšímu ovládání.
% Implementace tohoto řešení by tak mohla významně přispět
% ke zlepšení uživatelského komfortu a~zjednodušení ovládání světel v daném prostoru.


% Měření, Simulace, Sestavení světelného ovládání,

% Poukazuji na to, že se zapomíná na návrh intuitivního ovládání...

