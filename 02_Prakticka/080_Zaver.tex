\nonum\chap Závěr

Cílem diplomové práce bylo provést komplexní analýzu osvětlení ve školní učebně a~na základě získaných dat
navrhnout nový ovládací panel pro osvětlení, zjednodušující ovládání a~zlepšující uživatelský komfort.

V teoretické části práce jsme se seznámili s~legislativními požadavky na osvětlení ve školních prostorech,
fyziologií oka a protokolem DALI.

V praktické části práce bylo provedeno měření osvětlenosti v~učebně gymnázia U~Libeňského zámku pomocí spektrometru
a~jasového analyzátoru. Analýza proběhla na horizontální a~vertikální rovině. Dále byla provedena spektrální analýza,
aby bylo možné získat informace o spektrálním složení světla a~jeho vnímání z pohledu uživatelů. Na základě těchto měření
byla identifikována nejsvětlejší a~nejtemnější místa v učebně a~byly vyhodnoceny kontrasty mezi tabulí, lavicemi
a~okolním prostorem.

V softwaru DIALux byla provedena simulace navržených světelných scén, které zohledňují různé činnosti probíhající v~učebně.
Byl sestaven ovládací panel pro jednoduché přepínání mezi světelnými scénami v~laboratorních podmínkách.
Touto prací jsme chtěli zdůraznit, že i~když se řeší inteligentní osvětlení, často se opomíjí
intuitivního ovládání. Bez tohoto důležitého prvku může být instalace buď zbytečná, obtížně použitelná nebo dokonce nevyužitelná.

Do navrženého systému by v budoucnu mohly být integrovány senzory pro automatické přizpůsobení osvětlení dennímu světlu,
dotykový panel pro intuitivní regulaci jasu, Wi-Fi připojení pro vzdálené ovládání a aplikace pro chytré telefony.
Implementace těchto vylepšení by vedla k ještě většímu pohodlí uživatelů.














% Cílem mé diplomové práce bylo seznámit se s probelmitikou osvětlení ve školských prosterech. Bylo zapotřebí se seznámit s legisatviními požadackami
% pro osvětlenost v školských prostprech.

% Po prostudování problematiky jsem provedla důkladnou analýzu školní učebny na gymnázium U Libeňského zámku. Mšřila jsem osvětlenost na horizontalni a vertikalnin rovine
% abychom ziskali povedomi kolik luxu se nachazi na pracovni rovine a u vertiklani jak je osvetlenost
% vnimana z pohodledu zaka, dale probehla i spektrlni analyza a jasova analyza, abychom vedeli jake jsou  kontrasty při
% různých světlených scénách. Bylo zvoleno několik světelných scén, kde byla světla nastavena na maximální intenzitě.
% Z měření poté bylo vyhodnoceno, jesvteljeis a nejtmavsi mista a kontrasty mezitabulil, lavicemi,...

% Dale v praktické části bylo provedena simulace v DIALuxu z navrzenych scen a na závěr byl sestaven ovládací světelný panel
% pro ovládání různých světelných scén v laboratorních podmínkách. Touto pracá jsme chtěli poukázat, že se sice řeší inteligentní
% osvětlení, ale zapomíná se na intuitivní ovládání a poté je taková instalace zbytečná a nebo náročná na využití a nebo vůbec použitelná

% Na tuto práci by šlo navázat přidáním senzoru, kde by se osvětlenost měnila podle denního světla, přidat dotykový panel, kde by se daly hodnoty
% jasu mohli měnit. Přidat k zapojení WiFi a udělat k tomu aplikaci, kde by byla také možnost změn paramtrů osvětlení.
% Návrh nouzoveho osvětlení, přidat zpětnou komunikaci a~monitorovat světla na dálku

% Navrhované optimalizované ovládání světel se zaměřuje na zjednodušení a~zefektivnění obsluhy systému.
% Snížení počtu tlačítek, rozdělení funkcí a~intuitivní regulace jasu vedou
% k pohodlnějšímu a~uživatelsky přívětivějšímu ovládání.
% Implementace tohoto řešení by tak mohla významně přispět
% ke zlepšení uživatelského komfortu a~zjednodušení ovládání světel v daném prostoru.


% Měření, Simulace, Sestavení světelného ovládání,

% Poukazuji na to, že se zapomíná na návrh intuitivního ovládání...

