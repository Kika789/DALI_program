\sec Druhé měření
Druhé měření zahrnovalo měření osvětlenosti a spektrální analýzu, které proběhlo 4.3.2024 v čase 17:00--18:30 za tmy.
Byla změřena vertikální rovina při různých světelných scénách.
Celkem bylo zvoleno šest světelných scén a~změřeno 18 měřicích kontrolních bodů. U všech scén byla světla nastavena na maximální intenzitě a~okna byla zatažena.
Tyto body byly měřeny ve výšce 1200 mm, což odpovídá výšce očí žáků. Vertikální
rovinu je zapotřebí měřit z důvodu biologického komfortu, abychom měli informaci
o tom, jak je osvětlenost vnímaná z pohledu pozorovatele

\medskip\noindent
{\sbf Nastavené světelné scény:}
\begitems \style n
    *Maximální intenzita kombinovaného osvětlení.
    *Maximimální intenzita přímého osvětlení.
    *Maximální intenzita nepřímé osvětlení.
\enditems

\secc Grafy
\medskip \clabel[max_kom_vert]{Maximální intenzita kombinovaného osvětlení - vertikální rovina}
\picw=15cm \cinspic 04_Grafy/02_mereni/gr_02_MAX_kombi_podruhe.png
\caption/f Maximální intenzita přímého a~nepřímého osvětlení - vertikální rovina
%\medskip
\medskip \clabel[max_nepr_vert]{Maximální intenzita nepřímého osvětlení - vertikální rovina}
\picw=15cm \cinspic 04_Grafy/02_mereni/gr_03_MAX_horni.png
\caption/f Maximální intenzita nepřímého osvětlení - vertikální rovina
%\medskip

\medskip \clabel[max_prime_vert]{Maximální intenzita přímého osvětlení - vertikální rovina}
\picw=15cm \cinspic 04_Grafy/02_mereni/gr_04_MAX_dolni.png
\caption/f Maximální intenzita přímého osvětlení - vertikální rovina
\medskip

Analýza grafů ukazuje, že hodnoty osvětlení v luxech se u vertikálního a horizontálního měření příliš neliší.
To je důležité, protože umožňuje zachovat podobné podmínky pro různé úkoly a předcházet tak rychlé únavě očí v důsledku
střídání zrakových činností.

\noindent {\sbf Kombinované osvětlení}
\begitems \style o
* max: 796
* min:350
* rovnoměrnost: 0,4
\enditems

\noindent {\sbf Přímé osvětlení}
\begitems \style o
* max: 505
* min: 233
* rovnoměrnost: 0,5
\enditems

\noindent {\sbf Nepříme osvětlení}
\begitems \style o
* max: 291
* min: 117
* rovnoměrnost: 0,4
\enditems

Použitý měřicí přístroj (GL Spectis 1.0 Touch) zachycuje osvětlení z celé polokoule,
avšak lidské oko má zorné pole menší. Z toho důvodu mohou být výsledky mírně zkreslené.
Pro zpřesnění analýzy by bylo vhodné eliminovat z měření části osvětlení, které neodpovídají tvaru obličeje
(např. nadočnicové oblouky).


%%%%%%%%%%%%%%%%%%%%% to ne

% Z hlediska světelně technického posouzení tedy student vykonává činnosti charakteru
% podobnému kancelářím. Ve školském prostředí dochází ovšem k pravidelné změně zrakového úkolu.
% Žák mění polohu hlavy a očí, mění zorný úhel a tomu je nutné přizpůsobit podmínky v učebně. Zde je
% nutná rychlá a častá adaptace oka. Střídavá změna ostřící vzdálenosti a změna z vertikální na
% horizontální rovinu vede k rychlé únavě očí. Světelné podmínky by neměly být v horizontální rovině
% příliš odlišné od vertikální, tak aby podmínky činnosti zůstávaly při obou úkonech podobné. Již
% z tohoto faktu se dá vyvodit, že správně by se ve školách měla využívat kombinace rozptýleného a
% přímého světla.

% V rámci měření se nepodařilo eliminovat vznik případných chyb. Jednou
% z pozorovaných odchylek je zaměření vertikální roviny, kdy použitý přístroj GL Spectis
% 1.0 Touch měří světelné záření dopadající do bodu z celého poloprostoru, zatímco zorné
% pole člověka nemá plný rozsah, viz obrázek 24. Z toho vyplývá předpoklad, že v případě
% zohlednění skutečného zorného pole by došlo~eliminování části osvětlení přímo nad
% měřeným bodem (stínění tvarem obličeje, např. nadočnicovými oblouky), a~tedy i~ke
% snížení míry osvětlenosti měřené vertikální roviny

% Horizontální: MAX KOMBI, MAX přímý: u dveří, uprostřed, u okna


