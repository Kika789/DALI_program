\sec Druhé měření
Druhé měření proběhlo 4.3.2024 - 17:00 – 18:30 za tmy. Byla změřena vertikální rovina při různých světelných scénách. Celkem bylo zvoleno šest světelných scén a změřeno 18 vybraných bodů. Tyto body byly měřeny ve výšce 1200 mm, což odpovídá výšce očí žáků.
\medskip

{\sbf Nastavené světelné scény}

\begitems \style n
    *Maximální intenzita kombinovaného osvětlení
    *Maximimální intenzita přímého osvětlení
    *Maximální intenzita nepřímé osvětlení
\enditems

\secc Grafy
\medskip \clabel[max_kom_vert]{Maximální intenzita kombinovaného osvětlení - vertikální rovina}
\picw=15cm \cinspic 04_Grafy/02_mereni/gr_02_MAX_kombi_podruhe.png
\caption/f Maximální intenzita přímého a nepřímého osvětlení - vertikální rovina
%\medskip
\medskip \clabel[max_nepr_vert]{Maximální intenzita nepřímého osvětlení - vertikální rovina}
\picw=15cm \cinspic 04_Grafy/02_mereni/gr_03_MAX_horni.png
\caption/f Maximální intenzita nepřímého osvětlení - vertikální rovina
%\medskip

\medskip \clabel[max_prime_vert]{Maximální intenzita přímého osvětlení - vertikální rovina}
\picw=15cm \cinspic 04_Grafy/02_mereni/gr_04_MAX_dolni.png
\caption/f Maximální intenzita přímého osvětlení - vertikální rovina
\medskip

Z grafů vidíme, že hodnoty v luxech jsou nižší než u horizontální roviny. Vertikální rovinu je zapotřebí měřit z důvodu biologického komfortu, abychom měli informaci o~tom, jak je osvětlenost vnímaná z pohledu pozorovatele.

% Horizontální: MAX KOMBI, MAX přímý: u dveří, uprostřed, u okna


