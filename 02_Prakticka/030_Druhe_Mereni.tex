\sec Druhé měření
Druhé měření zahrnovalo spektrální analýzu, které proběhlo 4.3.2024 v čase 17:00--18:30 za tmy.
Byla změřena vertikální rovina při různých světelných scénách.
Celkem bylo zvoleno šest světelných scén a~změřeno 18 měřicích kontrolních bodů. U všech scén byla světla nastavena na maximální intenzitě a~okna byla zatažena.
Tyto body byly měřeny ve výšce 1200 mm, což odpovídá výšce očí žáků. Vertikální
rovinu je zapotřebí měřit z důvodu biologického komfortu, abychom měli informaci
o tom, jak je osvětlenost vnímaná z pohledu pozorovatele

\medskip\noindent
{\sbf Nastavené světelné scény:}
\begitems \style n
    *Maximální intenzita kombinovaného osvětlení.
    *Maximimální intenzita přímého osvětlení.
    *Maximální intenzita nepřímé osvětlení.
\enditems

\secc Grafy
\medskip \clabel[max_kom_vert]{Maximální intenzita kombinovaného osvětlení - vertikální rovina}
\picw=15cm \cinspic 04_Grafy/02_mereni/gr_02_MAX_kombi_podruhe.png
\caption/f Maximální intenzita přímého a~nepřímého osvětlení - vertikální rovina
%\medskip
\medskip \clabel[max_nepr_vert]{Maximální intenzita nepřímého osvětlení - vertikální rovina}
\picw=15cm \cinspic 04_Grafy/02_mereni/gr_03_MAX_horni.png
\caption/f Maximální intenzita nepřímého osvětlení - vertikální rovina
%\medskip

\medskip \clabel[max_prime_vert]{Maximální intenzita přímého osvětlení - vertikální rovina}
\picw=15cm \cinspic 04_Grafy/02_mereni/gr_04_MAX_dolni.png
\caption/f Maximální intenzita přímého osvětlení - vertikální rovina
\medskip

Z~grafů vidíme, že hodnoty v~luxech jsou nižší než u~horizontální roviny. U kombinovaného a~nepřímého osvětlení jsou výrazně nižší (přidat procenta)
u přímého se o~tolik neliší.

Použitý měřicí přístroj GL Spectis 1.0 Touch měří osvětlenost z celé polokoule, ale zorné pole lidského oko nemá takový rozsah, proto
jsou zde zahrnuty chyby z měření. Kdybych chtěli upřesnit výsledky bylo by zapotřebí eliminovat části osvětlení, aby to odpovídalo tvaru obličeje, může se např.
jednat o~nadočnicové oblouky.
\medskip
% V rámci měření se nepodařilo eliminovat vznik případných chyb. Jednou
% z pozorovaných odchylek je zaměření vertikální roviny, kdy použitý přístroj GL Spectis
% 1.0 Touch měří světelné záření dopadající do bodu z celého poloprostoru, zatímco zorné
% pole člověka nemá plný rozsah, viz obrázek 24. Z toho vyplývá předpoklad, že v případě
% zohlednění skutečného zorného pole by došlo~eliminování části osvětlení přímo nad
% měřeným bodem (stínění tvarem obličeje, např. nadočnicovými oblouky), a~tedy i~ke
% snížení míry osvětlenosti měřené vertikální roviny

% Horizontální: MAX KOMBI, MAX přímý: u dveří, uprostřed, u okna


