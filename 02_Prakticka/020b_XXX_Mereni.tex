\sec První měření

Tabulka s~parametry učebny…. Zapotřebí naměřit
\medskip

Diskuze s~učiteli
\begitems
    * tlačítko off/on
    * scéna - výklad
    * scéna - písemka
    * scéna - zapnout jen nějakou řadu, podle denního světla
\enditems

\begitems
Varianty řešení:
    * nastavit na pevno 3 - 4 světelné scény
    * režim přepínání
\enditems

\medskip
První měření proběhla pomocí spektrometru, kde bylo naměřeno 18 bodů a podle toho jsme dostali povědomí
o~osvětlenosti v~učebně, podle normy víme, že je požadováno 500 lx z~výsledků nám vyšlo …
Spektrometr je zařízení - popsat funkci, parametry + grafy s~výsledky

\medskip \clabel[spektrometr]{Spektrometr}
\picw=5cm \cinspic 02_Obrazky/spektrometr.png
\caption/f Spektrometr
\medskip

\sec 2.měření - 4.3.2024
\begitems
    * změřené parametry učebny
    * měření:
    * Vertikální rovina: MAX KOMBI, MAX přímý, MAX nepřímé
    * Horizontální: MAX KOMBI, MAX přímý: u~dveří, uprostřed, u~okna
\enditems

\sec 3.měření - 6.3.2024
{\bf MAX horni}
\begitems
    * dvěře
    * uprostřed
    * okno
\enditems
\medskip
\begitems
{\bf MAX kombi}
    * dveře
    * uprostřed
    * okno
\enditems
\medskip

\begitems
{\bf MAX dolní}
    * dveře
    * uprostřed
    * okno
\enditems

\sec Zaučení s~DALI protokolem

Schéma řídicí jednotky → popsat jednotlivé bloky

\medskip
\picw=12cm \cinspic 02_Obrazky/01DALI.png
\caption/f Schema
\medskip

\medskip
\picw=13cm \cinspic 02_Obrazky/02DALI.png
\caption/f Schema
\medskip

\begitems
    * V~kódu bylo zapotřebí ošetřit vstupy
    * Funkce zapnutí a vypnutí světelného zařízení → funguje
    * Funkce nastavení maximální intenzity → funguje
\enditems

\medskip
\picw=15cm \cinspic 02_Obrazky/kod01.png
\medskip

\medskip
\picw=15cm \cinspic 02_Obrazky/kod02.png
\caption/f Schema
\medskip