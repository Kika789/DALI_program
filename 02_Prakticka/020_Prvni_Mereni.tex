\chap Měření

V~praktické části diplomové práce proběhla analýza současného stavu umělého osvětlení ve školní učebně.
Nejprve byly změřeny rozměry učebny pomocí laserového měřicího přístroje {\sbf CP~--100P},
pro ověření poskytnutých podkladů.
% \medskip

Dále bylo provedeno měření umělého osvětlení. Měřila se osvětlenost E (lx) a~jas L~(cd/m2).
Osvětlenost v učebně byla změřena v~horizontální rovině, tedy na pracovní ploše studentů,
ale také ve vertikální rovině. Díky tomu byla získána představa o tom, jak je prostor osvětlený
a jak to ovlivňuje studenty při práci.
Horizontální měření nám dává informaci, zda je dostatek světla na
psaní a~další činnosti na lavici a~vertikální měření vnímání zrakové pohody při pohledu na tabuli.
% \medskip

Osvětlenost byla změřena pomocí spektrometru {\sbf GL Spectis 1.0 Touch}~\cite[spektrometr] (obr.~\ref[spektrometr]),
jedná se o~kalibrovaný přístroj
s~dotykovým displejem s parametry dle tab. \ref[tech_par_GL].
Jeho výhodou je, že je napájený baterií a~lze s~ním měřit i~v~terénu.
Obsahuje standardní difuzér -- kosinusově korigovaná měřicí hlava (obr.~\ref[polokoule]),
která snímá horní polokouli nad senzorem. Z~Lambertova
kosinového zákona vyplývá, že intenzita záření pozorovaná na „lambertovském“ povrchu je přímo úměrná kosinu
úhlu mezi dopadajícím světlem a~normálou k~povrchu.

\medskip \clabel[polokoule]{Měřicí křivka osvětlení}
\picw=9cm \cinspic 03_Obrazky/polokoule_graf.png
\caption/f Měřicí křivka osvětlení
\medskip
Se spektrometrem lze měřit osvětlení, celkový světelný tok, celkový zářivý výkon, barevnou teplota v~souladu
se standardem \glref{CIE} a~další světelné vlastnosti.

\medskip \clabel[spektrometr]{Spektrometr GL Spectis 1.0 Touch}
\picw=5cm \cinspic 03_Obrazky/spektrometr.jpg
\caption/f Spektrometr GL Spectis 1.0 Touch~\cite[spektrometr]
\medskip

% \medskip\noindent
% {\sbf Technické parametry GL Spectis 1.0 Touch}

\medskip
\midinsert \clabel[tech_par_GL]{Technické parametry GL Spectis 1.0 Touch}
\ctable{lc}{
% \hfil {\sbf Technické parametry GL Spectis 1.0 Touch Touch UVa - VIS:} \crl \tskip 4pt
Osvětlení:             &   10 -- 100 000 lx \cr
Ozáření:               & 0,03 -- 600 W$\!$/m$^2$\cr
Spektrální rozsah:     &  340 -- 780 nm \cr
}
\caption/t Technické parametry GL Spectis 1.0 Touch~\cite[spektrometr]
\endinsert

Bylo zvoleno 18 kontrolních měřicích bodů, aby zahrnovaly místa s~největším a~nejmenším osvětlením
v~učebně, pokrývaly celý prostor třídy a~byly zaznamenány jejich přesné souřadnice.
Analýza horizontální roviny osvětlení proběhla přibližně v~800~mm výšce a~ve vertikální
rovině ve výšce 1200~mm nad podlahou, což odpovídá průměrné výšce očí žáků.
Pro vertikální měření byly vybrány stejné měřicí body jako pro horizontální roviny.

Při provádění měření umělého osvětlení je klíčové zvolit vhodný stav místnosti,
který minimalizuje příspěvek denního světla. K tomuto účelu můžeme zvolit následující postupy~\cite[elektro_mereni]:
\begitems \style o
    * Měření v době bez přirozeného osvětlení, tedy za tmy.
    * Měření během dne za zatemněných podmínek, kdy jsou všechny okna zatažená.
\enditems

Před samotným zahájením měření je důležité umělé osvětlení zapnout s dostatečným předstihem,
aby se světelný tok ustálil a~stabilizoval~\cite[elektro_mereni].

Spektrometr snímá horní polokouli, proto je nutné zajistit, aby měřicí nástavec nebyl zastíněn
například měřící osobou nebo jinou překážkou.


% Při použití spektrometru je důležité zajistit, že měřicí nástavec není žádným způsobem zastíněn,
% například měřící osobou. Spektrometr totiž snímá horní polokouli světla, a~proto
% je nezbytné, aby byl umístěn tak, aby nebyl zastíněn žádnou překážkou.

Během analýzy byly změřeny různé světelné scény, aby mohly být následně navržené optimální světelné scény
v~souladu s~biologickými potřebami uživatelů. Tento přístup nám umožnil lépe porozumět potřebám osvětlení
v~učebním prostoru a~navrhnout efektivní osvětlení.% pro všechny uživatele.

\sec Měření osvětlenosti a~spektra na horizontální rovině

První měření zahrnovalo analýzu osvětlenosti na horizontální rovině ve výšce 800~mm
při různých světelných scénách a~spektrální analýzu,
které proběhlo 21.3.2024 od 18:00 do 18:55 za tmy.
U~všech scén byla světla nastavena na maximální intenzitu a~okna byla zatažena.
Celkem bylo zvoleno šest světelných scén a~změřeno 18 měřicích kontrolních bodů.

\medskip\noindent
{\sbf Nastavené světelné scény:}
\begitems \style n
    * Maximální intenzita kombinace (přímé + nepřímé svícení) osvětlení.
    * Maximální intenzita přímého (jen přímé svícení) osvětlení.
    * Maximální intenzita nepřímého (jen nepřímé svícení) osvětlení.
    * Minimální intenzita kombinace osvětlení.
    * Minimální intenzita přímého osvětlení.
    * Minimální intenzita nepřímého osvětlení.
\enditems

\secc Výsledky měření
Na grafech \ref[max_kombi] až \ref[max_neprime] jsou zobrazena jednotlivá spektra v~měřených bodech.
V~legendě jsou body seřazeny od nejvyšší
po nejnižší hodnotu a~vedle každého bodu je uvedená naměřená osvětlenost v~luxech.


\medskip \clabel[max_kombi]{Maximální intenzita kombinovaného  osvětlení - horizontální rovina}
\picw=15cm \cinspic 04_Grafy/01_mereni/gr_01_MAX_kombi.jpg
\caption/f Maximální intenzita kombinovaného osvětlení - horizontální rovina
%\medskip

\medskip \clabel[max_prime]{Maximální intenzita přímého osvětlení - horizontální rovina}
\picw=15cm \cinspic 04_Grafy/01_mereni/gr_02_MAX_spodni.jpg
\caption/f Maximální intenzita přímého osvětleni - horizontální rovina
%\medskip

\medskip \clabel[max_neprime]{Maximální intenzita nepřímého osvětlení - horizontální rovina}
\picw=15cm \cinspic 04_Grafy/01_mereni/gr_03_MAX_vrchni.jpg
\caption/f Maximální intenzita nepřímého osvětleni - horizontální rovina

Z grafů vidíme, že nejvyšší hodnota v luxech je v bodě~8 a~nejnižší v bodě~18.
Jelikož světla byla nastavená na maximální intenzitu
je vidět z hodnot, že nejsou vhodná pro běžné použití.

Z prvního grafu vidíme, že nejvyšší intenzita je 1436 lx, tato hodnota odpovídá součtu přímého a~nepřímého osvětlení. Nejnižší je
860 lx. Rovnoměrnost osvětlení je 0,7. Osvětlenost kombinovaného osvětlení je v 18. bodě {60\pcent} oproti maximální hodnotě.

Osvětlenost přímého osvětlení je v 18. bodě {50\pcent}  oproti maximální hodnotě. Rovnoměrnost je zde 0,6. Osvětlenost je nižší, protože přímé osvětlení
má slabší výkon a~neodráží se od stěn. U~přímého osvětlení jsou viditelné ostré píky v~modrém spektru.

Osvětlenost nepřímého je v 18. bodě {65\pcent}  oproti maximální hodnotě. Rovnoměrnost je 0,8.
Jedná se plnospektrální \glref{LED} osvětlení, proto v grafech
nevidíme ostré píky.


Měření s~minimální intenzitou světla bylo provedeno, avšak hodnoty se pohybovaly pod 10 lx. Vzhledem k~tomu,
že spektrometr má rozsah měření od 10 lx do 100 000 lx, se tyto hodnoty jevily spíše jako nesmyslné, a~proto
nebyly použity pro další zpracování.



% Z~grafů je patrné, že u~kombinovaného osvětlení jsou hodnoty v~luxech příliš vysoké a~nejsou vhodné pro běžné použití.
% U~přímého osvětlení jsou viditelné ostré píky v~modrém spektru, což není případ u~nepřímého osvětlení.
% Hodnoty u~nepřímého mají vyšší hodnoty v~luxech.
% Přímé osvětlení má nižší hodnoty než kombinované a nepřímé, a to z důvodu, že se svícení neodráží od stěn.