\chap Měření

V~praktické části diplomové práce proběhla současná analýza umělého osvětlení ve školní učebně.
Nejprve byly změřeny rozměry učebny pomocí laserového měřáku,
abychom si ověřili poskytnuté podklady.
Osvětlenost byla změřena pomocí spektrometru {\sbf GL Spectis 1.0 Touch}, jedná se o~kalibrovaný přístroj
s~dotykovým displejem. Jeho výhodou je, že je napájený baterií a lze s~ním měřit i v~terénu. Obsahuje standardní
difuzér - kosinusově korigovaná měřicí hlava, která snímá horní polokouli světla nad senzorem. Z~Lambertova
kosinového zákona vyplývá, že intenzita záření pozorovaná na „lambertovském“ povrchu je přímo úměrná kosinu
úhlu mezi dopadajícím světlem a normálou k~povrchu.

\medskip \clabel[polokoule]{Měřicí křivka osvětlení}
\picw=8cm \cinspic 03_Obrazky/polokoule_graf.png
\caption/f Měřicí křivka osvětlení
\medskip
Se spektrometrem lze měřit osvětlení, celkový světelný tok, celkový zářivý výkon, barevnou teplota v~souladu
se standardem CIE a další světelné vlastnosti.

\medskip \clabel[spektrometr]{Spektrometr GL Spectis 1.0 Touch}
\picw=5cm \cinspic 03_Obrazky/spektrometr.png
\caption/f Spektrometr GL Spectis 1.0 Touch
\medskip

\medskip
{\sbf Technické parametry GL Spectis 1.0 Touch Touch UVa - VIS:}
\medskip
\ctable{lrrrrr}{
% \hfil {\bf Technické parametry GL Spectis 1.0 Touch Touch UVa - VIS:} \crl \tskip 4pt
\table{llllll}{
Osvětlení (lux) & 10 - 100 000 lx \cr
Ozáření [W/m$^2$] & 0,03 - 600 W/m$^2$ \cr
Spektrální rozsah & 340 - 780 nm \cr}
}

\medskip
Body pro zjištění potřebných parametrů byly zvoleny tak, aby pokryly celou třídu a umožnily důkladné zkoumání
osvětlenosti v~celém prostoru. Analýza probíhala jak v~horizontální, tak ve vertikální rovině. Pro vertikální
rovinu byly zvoleny stejné měřicí body a měření probíhalo ve výšce 1200 mm, což odpovídá průměrné výšce očí žáků.
\medskip
Během analýzy byly změřeny různé světelné scény, aby mohly být následně navržené optimální světelné scény
v~souladu s~biologickými potřebami uživatelů. Tento přístup nám umožnil lépe porozumět potřebám osvětlení
v~učebním prostoru a navrhnout efektivní osvětlení pro všechny uživatele.
\medskip
\sec První měření
První měření probíhalo 21.3.2024 od 18:00 do 18:55 za tmy. Byla změřena horizontální rovina při různých
světelných scénách. Celkem bylo zvoleno šest světelných scén a změřeno 18 vybraných bodů.
\medskip

{\sbf Nastavené světelné scény:}
\medskip
\begitems \style n
    * Maximální intenzita kombinace (= přímé + nepřímé svícení) osvětlení
    * Maximální intenzita přímého osvětlení
    * Maximální intenzita nepřímého osvětlení
    * Minimální intenzita kombinace osvětlení
    * Minimální intenzita přímého osvětlení
    * Minimální intenzita nepřímého osvětlení
\enditems

\secc Grafy
Na grafech jsou zobrazena jednotlivá spektra v~měřených bodech. V~legendě jsou body seřazeny od nejvyšší
po nejnižší hodnotu, a vedle každého bodu je uvedená naměřená osvětlenost v~luxech.
\medskip \clabel[max_kombi]{Maximální intenzita kombinovaného  osvětlení}
\picw=15cm \cinspic 04_Grafy/01_mereni/gr_01_MAX_kombi.png
\caption/f Maximální intenzita kombinovaného osvětlení
%\medskip

\medskip \clabel[max_prime]{Maximální intenizta přímého osvětlení}
\picw=15cm \cinspic 04_Grafy/01_mereni/gr_02_MAX_spodni.png
\caption/f Maximální intenizita přímého osvětleni
%\medskip

\medskip \clabel[max_neprime]{Maximální intenzita nepřímého osvětlení}
\picw=15cm \cinspic 04_Grafy/01_mereni/gr_03_MAX_vrchni.png
\caption/f Maximální intenzita nepřímého osvětleni
\medskip
Z~grafů je patrné, že u~kombinovaného osvětlení jsou hodnoty v~luxech příliš vysoké a nejsou vhodné pro běžné použití.
U~přímého osvětlení jsou viditelné ostré píky v~modrém spektru, což není případ u~nepřímého osvětlení.
Hodnoty u~nepřímého mají vyšší hodnoty v~luxech.
\medskip
Měření s~minimální intenzitou světla bylo provedeno, avšak hodnoty se pohybovaly pod 10 lx. Vzhledem k~tomu,
že spektrometr má rozsah měření od 10 lx do 100 000 lx, se tyto hodnoty jevily spíše jako nesmyslné, a proto
nebyly použity pro další zpracování.

