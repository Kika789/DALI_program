\chap Návrh světelného ovládání

% \mnote{\inoval{DALI zapojeni}}

Poslední fáze praktické části se zaměřila na optimalizaci stávajícího světelného
ovládání s cílem zjednodušit jeho obsluhu a~zvýšit uživatelský komfort.

\medskip \clabel[ovl0]{Aktuální ovládání}
\picw=10cm \cinspic 03_Obrazky/ovladani.jpg
\caption/f Aktuální ovládání
\medskip

% Aktuálně je ovládání světel nastaveno tak, že každá řada světel reaguje na krátké nebo dlouhé stisknutí tlačítka.
% Krátké stisknutí způsobí buď zapnutí nebo vypnutí světel, zatímco dlouhé stisknutí slouží k~postupnému zvyšování
% nebo snižování jasu.
% Toto ovládání není ideální, proto bylo navrženo změnit ovládání na 4 tlačítka, spolu s dvěma tlačítky pro regulaci jasu.
% Tímto způsobem chceme vytvořit pohodlnější a~intuitivnější ovládání.

\sec Analýza stávajícího ovládání

V současnosti je systém vybaven 12 tlačítky, které slouží k~ovládání
jednotlivých řad světel (viz obrázek \ref[ovl0]).
Každé tlačítko ovládá jednu řadu světel a~reaguje na dva typy stisknutí:
\begitems
    * {\sbf Krátké stisknutí:} Zapíná nebo vypíná příslušné světelné zařízení.
    * {\sbf Dlouhé stisknutí:} Postupně zvyšuje nebo snižuje jas daného světelného zařízení.
\enditems

\medskip\noindent
Toto ovládání má svá omezení:
\begitems
    * {\sbf Složitost:} Velký počet tlačítek a~dva typy stisknutí pro každé tlačítko mohou být pro uživatele matoucí a~obtížně zapamatovatelné.
    * {\sbf Neintuitivní ovládání:} Postupné zvyšování a~snižování jasu dlouhým stisknutím tlačítka nemusí být pro uživatele intuitivní a~pohodlné.
\enditems


\sec Návrh optimalizovaného ovládání

Navrhované optimalizované řešení snižuje počet tlačítek na 4 a~doplňuje
je o~2~tlačítka pro regulaci jasu (viz obrázek \ref[tlac]).

% Nový navržený ovládací panel snižuje počet tlačítek na 4 a rozšiřuje je o 2 nová tlačítka
% pro nastavení jasu.



\medskip \clabel[tlac]{Tlačítka}
\picw=10cm \cinspic 03_Obrazky/Ovladani/tlacitka.png
\caption/f Tlačítka
\medskip


\noindent
Tlačítka jsou rozdělena do dvou sekcí:

\begitems
    * {\sbf Ovládání zapnutí/vypnutí:}
    \begitems \style o
        * {\sbf tlačítko 1:} slouží k~vypnutí všech světel,
        * {\sbf tlačítka 2--4:} slouží k~zapínání jednotlivých světelných scén.
    \enditems
    * {\sbf Regulace jasu:}
        \begitems \style o
        * {\sbf tlačítka 5--6:}  slouží k~postupnému zvyšování a~snižování jasu všech zapnutých světel.
        \enditems
\enditems

\medskip\noindent
Toto řešení přináší následující výhody:
\begitems
    * Menší počet tlačítek a~jednodušší typy stisknutí usnadňují pochopení a~ovládání systému.
    * Oddělení funkce zapnutí/vypnutí od regulace jasu umožňuje intuitivnější a~pohodlnější ovládání.
    % * Regulace jasu se týká všech zapnutých světel, čímž se eliminuje nutnost samostatného nastavování jasu pro každou sekci.
    * Regulace jasu ovládá všechna zapnutá světla najednou, takže není potřeba nastavovat jas pro každou sekci zvlášť.
\enditems

% Zaver
% Navrhované optimalizované ovládání světel se zaměřuje na zjednodušení a~zefektivnění obsluhy systému.
% Snížení počtu tlačítek, rozdělení funkcí a~intuitivní regulace jasu vedou
% k~pohodlnějšímu a~uživatelsky přívětivějšímu ovládání.
% Implementace tohoto řešení by tak mohla významně přispět
% ke zlepšení uživatelského komfortu a~zjednodušení ovládání světel v daném prostoru.



\sec Elektrické schéma

Elektrické schéma pro optimalizované ovládání světel je znázorněno na obrázku \ref[ovl2].
Schéma zahrnuje:
\begitems
    * Tlačítka pro zapnutí/vypnutí a regulaci jasu jsou připojena ke konektoru JP1.
    K~tlačítkům 5 a~6 jsou připojeny pull-down rezistory o hodnotě 10 kΩ,
    protože vstupní piny procesoru nemají integrované pull-down rezistory pro dané vstupy.
    * Tlačítka 1 až 4 mají invertovanou logiku, tj. při stisku se na vstupu procesoru objeví logická nula.
    * Tlačítka 5 a~6 mají neinvertovanou logiku, tj. při stisku se na vstupu procesoru objeví logická jednička.
\enditems

% Tlačítka určená pro regulaci jasu byla připojena k~pinům, které neobsahovaly pull-up rezistory.
% Proto bylo nezbytné přidat odpory o~h~dnotě 10~kΩ, aby bylo zajištěno správné fungování těchto tlačítek.



\medskip \clabel[ovl2]{Elektrické schéma}
\picw=10cm \cinspic 03_Obrazky/Ovladani/buttons.png
\caption/f Elektrické schéma tlačítek
\medskip


\sec Plošný spoj

Plošný spoj (\glref{PCB}) pro optimalizované ovládání světel je znázorněn na obrázku \ref[ovl1].
\glref{PCB} obsahuje všechny komponenty uvedené na schématu, včetně:
\begitems
    * {\sbf Napájecího konektoru:} Konektor pro připojení 5V napájení.
    * {\sbf Tlačítek:} Tlačítka pro zapnutí/vypnutí a~regulaci jasu.
    * {\sbf Pull-down rezistorů:} Rezistory 10 kΩ pro tlačítka regulace jasu.
\enditems

\noindent Rozměry desky plošného spoje jsou 91,5 x 30,5 mm.


\medskip \clabel[ovl1]{Plošný spoj}
\picw=10cm \cinspic 03_Obrazky/Ovladani/board2.png
\caption/f Plošný spoj
\medskip


\sec Řídíci jednotka

Pro řízení světel byla zvolena řídící jednotka s mikrokontrolérem \glref{ESP32} WROOM,
který se vyznačuje vysokým výkonem,
nízkým odběrem energie a~širokou škálou funkcí (viz obrázek \ref[pripojeni]).
Řídící jednotka je připojena k~plošnému spoji tlačítek pomocí konektoru JP2.
% Nabízí dostatek výpočetního výkonu pro náročné úlohy řízení světel pomocí protokolu DALI
% a zároveň umožňuje implementaci komplexních funkcí.


\medskip \clabel[pripojeni]{Řídící jednotka}
\picw=10cm \cinspic 03_Obrazky/rid_jed.jpg
\caption/f Řídící jednotka
\medskip



\sec Realizace řídicího systému

% Nově navržený panel nebylo nyní možné instalovat do učebny kvůli probíhající výuce na gymnáziu, neboť veškeré úpravy
% a~instalace ve školních prostorách probíhají mimo výukové období.
% Proto bylo veškeré zapojení provedeno a~otestováno v laboratorních podmínkách.

Montáž nově navrženého panelu do učebny musela být z provozních důvodů
odložena na období mimo výuku.
% Důvodem je probíhající výuka na gymnáziu, která vylučuje jakékoli úpravy
% a~instalace ve školních prostorách během školního roku.
Z tohoto důvodu proběhlo veškeré zapojení a~testování panelu v laboratorních podmínkách, byla použita LED světla
s~teplotou chromatičnosti 2700~K a 4000~K (viz ilustrační obr. \ref[dali_svetla]).

\medskip \clabel[dali_svetla]{Zapojení LED světel s protokolem DALI}
\picw=10cm \cinspic 03_Obrazky/DALI_svetla.png
\caption/f Zapojení LED světel s protokolem DALI \cite[dali_svetla2]
\medskip

\medskip\noindent
{\sbf Přípravné práce:}
\begitems
    * {\sbf Osazení plošného spoje tlačítek:}
        Nejprve byl plošný spoj osazen všemi součástkami, včetně rezistorů.
    * {\sbf Testování funkčnosti tlačítek:}
        Reakce řídící jednotky na stisknutí jednotlivých tlačítek byla ověřována pomocí
        testovacích programů, pro bezchybné fungování tlačítek a~jejich integraci s řídicím systémem.
\enditems

\medskip\noindent
{\sbf Zapojení systému:}
\begitems
    * {\sbf Propojení světel s řídící jednotkou:} Vodiče sběrnice DALI byly připojeny k~příslušným pinům řídící jednotky.
    * {\sbf Napájení řídicí jednotky:} Napájecí zdroje 12 V byly připojeny k~napájecím konektorům řídící jednotky.
    * {\sbf Zapojení světel do zásuvky}: Světla byla zapojena do zásuvky a~napájena síťovým napětím 230 V.
\enditems



\sec Vývojové prostředí

Jako vývojové prostředí pro programování řídící jednotky lze použít:

\begitems
    * {\sbf PlatformIO:} PlatformIO je open-source vývojové prostředí pro vývoj softwaru pro mikrokontroléry.
        Umožňuje programování v mnoha jazycích a~podporuje mnoho různých mikrokontrolérů~\cite[platformio_ide].
    * {\sbf Arduino IDE:} Arduino IDE je vývojové prostředí pro programování mikrokontrolérů z platformy Arduino.
        Umožňuje programování v jazyce C++~\cite[arduino_ide].
    * {\sbf ESP--IDF (Espressif IoT Development Framework):} ESP--IDF je oficiální vývojové prostředí pro mikrokontroléry ESP32.
        Umožňuje programování v jazyce C a~poskytuje širokou škálu funkcí pro vývoj softwaru~\cite[noauthor_espidf].
\enditems


Ukazuje se, že nejvhodnějším vývojovým prostředím pro programování řídící jednotky je PlatformIO,
protože je integrované do vývojového prostředí Visual Studio Code
s velmi pohodlným textovým editorem a~nabízí širokou škálu funkcí pro vývoj softwaru.
Nevýhodou je, že mnohé knihovny jsou portovány do prostředí často s výrazným zpožděním.
Popřípadě je nutné provést ruční konfiguraci knihoven, což může být značně časově náročné.

Prostředí Arduino IDE je původní vývojové prostředí pro mikrokontroléry z platformy Arduino
do kterého je možné nainstalovat knihovny pro ESP32.
Výhodou je jednoduchá instalace. Pokud je preferencí použít společný editor pro všechny projekty,
tak PlatformIO je vhodnější volbou a zahrnuje většinu funkcí Arduino IDE.

Framework ESP--IDF je vhodný pro uživatele, kteří potřebují využít pokročilé funkce mikrokontroléru ESP32.
Lze ho použít s libovolným textovým editorem, a~integrace s~Visual Studio Code je velmi dobrá.
% Pro využití všech funkcí mikrokontroléru je to nejlepší volba, ale pro jednoduché projekty to může být zbytečně složité.

Pro vývoj řídící jednotky bylo zvoleno vývojové prostředí PlatformIO, z důvodu, že byla
k dispozici jednoduchá knihovna pro komunikaci řídící jednotky se světly pomocí protokolu DALI verze 1.

% Pro případ, že by bylo nutné využít pokročilé funkce protokolu DALI verze 2, by bylo vhodné
% použít některou z knihoven pro ESP32, které jsou dostupné na internetu pro ESP--IDF.


%%%%%%%%%%%%%%%%%%%%%%%%%%%%%%%%%%%%%%%%%%%%%%%%%%%%%%%%%%%%

\sec Inicializace řídící jednotky:

\noindent
{\sbf Rozložení portů ESP32 WROOM:}


Rozložení portů je vidět na následujícím obrázku \ref[esp32_wroom].

\medskip \clabel[esp32_wroom]{Rozložení portů ESP32 WROOM}
\picw=14cm \cinspic 03_Obrazky/esp32_wroom.png
\caption/f Rozložení portů ESP32 WROOM \cite[esp32_wroom]
\medskip

\medskip\noindent
{\sbf Nastavení portů:}

V programu byly definovaný následující porty pro řídící jednotku:

\label[c_brightness]
\begtt\hisyntax{C}
#define DALI_TX_PIN  GPIO_NUM_32
#define DALI_RX_PIN  GPIO_NUM_33
#define LED_PIN      GPIO_NUM_2
#define PB1          GPIO_NUM_21
#define PB2          GPIO_NUM_19
#define PB3          GPIO_NUM_18
#define PB4          GPIO_NUM_17
#define PB5          GPIO_NUM_34
#define PB6          GPIO_NUM_35
\endtt

Inicializace portů byla provedena ve funkci {\tt setup()}.

\label[csetup]
\begtt\hisyntax{C}
void setup() {
    pinMode(LED_PIN, OUTPUT);
    digitalWrite(LED_PIN, LOW);

    pinMode(DALI_TX_PIN, OUTPUT);
    digitalWrite(DALI_TX_PIN, HIGH);

    pinMode(DALI_RX_PIN, INPUT);

    // buttons
    pinMode(PB1, INPUT_PULLUP);
    pinMode(PB2, INPUT_PULLUP);
    pinMode(PB3, INPUT_PULLUP);
    pinMode(PB4, INPUT_PULLUP);
    pinMode(PB5, INPUT);           // external pull-down
    pinMode(PB6, INPUT);           // external pull-down
    ...
 }
\endtt


\medskip\noindent
{\sbf Seznam příkazů pro práci s řídící jednotkou:}

Funkce PrintHelp() vypíše seznam používaných příkazů v kódu řídící jednotky.

\medskip
\label[c_help]
\begtt\hisyntax{C}
void PrintHelp() {
    Serial.println();
    Serial.println("Help");
    Serial.println("------------------");
    Serial.println("-1 = broadcast");
    Serial.println("(Switch Off) off: device");
    Serial.println("(Set Max Level) smax: device light level");
    Serial.println("(Set Min Level) smin: device light level");
    Serial.println("(Set System Failure Level) sfail:
                     device light level");
    Serial.println("(Set Power On Level) spower: device light level");
    Serial.println("(Set Fade Time) sft: device time");
    Serial.println("(SetFadeRate) sfr: device rate");
    Serial.println("(Set Short Address) ssa: device number");
    Serial.println("(Set Scene) ss: device number_of_scene light level");
    Serial.println("(Go To Scene) gts: device number_of_scene");
    Serial.println("(Identify Device) id: device");
    Serial.println("(Add To Group) atg: device number_of_group");
    Serial.println("(Remove From Group) rfg: device number_of_group");
    Serial.println("(Set Operating Mode) som: device
                     number_of_mode(hex) (0 default)");
    Serial.println();
}
\endtt


\medskip\noindent
{\sbf Konfigurace adres:}
\begitems
    * Inicializace zařízení a adres
      byla provedena postupem uvedeným v kapitole \ref[dali_inicializace].
      Každému světlu byla přiřazena unikátní adresa v rámci komunikační sítě.
      Základní ověření funkčnosti a identifikace byla ověřena příkazem IDENTIFY.
      Při správném nastavení se zařízení rozblikalo na 10 sekund.
    * V tomto případě byla světla nastavena na adresy v rozmezí 0 až 3, čímž
        se vytvořila adresní mapa pro komunikaci mezi mikrokontrolérem a~jednotlivými světly.
\enditems


% \medskip
% \noindent
% {\sbf Implementace scén:}
% \begitems
%     * Pro tlačítka byly vytvořeny scény, které definují chování světel při stisknutí daného tlačítka.
%         Scény zahrnují funkce jako zapnutí/vypnutí a~snižování/zvyšování jasu.
%     * Implementace scén byla provedena v programovacím prostředí pro mikrokontrolér \glref{ESP32} WROOM
%         s využitím jazyka C++ a~knihoven pro komunikaci a~ovládání světel.
% \enditems



%%%%%%%%%%%%%%%%%%%%%%%%%%%%%%%%%%%%%%%%%%

% postup nastavování:
% * přidání jednotuvých zařízení do kupin
% * poté bylo na skupinu nastavená daná scéna
% *

%     • Nebude, žádný tam není. Jsou tam jen vypínače připojené na diming vstupy předřadníků.
%     •

%     • zapojila jsem světla do zásuvky
%     • zapojila světla do řídící jednotky
%     • a~zdroj 12~V
%     • dokumentace řídicí jednotky
%     • byla pouzit protokol DALI
%     • ridici jednotka s~\glref{ESP32}

%     • na začátku bylo zapotřebí zvolit správné piny pro ovládání světel a~tlačítek
%     • dále jsme museli nastavit adresy jednotlivým světelným zařízením - pomcí inicialize a~nebo postupným odpojováním/zapojováním
%         světel a~pomocí fce set short adress - nastavime pomoci broadcast
%     • použité fce: on, off, set max level, set min level, set systém failure level, set power on level, set fade time,
%         set fade rate, set short adress, set scene, go to scene, identify device, add to group, remove from group,
%         set operating mode
%     • dále jsme si mohli spravne nastaveni adres overit fci identify device, při spravnem nastaveni svetla zacalo
%         blikat po dobu 10s
%     • v~mem pripade mam adresy nastavene od 0-3??? overit
%     • pote jsem nastavila groupy, sceny a~dale priradila jednotlive sceny ke konkretnim tlacitkum,
%     • tlacitko off - slouzi k~uplenemu vypnuti
%     • tlacitko vyklad
%     • tlacitko pisemka
%     • tlacitko denní svetlo
%     • tlacitko regulace jasu, kdy při dlouhem stisku se intenzita zvysovala/snizovala
%     • při zmene jasu, se při stiknuti tlacitka s~konkretni scenou, vratila scena do puvodniho stavu
%     • pro budoucí implemenatci jsou sceny navrzeny dle normy viz simulace
%     • pro laboratorni ulohy, abychom navrzene zapojeni mohli overit byly zvoleny tyto scény,
%     • scena off
%     • scena vyklad, svetla prima vypla nahore zapla
%     • scena pisemnka: prima zapla, horni zapla s~mensi intenzitou
%     • scena denní svetlo, svetal s~ruznou intenzitou od nejnizsi po nejvyssi
%     • tlacitka na regulaci jasu
%     • vyvojove diagamy s~logikou tlacitek: při kazdem stisku se zapne konkretni scena, u~regulaci jasu,
%         je~pouzit cas, kde při stisku se jas zvysuje nebo snizuje
%     • zapojeni bylo v~laboratori vyzkouseno a~funguje
%     • zapojeni v~praxi nelze nyní uskutecnit, protože vsehcny instalace ve skolach probihaji mino vyukove obdobi
%     •


%%%%%%%%%%%%%%%%%%%%%%%%%%%%%%%%%%%%%%%%%%%%%%%%%%%%%%%%%
% \medskip
% \medskip
% \medskip
% V~učebně je momentálně instalováno dvanáct tlačítek, která způsobují potíže běžným uživatelům.
% Z~navrhovaných scén bylo vybráno implementovat tlačítko pro úplné vypnutí všech světel, tlačítko pro scénu výuky,
% písemka a~denní osvětlení. Dále jsem přidala tlačítko pro regulaci jasu, aby uživatelé měli možnost
% přizpůsobit si osvětlení podle svých preferencí v~případě výrazných změn denního světla.

% Vzhledem k~tomu, že mi nebylo umožněno změnit ovládací panel v~učebně během školního roku. Ovládací zařízení jsem navrhla
% a~zapojila v~laboratorních podmínkách. Jako řídící jednotka byla zvolena ESP3 (viz dokumentace).
% Pro zapojení byly vybrány piny podle schématu. Jedná se o~digitální porty, a~k~analogovým prvkům byly
% přidány dva odpory pro vytvoření pull-up rezistoru. Logika tlačítek byla navržena tak, že každým stiskem
% se aktivuje odpovídající scéna. U~tlačítek pro regulaci jasu je nutné stisk držet déle, aby došlo ke změně jasu.

% Postup zapojení:
% \begitems
%     * zapojila jsem světla do zásuvky
%     * zapojila světla do řídící jednotky
%     * a~zdroj 12
% V~\enditems
% \medskip
% Postup implementace
% \medskip
% Nastaveni pinu

% \medskip
% \picw=8cm \cinspic 03_Obrazky/koood.png
% % \caption/f Učebna
% \medskip

% Dale bylo zapotrebi implementovat fucnkce - prikazy

% void setup, void loop, priklad funkce, diagram logiky tlacitka

% % \sec Laboratoř

% Postup pro nastavování scén:

% - nastavení set failure system = 180
% - nastavení set power on = 200
% - set max level = 254
% - set min level = 85
% - set fade time
% - set rate time
% - add to group číslo zařízení číslo skupiny
% - set scene č. zařízení číslo scén hodnota jasu


% Popis kroků pro nastavení scén:

\medskip\noindent
{\sbf Nastavení výchozích parametrů:}
\begitems
    * {\sbf set failure system (180):}
        Nastaví úroveň jasu světelných zařízení při výpadku řídící jednotky.
        Hodnota 180 přibližně odpovídá {70~\pcent} maximálně dosažitelného jasu.
    * {\sbf set power on (200):} Nastaví úroveň jasu při zapnutí zařízení.
        Hodnota 200 přibližně odpovídá {80~\pcent} maximálně dosažitelného jasu.
    * {\sbf set max level (254):} Nastaví maximální dosažitelný jas.
        Hodnota 254 odpovídá {100~\pcent} maximálně dosažitelného jasu.
    * {\sbf set min level (85):} Nastaví minimální dosažitelný jas.
        Hodnota 85 přibližně odpovídá {34~\pcent} maximálně dosažitelného jasu.
    * {\sbf set fade time:} Nastaví čas plynulého rozsvícení/zhasnutí při změně jasu.
       Dobu je možné vypočítat podle vzorce \ref[eq_fade_time].
      \label[eq_fade_time]
      $$ Fade\;time = 2^{({fade_{time} \over 2} -1)} \eqmark $$
      kde $fade_{time}$ je hodnota v~rozsahu 1 až 15.
      Minimální doba je 0,7~s a~maximální doba je 90,5~s (viz. tabulka~4 v~\cite[iec102]).

    * {\sbf set rate time:} Nastaví rychlost plynulé změny jasu na požadovanou úroveň.
    Rychlost je možné vypočítat podle vzorce \ref[eq_fade_rate].
    \label[eq_fade_rate]
    $$ Fade\;rate = {506 \over {2^{({fade_{rate} \over 2})}}} \eqmark $$
    kde $fade_{rate}$ je hodnota v~rozsahu 1 až 15.
    Minimální rychlost je 2,8~kroků/s a~maximální rychlost je 358~kroků/s (viz. tabulka~5 v~\cite[iec102]).
\enditems


\medskip\medskip
\noindent
V kódu byly vytvořeny funkce pro přidání zařízení do skupin a nastavovaní scén.

\begitems
    * Funkce {\sbf  AddToGroup(dev, grp):} Kde $dev$ je číslo zařízení a $grp$ je číslo skupiny.
        Přidá zadané zařízení do vybrané skupiny.
        To umožňuje ovládat více zařízení najednou.

\label[c_addtogroup]
\begtt\hisyntax{C}
void AddToGroup (int value, int group) {
    uint8_t device = parse_device(value);
    DaliTransmitCMD(device, 0x60+(group&0x0F));
    delayMilisec(DALI_TWO_PACKET_DELAY);

    DaliTransmitCMD(device, 0x60+(group&0x0F));
    delayMilisec(DALI_TWO_PACKET_DELAY);
} // add to group
\endtt

    * Funkce {\sbf SetScene(dev, scene, level):}
        Nastaví úroveň jasu ($level$) pro zadanou scénu ($scene$) a zadané zařízení
        ($dev$).

\label[c_setscene]
\begtt\hisyntax{C}
void SetScene(int value, int numberScene, int levelScene) {
    uint8_t device = parse_device(value);
    DaliTransmitCMD(0xA3, levelScene);
    delayMilisec(DALI_TWO_PACKET_DELAY);

    DaliTransmitCMD(device, 0x40+(numberScene&0x0F));
    delayMilisec(DALI_TWO_PACKET_DELAY);

    DaliTransmitCMD(device, 0x40+(numberScene&0x0F));
    delayMilisec(DALI_TWO_PACKET_DELAY);
} // set scene
\endtt

\enditems


\sec Nastavení scén

\noindent
{\sbf Pro laboratorní účely byly nastaveny tyto scény:}
\begitems
    * Tlačítko {\sbf Off:} Slouží k~úplnému vypnutí světel.
    * Tlačítko {\sbf Výklad:} Dva světelné prvky jsou zapnuté a~dva vypnuté.
    * Tlačítko {\sbf Písemka:} Dva světelné prvky jsou nastaveny na vyšší intenzitu a~dva na nižší intenzitu.
    * Tlačítko {\sbf Denní světlo:} Každý prvek je nastaven s jinou intenzitou -- od nejnižší po nejvyšší podle vzdálenosti od oken.
    * Tlačítka {\sbf Regulace jasu:} Při krátkém nebo dlouhém stisku se intenzita postupně zvyšuje/snižuje.
\enditems



\noindent {\sbf Popis kroků pro nastavení scén:}

\medskip\noindent
{\sbf První scéna} - dva světelné prvky s adresou 1 a 2 byly nastavené na intenzitu jasu 200 a dva zbylé byly vypnuté:

\begitems \style n
* {\sbf Přidání} světel 1 a 2 do skupiny 1:
    \begitems \style o
        * {\tt AddToGroup(1,1)}
        * {\tt AddToGroup(2,1)}
    \enditems
* {\sbf Vypnutí} všech světel ve scéně, výchozí stav:
    \begitems \style o
        * {\tt SetScene(-1,1,0)}
    \enditems
% * set scene(-1,1,0) - nastavení světelné scény pro groupu 1 na hodnotu jasu~0 - tato funkce je volána z důvodu vymazání předešlé scény
* {\sbf Nastavení} světel 1 a 2 na hodnotu jasu 200 pomocí skupiny 1 ve scéně 1:
    \begitems \style o
        * {\tt SetScene(101,1,200)}
    \enditems
% * set scene(101,1,200) - nastavení světelné scény pro groupu 1 na hodnotu jasu~200
\enditems

\medskip \clabel[scena01]{Scéna 1}
\picw=6cm \cinspic 03_Obrazky/Ovladani/01a.jpg
\caption/f Scéna 1
\medskip

\noindent
{\sbf Druhá scéna} - dva světelné prvky s adresou 0 a 2 byly nastavené na vyšší intenzitu jasu 200
a~dva zbylé (adresa 1 a 3) byly na nižší intenzitu jasu 120:

% \medskip\noindent {\sbf Postup} nastavení 2. scény:
\begitems \style n
* {\sbf Nastavení} světel 0 a 2 na hodnotu jasu 200
    \begitems \style o
        * {\tt SetScene(0,2,200)}
        * {\tt SetScene(2,2,200)}
    \enditems
* {\sbf Nastavení} světel 1 a 3 na hodnotu jasu 120
    \begitems \style o
        * {\tt SetScene(1,2,120)}
        * {\tt SetScene(3,2,120)}
    \enditems

% * set scene(0,2,200) - nastavení intenzity jasu pro zařízení s krátkou adresou~0
% * set scene(1,2,120) - nastavení intenzity jasu pro zařízení s krátkou adresou~1
% * set scene(2,2,200) - nastavení intenzity jasu pro zařízení s krátkou adresou~2
% * set scene(3,2,120) - nastavení intenzity jasu pro zařízení s krátkou adresou~3
\enditems

\medskip \clabel[scena02]{Scéna 2}
\picw=6cm \cinspic 03_Obrazky/Ovladani/02b.jpg
\caption/f Scéna 2
\medskip

\noindent
{\sbf Třetí scéna} - každý světelný prvek byl nastaven na jinou hodnotou jasu:

% \medskip\noindent {\sbf Postup nastavení 3. scény:}

\begitems \style n
* {\sbf Nastavení} světel 0 až 3 na hodnoty jasu 85, 240, 130 a 180:
    \begitems \style o
        * {\tt SetScene(0, 3, 85)}
        * {\tt SetScene(1, 3, 240)}
        * {\tt SetScene(2, 3, 130)}
        * {\tt SetScene(3, 3, 180)}
    \enditems

% * set scene(0,2,85) - nastavení intenzity jasu na 85 pro zařízení s krátkou adresou~0
% * set scene(1,2,240) - nastavení intenzity jasu na 240 pro zařízení s krátkou adresou~1
% * set scene(2,2,130) - nastavení intenzity jasu na 130 pro zařízení s krátkou adresou~2
% * set scene(3,2,180) - nastavení intenzity jasu na 180 pro zařízení s krátkou adresou~3
\enditems

\medskip \clabel[scena03]{Scéna 3}
\picw=6cm \cinspic 03_Obrazky/Ovladani/03c.jpg
\caption/f Scéna 3
\medskip

\noindent U scény 2 a 3 nebylo zapotřebí vytvářet skupiny, protože každému zařízení jsme nastavili přímo danou hodnotu jasu.

\noindent {\sbf Nastavení tlačítka off}

 Tlačítko {\tt Off} slouží k úplnému vypnutí světel

\label[c_off]
\begtt\hisyntax{C}
void Off(int value){
    uint8_t device = parse_device(value);
    DaliTransmitCMD(device, OFF_C);
    delayMilisec(DALI_TWO_PACKET_DELAY);
    Serial.println("Off");
} // off
\endtt

\noindent {\sbf Nastavení tlačitek pro regulaci jasu}

Tlačítka regulace jasu slouží k plynulému zvyšování nebo snižování intenzity světla.
Po stisknutí a podržení tlačítka se jas bude pomalu a plynule zvyšovat nebo snižovat,
dokud se tlačítko nepustí. Část kódu pro zvyšování a snižování jasu je uvedena níže.

\label[c_brightness]
\begtt\hisyntax{C}
// brightness increase using the button --------------
if(active_group > 0 && button6_new == PB_ON){
  actual_time = millis();
  delta_time = actual_time - time_b2;
  if(delta_time > 30){
    StepUp(100 + active_group);
    time_b2 = actual_time;
  }
}

// brightness decrease using the button --------------
if(active_group > 0 && button5_new == PB_ON){
  actual_time = millis();
  delta_time = actual_time - time_b3;
  if(delta_time > 30){
    StepDown(100 + active_group);
    time_b3 = actual_time;
  }
}
\endtt

% \medskip\noindent
% {\sbf Inicializace řídící jednotky:}

% \medskip \clabel[sceny]{Realizované scény}
% \picw=16cm \cinspic 03_Obrazky/sceny.png
% \caption/f Realizované scény
% \medskip

\sec Shrnutí
\begitems \style o
* ověřili
* komentář
* dává to smysl to tam instalovat?
* přidat obrázek ESP32 s piny
\enditems

Cílem bylo navrhnout intuitivní ovládání pro regulaci světelného systému. Zvolili jsme navrhnout
tři tlačítka, kterýma se bodou zapinat konkrétní scény, pro komunkaci se světly, byl zvolen sběrnicový systém DALI.
Ze začatku byla ověřována funkčnost řídíci jednotky a poté tlačítka. Po úspšném testování byly implementovány scény
pomocí příkazových funkcí.

Měli jsme k dispozici čtyři světelná zařízení v laboratorních podmínkách, která umí komunikaci DALI,
byly propojeny pomocí svorkovnice a každému byla přidělena unikátní adresa 0 až 3. Správnost adres byla otestována pomíci
příkazu IDENTYFY, kde se světelné zařízení rozblikalo na 10s.

%%%%%%%%%%%%%%%%%%%%%%%%%%%%%%%%%%%%%%%%%%%%%%%%%%%%%%%
% \sec Návrh ovládacího panelu

% Navrhla jsem ovládací panel, aby bylo jednodušší ovládání pro uživatelé.

% \medskip
% Aktuálně je v učebně ovládací panel s 12 tlačítky. Navrhla jsem 3 hlavní scény, které se budou zapínat 3 tlačítky, dále je přidáno tlačítko off na ṕlné vypnutí a~také jsem přidala tlačítko na regulaci intenzity jasu. Kdyby docházelo k~nečekaným změnám venkovního počasí.

% Tlačítko výklad, slouží k~zapnutí nepřímého osvětlení a~vypnutí první světelné řady u tabule.

% Tlačítko písemka umožňuje nasvícení místa úkonu na 500 lx, abychom dodržovali zásady normy.

% Tlačítko adaptace na denní světlo rozsvicuje řadu světel u dveří a~a postupnou změnu intezity

% \medskip
% Simulace

% \medskip
% jasova analyza