\sec Analýza osvětlenosti v~horizontální rovině prostoru
Třetí měření proběhlo 6.3.2024 v~době od 17:00 do 18:30 za tmy.
Byla změřena horizontální rovina osvětlenosti při různých světelných scénách.
Celkem bylo zvoleno devět světelných scén a~změřeno 18 vybraných měřicích bodů ve výšce 850~mm.
Scény byly tentokrát nastavovány po jednotlivých
řadách postupně od dveří k~oknu a~byly změřeny tři režimy osvětlení při maximální intenzitě.\noindent

\medskip\noindent
{\sbf Maximální intenzita nepřímého osvětlení:}
\begitems \style n
    * Rozsvícená řada jen u~dveří.
    * Rozsvícená řada jen ve středu.
    * Rozsvícená řada jen u~okna.
\enditems

\medskip\noindent
{\sbf Maximální intenzita kombinovaného osvětlení:}
\begitems \style n
    * Rozsvícená řada jen u~dveří.
    * Rozsvícená řada jen ve středu.
    * Rozsvícená řada jen u~okna.
\enditems


\medskip\noindent
{\sbf Maximimální intenzita přímého osvětlení:}
\begitems \style n
    * Rozsvícená řada jen u~dveří.
    * Rozsvícená řada jen ve středu.
    * Rozsvícená řada jen u~okna.
\enditems

\noindent Na obrázku \ref[pud] je žlutou barvou zobrazena měřicí horizontální rovina v učebně.

\medskip \clabel[pud]{Půdorys s měřicí rovinou}
\picw=8cm \cinspic 04_Grafy/03_mereni/rovine.png
\caption/f Půdorys s měřicí rovinou
\medskip

Na obrázku \ref[svet_mapa0] vidíme světelnou mapu zapnutých všech svítidel na maximální intenzitě.
Z hodnot isofotů je patrné, že osvětlenost je nejnižší u~oken a~nejvyšší uprostřed místnosti.
% abychom měli představu o chování světla v celém prostoru.

\medskip \clabel[svet_mapa0]{Světelná mapa kombinovaného osvětlení}
\picw=8cm \cinspic 04_Grafy/03_mereni/max_01_MAX_kombi_max_2d.png
\caption/f Světelná mapa kombinovaného osvětlení
\medskip

\secc[vysledky_mereni_horizontalni] Výsledky měření

\medskip \clabel[svet_mapa]{Světelné mapy kombinovaného osvětlení}
\picw=15cm \cinspic 04_Grafy/03_mereni/x_kombi.jpg
\caption/f Světelné mapy kombinovaného osvětlení - okno, střed, dveře
\medskip

\medskip \clabel[svet_mapa2]{Světelné mapy nepřímého osvětlení}
\picw=15cm \cinspic 04_Grafy/03_mereni/x_horni.jpg
\caption/f Světelné mapy nepřímého osvětlení - okno, střed, dveře
\medskip

\medskip \clabel[svet_mapa3]{Světelné mapy přímého osvětlení}
\picw=15cm \cinspic 04_Grafy/03_mereni/x_dolni.jpg
\caption/f Světelné mapy přímého osvětlení - okno, střed, dveře
\medskip

% Z~grafů vidíme přínos osvětlení z~jednotlivých řad. % -------- Jak toto hodnotit?

\secc Analýza měření

Grafy ukazují rozložení osvětlení v učebně. Nejprve s maximálním výkonem všech světel a~poté s kombinací,
nepřímým a~přímým osvětlením po jednotlivých řadách.


% Z grafů vidíme rozložení osvětlení v učebně při zapnutí všech svítidel na maximum a poté po jednotlivých řadách
% kombinované, nepřímé a~přímé osvětlení.

Měření bylo provedeno za účelem stanovení příspěvku jednotlivých světelných řad k~celkovému osvětlení prostoru.
% Toto měření bylo provedeno z důvodu, abychom věděli jak jednotlivé řady přispívají do prostoru se svým světlem.

\medskip \noindent {\sbf Graf \ref[svet_mapa0]: Světelná mapa kombinovaného osvětlení}
\begitems
  * První graf zobrazuje rozložení osvětlenosti v učebně při zapnutí všech světel na maximum. Je zřejmé, že takto vysoké hodnoty luxů
    (nad 1000 luxů) nejsou vhodné pro běžné používání, protože můžou snižovat kontrast na interaktivní tabuli a~způsobovat nadměrnou
    spotřebu energie.
    Osvětlenost u okna je nižší než jinde, protože jsou zatažená černými látkovými roletami,
    které absorbují světlo které blokují přirozené denní světlo.
    V reálných podmínkách, kdy rolety nebudou zataženy, se osvětlenost v~této oblasti výrazně zvýší.
\enditems


\medskip \noindent {\sbf Graf \ref[svet_mapa]: Světelné mapy kombinovaného osvětlení - okno, střed, dveře}

% Vidíme, že osvětlení je nejvyšší u okna a~dveří a~nejnižší uprostřed místnosti.
% To je pravděpodobně způsobeno tím, že přímé osvětlení z okna a~dveří dosahuje dále než nepřímé osvětlení
% odrážející se od stěn a~stropu.

% Z těchto grafů vidíme, že jednotlivé řady přispívají stějně do prostoru, což znamená že jsou správně nastaveny.
\begitems
    * Na základě grafů můžeme usuzovat, že všechny řady svítidel přispívají k osvětlení učebny relativně rovnoměrně.
        To naznačuje, že jejich umístění a~nastavení jsou pravděpodobně správné.
\enditems

\medskip \noindent {\sbf Graf \ref[svet_mapa2]: Světelné mapy nepřímého osvětlení - okno, střed, dveře}

% Vidíme, že osvětlení je relativně rovnoměrně rozloženo po celé místnosti. To je pravděpodobně způsobeno tím,
% že nepřímé osvětlení se odráží od stěn a~stropu.Což je velmi příjemné pro požívání a takto se světlo chová ve skutečnosti i v přirodě.
% Rozptýlené, jak se odráží od strupu a stěn.
\begitems
    * Grafy ukazují, že v učebně je osvětlení relativně rovnoměrné rozložené.
    To je dosaženo díky nepřímému způsobu osvětlení, kdy se světlo ze svítidel odráží od stěn a~stropu.
    Výhoda tohoto typu osvětlení je, že napodobuje přirozené rozptýlené světlo, které vnímáme z~venkovního prostřední
    a~vytváří tak přirozený vzhled, příjemnou a~komfortní atmosféru, která je méně namáhavá pro zrak.

    * Díky velké odrazné ploše má nízký plošný jas než přímé osvětlení, čímž se minimalizuje oslnění, což je také důležité pro
    studenty v učebnách, kde tráví dlouho čas soustředěnou práci.
    Dále nepřímé světlo vytváří měkké a rozptýlené světlo, proto nevzniká tolik nerovnoměrností v prostoru.

    * Vidíme, že nepřímé osvětlení je ideálním řešením pro učebny. Poskytuje nám rovnoměrné,
    komfortní a zdravé osvětlení, které podporuje koncentraci a učení studentů.
\enditems

\medskip \noindent {\sbf Graf \ref[svet_mapa3]: Světelné mapy přímého osvětlení - okno, střed, dveře}
\begitems
  *  Z grafů je patrné, že přímé osvětlení do prostoru přispívá méně. Je to způsobeno několika faktory.
    Pro přímé osvětlení se používá světlo s nižším světelným výkonem, aby nedocházelo k oslnění.
    % Přímé osvětlení má slabší intenzitu než nepřímé, to je záměrné, aby nedocházelo k oslnění.
    Dále přímé osvětlení slouží spíše jako doplňkové osvětlení k~nepřímému, aby se osvětlila určitá místa v učebně,
    jako jsou stoly studentů, katedra nebo tabule. Dalším důvodem je, že se přímé osvětlení odráží
    méně než nepřímé, což je způsobeno jeho úhlem dopadu.

  *  I když je přínos přímého osvětlení menší, jeho kombinace s nepřímým osvětlením je vhodná pro dosažení
    optimálního osvětlení.

    % Přímé osvětlení zajistí dostatečnou intenzitu v klíčových oblastech a podporuje soustředění studentů.

  *Přímé osvětlení má vyšší účinnost, dosahuje cílové osvětlenosti s~menším instalovaným výkonem, a~proto se používá častěji než
  nepřímé osvětlení. Přímé osvětlení je směrovější, vytváří prostředí s~většími kontrasty mezi světlem a~stínem. Pokud kontrasty
  nejsou příliš výrazné, může to prostor oživit, zatímco nepřímé osvětlení může působit jednotvárně. Je to jako rozdíl mezi
  zataženou oblohou s~difúzním světlem a~slunečným dnem s~přímým, směrovým zářením.

% Přímé osvětlení má vyšší účinnost, cílovou osvětlenost na ploše dosáhne s menším instalovaným výkonem,
% proto se používá častěji než nepřímé. Přímé je směrovější než nepřímé, vytváří prostředí s většími kontrasty
% (světlo x stín). Když nejsou přehnaně veliké, tak to může být v prostoru oživující, zatímco nepřímé osvětlení může být fádní.
% Jako když máte zataženou oblohu (difúzní světlo) a slunečný den (směrové , přímé záření).

\enditems


% Vidíme, že příme sovětlení příspívá méně do prostoru, z důvodu, že světla jsou slabší, slouží jako doplňkove k neprimemu a nema
% moznost se osrazet od okolniho prostredi.

% Shrnutí:

% Tento graf ukazuje rozložení osvětlení v učebně při přímém osvětlení. Vidíme, že osvětlení je nejvyšší
% u okna a~dveří a~nejnižší uprostřed místnosti. To je v souladu s grafem 1.

% Z této analýzy jsme chtěli zjistit, jak jednotlivé řady přispívají k~osvětlení v učebně. Z grafů vidíme:

% - nejvíce přispívá nepřímé osvětlení
% - kombinované osvětlení je součet přímého a~nepřímého osvtělení
% - přímé osvtělení přispívá do prostoru jako pruh

% Světelná mapa kombinovaného osvětlení - okno, střed, dveře
%     • 400 – 600 lx

% Světelná mapa nepřímého osvětlení - okno, střed, dveře
%     • 200 – 300 lx

% Světelná mapa přímého osvětlení - okno, střed, dveře
%     • 100 – 300 lx

%     \medskip zvetsit fonty u izofotu