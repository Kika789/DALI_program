\sec Třetí měření
Třetí měření proběhlo 6.3.2024 v době od 17:00 do 18:30 za tmy.
Byla změřena horizontální rovina při různých světelných scénách.
Celkem bylo zvoleno devět světelných scén a~změřeno 18 vybraných bodů.
Scény byly tentokrát nastavovány po jednotlivých
řadách postupně od dveří k~oknu a~byly změřeny tři režimy osvětlení při maximální intenzitě.\noindent

\medskip\noindent
{\sbf Maximální intenzita nepřímého osvětlení:}
\begitems \style n
    * Řada u~dvěří.
    * Řada uprostřed.
    * Řada u~okna.
\enditems

\medskip\noindent
{\sbf Maximální intenzita kombinovaného osvětlení:}
\begitems \style n
    * Řada u~dvěří.
    * Řada uprostřed.
    * Řada u~okna.
\enditems


\medskip\noindent
{\sbf Maximimální intenzita přímého osvětlení:}
\begitems \style n
    * Řada u~dvěří.
    * Řada uprostřed.
    * Řada u~okna.
\enditems

\medskip \clabel[svet_mapa]{Světelná mapa kombinovaného osvětlení}
\picw=18cm \cinspic 04_Grafy/03_mereni/kombinovane.png
\caption/f Světelná mapa kombinovaného osvětlení - okno, střed, dveře
\medskip

\medskip \clabel[svet_mapa2]{Světelná mapa nepřímého osvětlení}
\picw=18cm \cinspic 04_Grafy/03_mereni/neprime.png
\caption/f Světelná mapa nepřímého osvětlení - okno, střed, dveře
\medskip

\medskip \clabel[svet_mapa3]{Světelná mapa přímého osvětlení}
\picw=18cm \cinspic 04_Grafy/03_mereni/prime.png
\caption/f Světelná mapa přímého osvětlení - okno, střed, dveře
\medskip

% Z~grafů vidíme přínos osvětlení z~jednotlivých řad. % -------- Jak toto zhodnotit?

\medskip\noindent
{\sbf Zhodnocení grafů}

Z grafů vidíme rozložení osvětlení v učebně při kombinovaném, nepřímém a~přímém osvětlení z jednotlivých řad.

Graf 1: Světelná mapa kombinovaného osvětlení

Vidíme, že osvětlení je nejvyšší u okna a~dveří a~nejnižší uprostřed místnosti.
To je pravděpodobně způsobeno tím, že přímé osvětlení z okna a~dveří dosahuje dále než nepřímé osvětlení
odrážející se od stěn a~stropu.

Graf 2: Světelná mapa nepřímého osvětlení

Vidíme, že osvětlení je relativně rovnoměrně rozloženo po celé místnosti. To je pravděpodobně způsobeno tím,
že nepřímé osvětlení se odráží od stěn a~stropu.

Graf 3: Světelná mapa přímého osvětlení

Tento graf ukazuje rozložení osvětlení v učebně při přímém osvětlení. Vidíme, že osvětlení je nejvyšší
u okna a~dveří a~nejnižší uprostřed místnosti. To je v souladu s grafem 1.

Z této analýzy jsme chtěli zjistit, jak jednotlivé řady přispívají k~osvětlení v učebně. Z grafů vidíme:

- nejvíce přispívá nepřímé osvětlení
- kombinované osvětlení je součet přímého a~nepřímého osvtělení
- přímé osvtělení přispívá do prostoru jako pruh

Světelná mapa kombinovaného osvětlení - okno, střed, dveře
    • 400 – 600 lx

Světelná mapa nepřímého osvětlení - okno, střed, dveře
    • 200 – 300 lx

Světelná mapa přímého osvětlení - okno, střed, dveře
    • 100 – 300 lx