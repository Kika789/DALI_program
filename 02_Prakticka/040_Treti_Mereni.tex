\sec Třetí měření
Třetí měření proběhlo 6.3.2024 - 17:00 - 18:30 za tmy. Byla změřena horizontální rovina při různých světelných scénách.
Celkem bylo zvoleno devět světelných scén a~změřeno 18 vybraných bodů. Scény byly tentokrát nastavovány po jednotlivých
řadách postupně od dveří k~oknu a~byly změřeny tři režimy osvětlení při maximální intenzitě.
\medskip
{\sbf Maximální intenzita nepřímého osvětlení}
\medskip
\begitems \style n
    * Řada u~dvěří.
    * Řada uprostřed.
    * Řada u~okna.
\enditems

\medskip
\begitems \style n
{\sbf Maximální intenzita kombinovaného osvětlení}
\medskip
    * Řada u~dvěří.
    * Řada uprostřed.
    * Řada u~okna.
\enditems
\medskip

\begitems \style n
{\sbf Maximimální intenzita přímého osvětlení}
\medskip
    * Řada u~dvěří.
    * Řada uprostřed.
    * Řada u~okna.
\enditems

\medskip \clabel[svet_mapa]{Světelná mapa kombinovaného osvětlení}
\picw=18cm \cinspic 04_Grafy/03_mereni/kombinovane.png
\caption/f Světelná mapa kombinovaného osvětlení - okno, střed, dveře
\medskip

\medskip \clabel[svet_mapa2]{Světelná mapa nepřímého osvětlení}
\picw=18cm \cinspic 04_Grafy/03_mereni/neprime.png
\caption/f Světelná mapa nepřímého osvětlení - okno, střed, dveře
\medskip

\medskip \clabel[svet_mapa3]{Světelná mapa přímého osvětlení}
\picw=18cm \cinspic 04_Grafy/03_mereni/prime.png
\caption/f Světelná mapa přímého osvětlení - okno, střed, dveře
\medskip

Z~grafů vidíme přínos osvětlení z~jednotlivých řad. -------- Jak toto zhodnotit?

\medskip
Z této analýzy jsme chtěli zjistit, jak jednotlivé řady přispívají k osvětlení v učebně. Z grafů vidíme:

- nejvíce přispívá nepřímé osvětlení
- kombinované osvětlení je součet přímého a~nepřímého osvtělení
- přímé osvtělení přispívá do prostoru jako pruh

Světelná mapa kombinovaného osvětlení - okno, střed, dveře
    • 400 – 600 lx

Světelná mapa nepřímého osvětlení - okno, střed, dveře
    • 200 – 300 lx

Světelná mapa přímého osvětlení - okno, střed, dveře
    • 100 – 300 lx