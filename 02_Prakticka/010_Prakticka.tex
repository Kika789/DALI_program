\chap Praktická část

V~praktické části diplomové práce byl proveden důkladný průzkum prostředí výukové místnosti na gymnáziu u~Libeňského zámku,
který byl zaměřen na optimalizaci osvětlení pro výuku biologie.
Prvním krokem bylo změření parametrů učebny, abychom si je porovnali s~poskytnutými podklady, a~následně měření
osvětlenosti ve vertikální a~horizontální rovině v~různých světelných scénách.
Měření bylo prováděno luxmetrem při zapnutí různých světelných scén, včetně kombinací přímého a~nepřímého osvětlení
a postupného rozvicování světelných řad,
 aby byl získán komplexní přehled o~distribuci světla v~učebně.
\medskip
Dále byla provedena jasová analýza pomocí jasového analyzátoru, abychom měli přehled o~kontrastech jasu na tabuli
z~kritických míst v~učebně.
\medskip
Na základě získaných dat byla v~programu Dialux provedena simulace různých světelných scén s~cílem navrhnout optimální
nastavení osvětlení podle biologického komfortu žáků.
Tyto simulace umožnily detailně zohlednit distribuci světla v~prostoru a~jeho dopad na uživatele, což bylo
klíčové pro návrh efektivních světelných scén.
\medskip
Nakonec byla věnována pozornost návrhu ovládacího panelu pro osvětlení, který byl zaměřen na redukci stávajícího
počtu ovládacích prvků.
\medskip
Hlavním cílem bylo snížit počet ovládacích prvků na tři tlačítka, která by spouštěla konkrétní světelné scény,
včetně režimů jako výklad, psaní a~denní světlo. Navíc bylo přidáno tlačítko "Off" pro vypnutí osvětlení a~tlačítka pro regulaci jasu.
\medskip
Nový ovládací systém je navržen s~důrazem na jednoduchost a~intuitivnost, aby uživatelé mohli snadno vybrat
požadovaný režim osvětlení bez zbytečné komplikace. Tímto přístupem se snažíme eliminovat potíže spojené
se stávajícím systémem ovládání a~přispět k~zlepšení uživatelského komfortu výukového prostředí.

\sec Popis učebny
Pro svoji diplomovou práci byla zvolena výuková místnost na gymnáziu u~Libeňského zámku, kde probíhá výuka
biologie. Učebna se nachází ve druhém patře, kde jsou k~nalezení modely lidských orgánů, v~zadní části se nachází
terárium s~chameleonem a~šedé skříně. Interiér místnosti je v~přední části vybaven velkou učitelskou katedrou
a~16 lavicemi pro 32 žáků. Učebna je osvětlovaná v~plném rozsahu i při nižším počtu žáků. Podlaha, zadní stěna
a~stěna u~oken jsou bílé barvy a~zbývající dvě stěny jsou světle oranžové. Okna jsou orientovaná na západ.
U~oken se nacházejí automatizované černé látkové rolety.
\medskip
Umělé osvětlení uvnitř místnosti je řešeno pomocí moderního integrativního osvětlení, které se skládá
z~plnospektrálního LED nepřímého osvětlení a~přímého LED osvětlení s~nižší teplotou chromatičnosti.
Světla jsou rozdělena do tří řad, každá řada se skládá z~šesti nezávislých prvků.
Nad tabulí jsou světelné prvky dlouhé 1.5~m a~nad lavicemi jsou dva světelné prvky a~každé z~nich je dlouhé 3~m.
To umožňuje nastavení přímého a~nepřímého osvětlení, maximální a~minimální hodnoty světla
a~také jejich intenzity v~závislosti na příspěvku denního světla.
Osvětlení je nyní možné ovládat pomocí dvanácti tlačítek. Lze ovládat každou řadu zvlášť nad lavicemi
a~také jen přední řadu u~tabule.
%Cílem mé práce je minimalizovat počet tlačítek pro jednoduchou a~uživatelsky přívětivou manipulaci s osvětlením.


\medskip
{\sbf Parametry učebny:}
\medskip
%\midinsert \clabel[Parametry ucebny]{Parametry učebny}
\ctable{lrrrrr}{
\table{llllll}{
%\hfil Spektrum & Vlnová délka \crl \tskip 4pt
Půdorys & Obdelníkový \cr
Rozměry & 9,2 m $\times$ 6,9 m $\times$ 4 m \cr
Velikost oken & 2,9 m $\times$ 1,84 m \cr
Výška parapetu & 0,88 m \cr
Výška světel & 3,1 m \cr
Lavice & 1,3 m $\times$ 0,5 m $\times$ 0,76 m \cr
Nižší katedra  & 1 m $\times$ 0,70 m $\times$ 0,77 m \cr
Vyšší katedra  & 1,91 m $\times$ 0,70 m $\times$ 0,85 m \cr
Zadní skříně & 6 m $\times$ 0,5 m $\times$ 2,6 m \cr}
}
%\caption/t Parametry učebny
%\endinsert

% \medskip \clabel[ps]{Parametry světla}
% \picw=15cm \cinspic 02_Obrazky/svspec.png
% \caption/f Parametry světla
% \medskip

\medskip
\medskip
\medskip
V~učebně jsou nainstalovaná světla: EVA - LED linear light  \mnote{\inoval{odkaz}}
\medskip
{\sbf Pozitivní vlastnosti svítidel}:
\begitems
    * vysoká citlivost,
    * okamžitý výkon,
    * dlouhá životnost,
    * barevná složka CCT 4000K → podpora kognitivní funkce,
    * hliníkové těleso → přímé a~nepříme svícení.
\enditems
\medskip

{\sbf Technické parametry svítidel:}  \mnote{\inoval{odkaz}}
\medskip
\ctable{lrrrrr}{
% \frame - pridani ramecku
\table{llllll}{  %pridala jsem
%\hfil {\bf Světelné parametry} \crl \tskip 4pt
{\sbf Světelné parametry} \cr
Distribuce světla & lze kombinovat přímě a~nepřímé \cr
Optický systém & difuzor - opál \cr
Index podání barev - CRI &   $\leq$ 90 \cr
Teplota chromatičnosti - CCT & 4000 K~\cr
Svítivost & 8 000 lm/m \cr
Životnost & 50 000 h \cr
UGR  & >19 \cr
{\sbf Elektronické parametry} \cr
Světelný zdroj & LED \cr
Napájecí napětí & AC 230V / 50 Hz \cr
Předřadná část & elektronický předřadník \cr
Možnosti regulace & ano \cr
{\sbf Parametry produktu} \cr
Rozměr & 60x60x3000mm \cr
Materiál & eloxovaný hliník, PMMA \cr
Barva & stříbrná \cr}
}

\sec Aktuální stav učebny:
\medskip
Ovládání aktuálního stavu světelného zařízení v~učebně je řešeno dvanácti tlačítky, což způsobuje složitější ovládání
a potíže novým uživatelům. I~přesto však existují pozitiva, která vyplývají z~tohoto systému, když jsou parametry
osvětlení správně nastaveny. Díky kvalitnímu osvětlení se lépe udržuje pozornost a~zvyšuje se energetická úspornost.
\medskip
Před měřením proběhla diskuze s~učiteli, kde byly zjištěny pozitivní a~negativní aspekty:
\medskip
{\sbf Pozitivní aspekty:}
Učitelé vyjádřili pozitivní zkušenosti s~aktuálním osvětlením ve třídě. Když jsou světla správně nastavena,
pozorují, že děti lépe udržují pozornost a~celkově se lépe soustředí na výuku. Navíc si všimli, že rostliny
umístěné v~učebně také lépe prosperují pod optimálním osvětlením, což přispívá k~příjemné atmosféře ve třídě.
\medskip
{\sbf Negativní aspekty:}
Učitelé zaznamenali několik negativních stránek současného osvětlení. Jedním z~hlavních nedostatků je přebytek
tlačítek na ovládání, což způsobuje zmatek a~náročné ovládání pro nové uživatele. Problematické je chybějící
tlačítko pro rychlé a~jednoduché vypnutí a~zapnutí všech světel. Dalším problémem je obtížnost rozpoznání
stavu světel při nastavení na nejnižší intenzitu, a~není tak jasné, zda jsou světla zapnutá nebo vypnutá
a dochází tak ke zbytečné spotřebě energie.
\medskip
{\bf Po diskuzi s~učiteli bylo navrženo nastavit:}
\medskip
\begitems
    * {\sbf Tlačítko - Off:} Přidat jednoduché tlačítko umožňující úplné vypnutí osvětlení.
\medskip
    * {\sbf Scéna - Výklad:} Vytvořit přednastavenou scénu pro situace, kdy probíhá výklad, která optimalizuje
    osvětlení pro efektivní prezentaci a~snadnou čitelnost pro všechny žáky.
\medskip
    * {\sbf Scéna - Písemka:} Vytvořit přednastavenou scénu určenou pro písemné práce a~testy, která
    zajistí vhodné osvětlení pro psaní a~soustředěnou práci žáků.
\medskip
    * {\sbf Scéna - Adaptace na denní světlo:} Implementovat scénu, která by automaticky přizpůsobila
    osvětlení v~učebně v~závislosti na množství přirozeného denního světla, což by pomohlo minimalizovat
    nepříjemné odlesky a~zajišťovalo optimální viditelnost pro všechny žáky.
\enditems

\medskip \clabel[ucebna]{Učebna}
\picw=15cm \cinspic 03_Obrazky/ucebna.png
\caption/f Učebna
\medskip

\mnote{\inoval{\vbox{Přidat půdorys třídy}}} 

\medskip \clabel[pudorys]{Půdorys učebny}
\picw=15cm \cinspic 03_Obrazky/vykres03.png
\caption/f Půdorys učebny
\medskip






