\chap Simulace v~DIALuxu

V této části jsme se soustředili na vytvoření 3D světelného modelu učebny
a~simulaci osvětlení v programu DIALux.
Modelovali jsme osvětlení pro různé činnosti, jako je výklad, písemka a denní světlo.
Výsledkem byly číselné parametry jednotlivých scén, které slouží řídicí jednotce pro řízení osvětlení v učebně.


DIALux \cite[dialux] je software pro simulaci osvětlení ve vnitřních a venkovních podmínkách.
Umožňuje 3D modelování různých typů prostorů, od malých bytů po velké komerční budovy,
a~vytváření modelů pro venkovní osvětlení, jako jsou silnice, parky a sportovní zařízení.
Program umožňuje simulace s denním i nočním světlem a nabízí možnost vkládání svítidel
od různých výrobců, přičemž parametry těchto svítidel lze individuálně upravit.


DIALux slouží k návrhu, vizualizaci a analýze osvětlenosti, jasu, distribuce světla,
teploty barvy, indexu podání barev, energetické účinnosti, oslnění a dalších parametrů osvětlení.



\medskip \noindent
{\sbf Proces tvorby modelu v programu DIALux:}
% {\sbf Postup návrhu pro vytvoření modelu v DIALuxu:}


\begitems % \style x
* nakreslení půdorysu,
* vytvoření 3D modelu učebny,
* zvolení světelných prvků,
* nastavení parametrů světel,
* verifikace modelu,
* validace modelu,
* výpočty pro zvolené světelné scény -- výklad, písemka, denní světlo,
* určení parametrů pro řídicí jednotku.
\enditems
\medskip

% \medskip \clabel[pudorys2]{Půdorys v~DIALuxu}
% \picw=8cm \cinspic 03_Obrazky/pudorys2.png
% \caption/f Půdorys v~DIALuxu
% \medskip

\medskip \clabel[3D_model]{3D model v~DIALuxu}
\picw=8cm \cinspic 03_Obrazky/3D_model2.png
\caption/f 3D model v~DIALuxu
\medskip

Ze začátku se připravily parametry pro simulaci. Byl vybrány typ místnosti:
Třída - obecné činnosti, kde jsou nastaveny parametry dle normy
a~nelze je měnit. Po nastavení byl nakreslen půdorys z naměřených parametrů učebny.
Poté byl vytvořen 3D model a~zvoleny materiály
a~barvy, aby se model co nejvíce přiblížil realitě.
Do 3D modelu byly vložena LED svítidla od Spectrasol - Spectra Mikado Linear 3000~mm
a~1500~mm (1x~LED linear) s parametry (viz. Příloha \ref[par_svetel])


Zpočátku byly připraveny parametry pro simulaci.
Byl vybrán typ místnosti: Třída - obecné činnosti, kde jsou nastaveny normované parametry, které nelze měnit.
Po nastavení byl podle naměřených údajů nakreslen půdorys učebny.
Poté byl vytvořen 3D model, včetně volby materiálů a barev,
aby se model co nejvíce přiblížil realitě.
Do 3D modelu byla vložena LED svítidla od Spectrasol - Spectra Mikado Linear 3000 mm a 1500 mm (1x LED linear)
s odpovídajícími parametry (viz. příloha \ref[par_svetel]).



\medskip \clabel[para_dial]{parametry v~DIALuxu}
\picw=14cm \cinspic 04_Grafy/05_simulace/parametry.png
\caption/f Zvolené parametry v~DIALuxu
\medskip

Ze začatku byl ozkoušen model změněných parametrů, ale luxů bylo příliš mnoho - poté bylo podle výsledků z měření nastaveny parametry

\secc Validace simulace

\medskip \clabel[val0]{Validace kombinovaného osvětlení}
\picw=18cm \cinspic 04_Grafy/05_simulace/validace/010_combi1.png
\caption/f Validace kombinovaného osvětlení - okno, střed, dveře
\medskip

\medskip \clabel[val1]{Validace nepřímého osvětlení}
\picw=18cm \cinspic 04_Grafy/05_simulace/validace/020_neprime1.png
\caption/f Validace nepřímého osvětlení - okno, střed, dveře
\medskip

\medskip \clabel[val2]{Validace přímého osvětlení}
\picw=18cm \cinspic 04_Grafy/05_simulace/validace/030_prime1.png
\caption/f Validace přímého osvětlení - okno, střed, dveře
\medskip

Nastavené parametry světel:

nepřímé: tabule = 5500, lavice = 11 000, CCT = 5470K

přímé: tabule = 2500, lavice = 5000, CCT = 5470K




% \noindent Byly provedeny tři simulace pro navržené scény - Výklad, Písemka a~Denní světlo

% Cílem je vytvořit optimální osvětlené prostředí pro studenty s ohledem na jejich vizuální pohodlí, soustředění a produktivitu.

% \noindent {\sbf Nastavení simulace:}

% - v nastavení byl vybrán typ místnosti: Třída - obecné činnosti

% - u výkladu byla zvolena jen světla nepřímá a~aby bylo požadováno 300-500lx
% - u písemky byly zvoleny přímá a~nepřímá
% - u denního světla - každá řada s jinou intenzitou od nejnižší po nejvyšší jas

\sec Simulace - Výklad

\begitems
* Zvolený typ osvětlení: Nepřímé
* Cíl: Zajistit měkké a rovnoměrné osvětlení, které minimalizuje oslnění a umožňuje studentům snadno vidět na tabuli a učitele.
* Nastavení: Zajistěte, aby intenzita osvětlení na tabuli a v prostoru pro učitele byla 300 - 500 luxů, vyhnout se přímému osvětlení tabule, které by mohlo způsobovat oslnění.
\enditems

\noindent Světelný tok svítidla nastaven pro nepřímá: tabule = 2500~lm, lavice = 5000~lm

\secc Výsledky simulace


\sec Simulace - Písemka

\begitems
* Zvolený typ osvětlení: Kombinace přímého a nepřímého
* Cíl: Zajistit dostatek osvětlení na pracovní plochy studentů a zároveň minimalizovat oslnění a únavu očí.
* Nastavení: Zajistit, aby intenzita osvětlení na pracovních plochách studentů byla 300-500 luxů, vyhněte se přímému osvětlení očí studentů.
\enditems


\noindent Při nastavení na 4000~lm přímých světel to vychází na 500~lx

\noindent Světelný tok svítidla nastaven pro nepřímé: tabule = 1000~lm, lavice = 2000~lm

pro přímé: tabule = 1500~lm, lavice = 3000~lm

\secc Výsledky simulace

\sec Simulace - Denní světlo

\begitems
* Zvolený typ osvětlení: kombinace denního a umělého osvětlení
* Cíl: Využít co nejvíce denního světla a minimalizovat potřebu umělého osvětlení.
* Nastavení: Použít žaluzie nebo rolety pro regulaci denního světla a zabránění oslnění; v případě potřeby doplňte denní světlo umělým osvětlením tak, aby byla zajištěna požadovaná intenzita osvětlení (300-500 luxů).
\enditems

\noindent Zatažená obloha
\noindent Světelný tok svítidla nastaven pro nepřímá i~přímá u lavic: 3000~lx, 2000~lx, 1000~lx

u tabaule: 1500~lx, 1000~lx, 500~lx

\secc Výsledky simulace

% Osvětlenost
% \medskip \clabel[uzivatel]{Osvětlenost na lavicích}
% \picw=15cm \cinspic 04_Grafy/05_simulace/uzivatel.png
% \caption/f Osvětlenost na lavicích
% \medskip

% Tabule
% \medskip \clabel[tabule]{Tabule}
% \picw=8cm \cinspic 04_Grafy/05_simulace/tabule.png
% \caption/f Tabule
% \medskip

% Strop
% \medskip \clabel[strop]{Strop}
% \picw=15cm \cinspic 04_Grafy/05_simulace/strop.png
% \caption/f Strop
% \medskip

\secc Shrnutí

\noindent {\sbf Scéna 1: Výklad}

Simulace osvětlení pro scénu Výklad ukázala, že nepřímé osvětlení zajišťuje měkké
a rovnoměrné osvětlení v celé učebně. Intenzita osvětlení na tabuli a v prostoru pro učitele dosahuje 200-350 luxů,
což splňuje požadované normy. Simulace neprokázala žádné problematické oslnění ani tmavé skvrny. Barevná teplota
osvětlení 4500 K vytváří příjemnou atmosféru, která podporuje soustředění studentů.

\noindent {\sbf Scéna 2: Písemka}

Kombinace přímého a nepřímého osvětlení v simulaci scény Písemka zajišťuje dostatek osvětlení na pracovních plochách studentů
(500lx) a zároveň minimalizuje oslnění a únavu očí.
Simulace neprokázala žádné problematické stíny ani tmavé skvrny.

\noindent {\sbf Scéna 3: Denní světlo}

Simulace scény Denní světlo ukázala, že správně orientovaná okna umožňují optimální využití denního
světla v učebně. V případě potřeby je denní světlo doplňováno umělým osvětlením tak, aby byla zajištěna požadovaná
intenzita osvětlení (300-500 luxů) v závislosti na denní době a povětrnostních podmínkách. Simulace neprokázala žádné
problematické oslnění ani tmavé skvrny. Barevná teplota denního světla se mění v závislosti na ročním období a počasí,
ale v průměru se pohybuje kolem 5000 K, čímž podporuje soustředění a produktivitu studentů.

\noindent {\sbf Shrnutí}

Simulace osvětlení učebny v programu DIALux pro scény Výklad, Písemka a Denní světlo ukázaly, že navržené osvětlení
splňuje všechny platné normy a předpisy, a zajišťuje optimální vizuální komfort a vhodné podmínky pro studenty.
Osvětlení je rovnoměrné, bez oslnění a tmavých míst, a jeho barevná teplota podporuje soustředění a produktivitu.
Využití denního světla snižuje spotřebu energie a zvyšuje celkovou pohodu studentů.

% Návrh osvětlení učebny v DIALuxu umožňuje vytvořit optimální osvětlené prostředí pro studenty s ohledem na jejich
% vizuální pohodlí, soustředění a produktivitu. Důležité je zvážit specifické potřeby pro každou scénu
% (Výklad, Písemka, Denní světlo) a vybrat vhodné typy osvětlení a parametry. Díky pečlivému plánování
% a simulaci osvětlení v DIALuxu lze dosáhnout vynikajících výsledků a vytvořit učebnu, která podporuje učení a well-being studentů.

% Další tipy:

% Při navrhování osvětlení učebny je důležité dodržovat platné normy a předpisy.
% Doporučuje se zapojit do procesu návrhu osvětlení pedagogy a studenty, aby se zohlednily jejich specifické potřeby a preference.
% Je důležité pravidelně kontrolovat a udržovat osvětlovací systém, aby se zajistila jeho optimální funkce.


% Uvidim jestli dam shrnuti do tabulky???
% \medskip
% \midinsert \clabel[vysl_sim]{Výsledky ze simulace}
% \ctable{lccc}{
% %\hfil Spektrum & Vlnová délka \crl \tskip 4pt
% {\sbf Typ scény}  &    {\sbf Max} [lx] & {\sbf Min} [lx]  &   {\sbf Rovnoměrnost} [-]   \cr
% Kombinované       &     796       &   350       &       0.4         \cr
% Přímé             &     505       &   233       &       0.5         \cr
% Nepřímé           &     291       &   117       &       0.4         \cr
% }
% \caption/t Výsledky ze simulace
% \endinsert