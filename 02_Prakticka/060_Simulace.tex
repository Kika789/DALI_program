\chap Simulace v~DIALuxu

Pro návrh optimálních světelných scén byla provedena simulace v~programu DIALux.

\medskip
Postup návrhu:

\begitems \style n
* nakreslení půdorysu
* vytvoření 3D modelu učebny
* zvolení světelných prvků
* nastavení parametrů světel
* výpočty pro zvolené světelné scény - výklad, písemka, denní světlo
\enditems
\medskip

\medskip \clabel[pudorys2]{Půdorys v~DIALuxu}
\picw=8cm \cinspic 03_Obrazky/pudorys2.png
\caption/f Půdorys v~DIALuxu
\medskip

\medskip \clabel[3D_model]{3D model v~DIALuxu}
\picw=8cm \cinspic 03_Obrazky/3D_model2.png
\caption/f 3D model v~DIALuxu
\medskip

Byly provedeny tři simulace pro navržené scény - Výklad, Písemka a Denní světlo

Nastavení simulace:

- v nastavení byl vybrán typ místnosti: Třída - obecné činnosti
- obrázek
- u výkladu byla zvolena jen světla nepřímá a aby bylo požadováno 300-500lx
- u písemky byly zvoleny přímá a nepřímá 
- u denního světla - každá řada s jinou intenzitou od nejnižší po nejvyšší jas

Výpočty ze simulace

Osvětlenost
\medskip \clabel[uzivatel]{Osvětlenost na lavicích}
\picw=15cm \cinspic 04_Grafy/05_simulace/uzivatel.png
\caption/f Osvětlenost na lavicích
\medskip

Tabule
\medskip \clabel[tabule]{Tabule}
\picw=8cm \cinspic 04_Grafy/05_simulace/tabule.png
\caption/f Tabule
\medskip

Strop
\medskip \clabel[strop]{Strop}
\picw=15cm \cinspic 04_Grafy/05_simulace/strop.png
\caption/f Strop
\medskip



Výklad 
Světelný tok svítidla nastaven pro nepřímá: 4000 lm, pro přímá: 1 lm
+ výpočty

Písemka
Při nastavení na 4000lm přímých světel to vychází na 500 lx

Světelný tok svítidla nastaven pro nepřímá: 2000lm , pro přímá: 3000lm

Zatažená obloha
Světelný tok svítidla nastaven pro nepřímá i přímá: 3000 lx, 2000 lx, 1000lx