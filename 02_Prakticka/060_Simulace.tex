\chap Simulace v~DIALuxu

Pro návrh optimálních světelných scén byla provedena simulace v~programu DIALux.

DIALux je software pro simulaci osvětlení ve vnitřních a~venkovních podmínkách. Umožňuje výpočty,.....

\medskip
Postup návrhu:

\begitems \style n
* nakreslení půdorysu
* vytvoření 3D modelu učebny
* zvolení světelných prvků
* nastavení parametrů světel
* výpočty pro zvolené světelné scény - výklad, písemka, denní světlo
\enditems
\medskip

\medskip \clabel[pudorys2]{Půdorys v~DIALuxu}
\picw=8cm \cinspic 03_Obrazky/pudorys2.png
\caption/f Půdorys v~DIALuxu
\medskip

\medskip \clabel[3D_model]{3D model v~DIALuxu}
\picw=8cm \cinspic 03_Obrazky/3D_model2.png
\caption/f 3D model v~DIALuxu
\medskip

 

Byly provedeny tři simulace pro navržené scény - Výklad, Písemka a~Denní světlo

Cílem je vytvořit optimální osvětlené prostředí pro studenty s ohledem na jejich vizuální pohodlí, soustředění a produktivitu.

Nastavení simulace:

- v nastavení byl vybrán typ místnosti: Třída - obecné činnosti
- obrázek
- u výkladu byla zvolena jen světla nepřímá a~aby bylo požadováno 300-500lx
- u písemky byly zvoleny přímá a~nepřímá
- u denního světla - každá řada s jinou intenzitou od nejnižší po nejvyšší jas

\sec Simulace - Výklad

\begitems \style o
* Zvolený typ osvětlení: Nepřímé
* Cíl: Zajistit měkké a rovnoměrné osvětlení, které minimalizuje oslnění a umožňuje studentům snadno vidět na tabuli a učitele.
* Nastavení: Zajistěte, aby intenzita osvětlení na tabuli a v prostoru pro učitele byla 300 - 500 luxů, vyhnout se přímému osvětlení tabule, které by mohlo způsobovat oslnění.
\enditems

Světelný tok svítidla nastaven pro nepřímá: 4000 lm, pro přímá: 1 lm

\medskip \clabel[vyklad]{Výklad}
\picw=10cm \cinspic 04_Grafy/05_simulace/Vyklad.png
\caption/f Výklad
\medskip

\medskip \clabel[vyklad2]{Výklad2}
\picw=10cm \cinspic 04_Grafy/05_simulace/vyklad_u.png
\caption/f Výklad2
\medskip

\sec Simulace - Písemka

\begitems
* Zvolený typ osvětlení: Kombinace přímého a nepřímého
* Cíl: Zajistit dostatek osvětlení na pracovní plochy studentů a zároveň minimalizovat oslnění a únavu očí.
* Nastavení: Zajistit, aby intenzita osvětlení na pracovních plochách studentů byla 300-500 luxů, vyhněte se přímému osvětlení očí studentů.
\enditems


Při nastavení na 4000~lm přímých světel to vychází na 500~lx

Světelný tok svítidla nastaven pro nepřímá: 2000~lm, pro přímá: 3000~lm

\medskip \clabel[pisemka]{Písemka}
\picw=10cm \cinspic 04_Grafy/05_simulace/Pisemka.png
\caption/f Písemka
\medskip

\medskip \clabel[pisemka2]{Písemka2}
\picw=10cm \cinspic 04_Grafy/05_simulace/pisemka_u.png
\caption/f Písemka2
\medskip

\sec Simula - Denní světlo

\begitems \style o
* Zvolený typ osvětlení: kombinace denního a umělého osvětlení
* Cíl: Využít co nejvíce denního světla a minimalizovat potřebu umělého osvětlení.
* Nastavení: Použít žaluzie nebo rolety pro regulaci denního světla a zabránění oslnění; v případě potřeby doplňte denní světlo umělým osvětlením tak, aby byla zajištěna požadovaná intenzita osvětlení (300-500 luxů).
\enditems

Zatažená obloha
Světelný tok svítidla nastaven pro nepřímá i~přímá: 3000~lx, 2000~lx, 1000~lx

\medskip \clabel[denni_svetlo]{Denní světlo}
\picw=10cm \cinspic 04_Grafy/05_simulace/Pisemka.png
\caption/f Denní světlo
\medskip

% Osvětlenost
% \medskip \clabel[uzivatel]{Osvětlenost na lavicích}
% \picw=15cm \cinspic 04_Grafy/05_simulace/uzivatel.png
% \caption/f Osvětlenost na lavicích
% \medskip

% Tabule
% \medskip \clabel[tabule]{Tabule}
% \picw=8cm \cinspic 04_Grafy/05_simulace/tabule.png
% \caption/f Tabule
% \medskip

% Strop
% \medskip \clabel[strop]{Strop}
% \picw=15cm \cinspic 04_Grafy/05_simulace/strop.png
% \caption/f Strop
% \medskip

\sec Shrnutí

Návrh osvětlení učebny v DIALuxu umožňuje vytvořit optimální osvětlené prostředí pro studenty s ohledem na jejich
vizuální pohodlí, soustředění a produktivitu. Důležité je zvážit specifické potřeby pro každou scénu
(Výklad, Písemka, Denní světlo) a vybrat vhodné typy osvětlení a parametry. Díky pečlivému plánování
a simulaci osvětlení v DIALuxu lze dosáhnout vynikajících výsledků a vytvořit učebnu, která podporuje učení a well-being studentů.

Další tipy:

Při navrhování osvětlení učebny je důležité dodržovat platné normy a předpisy.
Doporučuje se zapojit do procesu návrhu osvětlení pedagogy a studenty, aby se zohlednily jejich specifické potřeby a preference.
Je důležité pravidelně kontrolovat a udržovat osvětlovací systém, aby se zajistila jeho optimální funkce.