\chap Simulace v~DIALuxu

V této části jsme se soustředili na simulaci osvětlení ve 3D modelu učebny v~programu DIALux.
Modelovali jsme osvětlení pro různé školní činnosti, jako je výklad, písemka a~adaptace na denní světlo.
Výsledkem simulace byly číselné parametry jednotlivých scén, které budou využity pro nastavení jasu
světel v~řídicí jednotce při budoucí realizaci řízení osvětlení v~učebně.

DIALux \cite[dialux] je software pro simulaci osvětlení ve vnitřních a~venkovních podmínkách.
Umožňuje 3D modelování různých typů prostorů, od malých bytů po velké komerční budovy,
a~vytváření modelů pro venkovní osvětlení, jako jsou silnice, parky a~sportovní zařízení.
Program umožňuje simulace s~denním i~nočním světlem a~nabízí možnost vkládání svítidel
od různých výrobců, přičemž parametry těchto svítidel lze individuálně upravit.


Celkově lze říct, že DIALux slouží k~návrhu, vizualizaci a~analýze osvětlenosti, jasu, distribuce světla,
teploty barvy, indexu podání barev, energetické účinnosti, oslnění a~dalších parametrů osvětlení.
V~základní verzi je program zdarma~\cite[dialux].
% a~je dostupný výhradně pro operační systémy Windows


\medskip \noindent
{\sbf Proces tvorby modelu v programu DIALux:}
% {\sbf Postup návrhu pro vytvoření modelu v DIALuxu:}


\begitems % \style x
* nakreslení půdorysu,
* vytvoření 3D modelu učebny,
* zvolení světelných prvků,
* nastavení parametrů světel,
* verifikace modelu,
* validace modelu,
* výpočty pro zvolené světelné scény -- výklad, písemka, denní světlo,
* určení parametrů pro řídicí jednotku.
\enditems

% \medskip \clabel[pudorys2]{Půdorys v~DIALuxu}
% \picw=8cm \cinspic 03_Obrazky/pudorys2.png
% \caption/f Půdorys v~DIALuxu
% \medskip


Podle uvedeného postupu byly připraveny parametry pro simulaci.
Byl vybrán typ místnosti: {\sbf Třída} -- obecné činnosti (viz obr.~\ref[para_dial]), kde jsou nastaveny parametry z~normy, které nelze měnit.
Po nastavení byl podle naměřených údajů nakreslen půdorys učebny.
Poté byl vytvořen 3D model, včetně volby materiálů a barev.
Do 3D modelu byla vložena LED svítidla firmy {\sbf Spectrasol -- Spectra Mikado Linear} 3000~mm a 1500~mm (1~$\times$~LED linear)
s~odpovídajícími parametry (viz. příloha \ref[par_svetel]),
která se jeví jako nejvíce podobná skutečným světlům umístěným v učebně.

\medskip \clabel[para_dial]{parametry v~DIALuxu}
\picw=14cm \cinspic 04_Grafy/05_simulace/parametry.png
\caption/f Zvolené parametry v~DIALuxu
\medskip

Výsledný model je vidět na obrázku \ref[3D_model].
Současně tento náhled slouží i~jako verifikace modelu,
kde jsme zkontrolovali správnost umístění svítidel a~materiálů,
aby simulace co nejlépe odpovídala reálnému prostředí.
Tento krok ověřuje, že všechny prvky jsou správně
nastaveny a~připraveny pro další analýzu a~optimalizaci osvětlení v~učebně.

\medskip \clabel[3D_model]{3D model v~DIALuxu}
\picw=8cm \cinspic 03_Obrazky/3D_model2.png
\caption/f 3D model v~DIALuxu
\medskip


\sec Validace modelu

Pro ověření přesnosti a~správnosti vytvořeného 3D modelu v~programu DIALux
byla provedena validace.
Tento krok zahrnuje porovnání simulovaných hodnot s~reálným měřením v~učebně a~kontrolu,
zda model splňuje stanovené požadavky a~normy.
Validace je klíčová pro zajištění, že simulace odpovídá skutečným
podmínkám a~že navržené osvětlení bude efektivní a~funkční v~reálném prostředí.



\medskip
\noindent{\sbf Postup provedení validace:}
\begitems
    %* Data hodnotíme pouze vizuálně.
    * Porovnáváme kombinované, nepřímé a přímé osvětlení mezi měřením a simulací.
    % * Hodnotíme míru tvarové shody isofot.
    % * Porovnáme intenzitu osvětlení s rezervou  10~{\pcent} až 15~{\pcent}.
\enditems


\noindent Parametry světel, které nejvíce odpovídaly měřeným hodnotám,
vidíme v tabulce~\ref[nastav_parametry].

\medskip
\midinsert \clabel[nastav_parametry]{Nastavené parametry světel}
\ctable{lccc}{
{\sbf Typ osvětlení}  &    {\sbf Světla u tabule} & {\sbf Světla nad lavicemi}   \cr
                      &    [lm]       &       [lm]              \crl
{\sbf Nepřímé}      &     5 500       &   11 000                \cr
{\sbf Přímé}        &     2 500       &    5 000                \cr
}
\caption/t Nastavené parametry světel
\endinsert


% \medskip
\noindent{\sbf Porovnání simulovaných hodnot s reálným měřením v učebně:}


Kombinované osvětlení podle měření z obr.~\ref[svet_mapa] dosahuje maxima kolem 600 luxů,
což odpovídá simulovaným hodnotám na obr.~\ref[val0].
Tvarově se oba obrázky velmi dobře podobají, což potvrzuje správnost modelu.
% Simulace věrně kopíruje rozložení osvětlenosti v místnosti,
% což je klíčové pro validaci modelu.
Toto srovnání ukazuje, že jak přímé, tak nepřímé osvětlení
je v modelu dobře zachyceno, což je důležité pro řízení světelných scén.


Podobné výsledky byly získány také při porovnání měřených a simulovaných dat
u~nepřímého osvětlení na obr.~\ref[svet_mapa2] a obr.~\ref[val1]
a~přímého osvětlení na obr.~\ref[svet_mapa3] a obr.~\ref[val2].

Všechny simulované obrázky ukazují, že katedra u tabule má nedostatečné osvětlení.
Důvodem je, že simulační rovina byla umístěna do výšky lavic (800~mm).
Katedra je ale umístěna na vyvýšeném pódiu, takže se nachází nad simulační rovinou
a~to co vidíme je vlastně osvětlení pod stolem.


\medskip \clabel[val0]{Validace kombinovaného osvětlení}
\picw=15cm \cinspic 04_Grafy/05_simulace/validace/010_combi1.jpg
\caption/f Validace kombinovaného osvětlení - okno, střed, dveře
\medskip

\medskip \clabel[val1]{Validace nepřímého osvětlení}
\picw=15cm \cinspic 04_Grafy/05_simulace/validace/020_neprime1.jpg
\caption/f Validace nepřímého osvětlení - okno, střed, dveře
\medskip

\medskip \clabel[val2]{Validace přímého osvětlení}
\picw=15cm \cinspic 04_Grafy/05_simulace/validace/030_prime1.jpg
\caption/f Validace přímého osvětlení - okno, střed, dveře
\medskip


% \noindent Byly provedeny tři simulace pro navržené scény - Výklad, Písemka a~Denní světlo

% Cílem je vytvořit optimální osvětlené prostředí pro studenty s ohledem na jejich vizuální pohodlí, soustředění a produktivitu.

% \noindent {\sbf Nastavení simulace:}

% - v nastavení byl vybrán typ místnosti: Třída - obecné činnosti

% - u výkladu byla zvolena jen světla nepřímá a~aby bylo požadováno 300-500lx
% - u písemky byly zvoleny přímá a~nepřímá
% - u denního světla - každá řada s jinou intenzitou od nejnižší po nejvyšší jas


Na základě úspěšné validace modelu lze přistoupit k optimalizaci parametrů osvětlení v učebně pro různé scény.

\sec Simulace -- Výklad

Pro světelnou scénu {\sbf Výklad} byl zvolen typ nepřímého osvětlení.

% \noindent {\sbf Parametry a cíle pro simulaci:}
\begitems
* {\sbf Typ osvětlení:} {Nepřímé.}
* {\sbf Cíl:} Zajistit měkké a rovnoměrné osvětlení,
        které minimalizuje oslnění a~umožňuje studentům snadno vidět na interaktivni tabuli.
* {\sbf Osvětlení:} Požadavkem je, aby intenzita osvětlení na tabuli
a~v~prostoru byla 300--500 luxů a~vyhnout se přímému osvětlení tabule,
které by mohlo způsobovat nižší viditelnost na interaktivní tabuli.
\enditems



% \noindent Světelný tok svítidla nastaven pro nepřímá: tabule = 2500~lm, lavice = 5000~lm
Ze simulace vyplývá, že nastavení světelného toku svítidel pro nepřímé osvětlení je následující:

% \medskip
\midinsert \clabel[param_vyklad]{Parametry světel -- Výklad}
\ctable{lccc}{
{\sbf Typ osvětlení}  &    {\sbf Světla u tabule} & {\sbf Světla nad lavicemi}   \cr
                      &    [lm]                   &       [lm]                   \crl
{\sbf Nepřímé}       &     2 500                  &   5 000                      \cr
}
\caption/t Nastavené parametry světel pro scénu -- Výklad
\endinsert


% \secc Výsledky simulace

Na následujícím obrázku vidíme výsledky simulace osvětlení v učebně pro scénu Výklad
se srovnávací rovinou ve výšce 800~mm nad podlahou (levý obrázek \ref[simulace0])
a osvětlenost na stropě jako kontrola rovnoměrnosti osvětlení.


\medskip \clabel[simulace0]{Simulace scény -- Výklad}
\picw=15cm \cinspic 04_Grafy/05_simulace/sceny/020_vyklad.jpg
\caption/f Simulace scény výklad. Vlevo srovnávací rovina 800~mm, vpravo strop. Isofoty v luxech.
\medskip

\sec Simulace -- Písemka

Pro světelnou scénu {\sbf Písemka} byl zvolen typ kombinovaného osvětlení.


\begitems
* {\sbf Typ osvětlení:} Kombinace přímého a nepřímého
* {\sbf Cíl:} Zajistit dostatek osvětlení na pracovní plochy
         studentů a zároveň minimalizovat oslnění a únavu očí.
* {\sbf Nastavení:} Požadavkem je, aby intenzita osvětlení na pracovních plochách
        studentů byla 300--500 luxů a vyhnout se přímému osvětlení očí studentů.
\enditems


% \noindent Při nastavení na 4000~lm přímých světel to vychází na 500~lx
% \noindent Světelný tok svítidla nastaven pro nepřímé: tabule = 1000~lm, lavice = 2000~lm
% pro přímé: tabule = 1500~lm, lavice = 3000~lm

Ze simulace vyplývá, že nastavení světelného toku svítidel pro přímé a nepřímé osvětlení je následující:

% \medskip
\midinsert \clabel[param_pisemka]{Parametry světel -- Písemka}
\ctable{lccc}{
{\sbf Typ osvětlení}  &    {\sbf Světla u tabule} & {\sbf Světla nad lavicemi}     \cr
                      &    [lm]                   &       [lm]                     \crl
{\sbf Nepřímé}        &     1 000                 &    2 000                       \cr
{\sbf Přímé}          &     1 500                 &    3 000                       \cr
}
\caption/t Nastavené parametry světel pro scénu -- Písemka
\endinsert


% \secc Výsledky simulace


Na následujícím obrázku vidíme výsledky simulace osvětlení v učebně pro scénu Písemka
se srovnávací rovinou ve výšce 800~mm nad podlahou (levý obrázek \ref[simulace1])
a osvětlenost na stropě jako kontrola rovnoměrnosti osvětlení nepřímého světla.


\medskip \clabel[simulace1]{Simulace scény -- Písemka}
\picw=15cm \cinspic 04_Grafy/05_simulace/sceny/010_pisemka.jpg
\caption/f Simulace scény písemka. Vlevo srovnávací rovina 800~mm, vpravo strop. Isofoty v luxech.
\medskip

\sec Simulace -- Denní světlo

Pro světelnou scénu {\sbf Denní světlo} byl zvolen typ kombinovaného umělého osvětlení a~přirozené denní světlo.

\begitems
* {\sbf Typ osvětlení:} kombinace denního a umělého osvětlení
* {\sbf Cíl:} Využít co nejvíce denního světla a minimalizovat potřebu umělého osvětlení.
* {\sbf Nastavení:} V případě potřeby doplnit denní světlo umělým osvětlením tak, aby byla zajištěna požadovaná intenzita
osvětlení (300-500 luxů).
% Použít žaluzie nebo rolety pro regulaci denního světla a zabránění oslnění;

\enditems

% \noindent Zatažená obloha
% \noindent Světelný tok svítidla nastaven pro nepřímá i~přímá u lavic: 3000~lx, 2000~lx, 1000~lx
% u tabaule: 1500~lx, 1000~lx, 500~lx

\midinsert \clabel[param_svetlo]{Parametry světel -- Denní světlo}
\ctable{lcccc}{
{\sbf Typ osvětlení}  &    {\sbf Světla u dveří} & {\sbf Světla ve středu}   &  {\sbf Světla u okna}              \cr
                      &    [lm]                  &        [lm]               &    [lm]                            \crl
{\sbf Nepřímé}        &     3 000                &     2 000                 &    1000                            \cr
{\sbf Přímé}          &     1 500                &     1 000                 &     500                            \cr
}
\caption/t Nastavené parametry světel pro scénu -- Denní světlo
\endinsert


% \secc Výsledky simulace

Na následujícím obrázku vidíme výsledky simulace osvětlení v učebně pro scénu Denní světlo
se srovnávací rovinou ve výšce 800~mm nad podlahou.
Vlevo na obrázku \ref[simulace2] je simulace příspěvku denního
světla s kombinací umělého osvětlení a~vpravo výsledky simulace při zatažené obloze bez umělého osvětlení.

\medskip \clabel[simulace2]{Simulace scény -- Denní světlo}
\picw=18cm \cinspic 04_Grafy/05_simulace/sceny/030_svetlo1.jpg
\caption/f Simulace scény denní světlo. Vlevo denní a umělé osvětlení, vpravo zatažená obloha. Srovnávací rovina 800~mm. Isofoty v luxech.
\medskip

% Osvětlenost
% \medskip \clabel[uzivatel]{Osvětlenost na lavicích}
% \picw=15cm \cinspic 04_Grafy/05_simulace/uzivatel.png
% \caption/f Osvětlenost na lavicích
% \medskip

% Tabule
% \medskip \clabel[tabule]{Tabule}
% \picw=8cm \cinspic 04_Grafy/05_simulace/tabule.png
% \caption/f Tabule
% \medskip

% Strop
% \medskip \clabel[strop]{Strop}
% \picw=15cm \cinspic 04_Grafy/05_simulace/strop.png
% \caption/f Strop
% \medskip

\medskip
\sec Vyhodnocení simulace

% zavislost intenzity osvětlení a vykon osvetleni
% https://ozuno.com/blog/step-darkness-dali-brightness-levels/

\medskip
\noindent {\sbf Scéna Výklad:}

Simulace osvětlení pro scénu Výklad ukázala, že nepřímé osvětlení zajišťuje měkké
a rovnoměrné osvětlení v celé učebně. Intenzita osvětlení v prostoru dosahuje 200--350 luxů,
což splňuje požadované normy.
% Barevná teplotaosvětlení 5470~K vytváří příjemnou atmosféru, která podporuje soustředění studentů.

Z poměru světelného toku (tab.~\ref[param_vyklad])
a světelného toku pro maximální intenzitu (tab.~\ref[nastav_parametry])
plyne, že světelný tok nepřímých svítidel pro tuto scénu je nastaven na 45~{\pcent } maximálního světelného toku.



% Výpočet: interval: 85 – 254, 254 – 85 = 169 → 100 \pcent
% tabule: ${2500 \over 5500} = 0,45$
%         ${0,45 \cdot 169} = 76,8$
%         $76,8 + 85 = 161,8$ – nastavíme
% lavice: ${5000 \over 11000} = 0,45$
%         ${0,45 \cdot 169} = 76,8$
%         $76,8 + 85 = 161,8$ – nastavíme

\medskip
\noindent {\sbf Scéna Písemka:}

Kombinace přímého a nepřímého osvětlení v simulaci scény Písemka zajišťuje dostatek osvětlení na pracovních plochách studentů
(500~lx) a zároveň minimalizuje oslnění a únavu očí.


Z poměru světelného toku (tab.~\ref[param_pisemka])
a světelného toku pro maximální intenzitu (tab.~\ref[nastav_parametry])
plyne, že světelný tok nepřímých svítidel pro tuto scénu je nastaven na 18~{\pcent } maximálního světelného toku
a světelný tok přímých svítidel na 27~{\pcent } maximálního světelného toku.

% Pro nepřímé:
% tabule: ${1000 \over 5500} = 0,18$
%         ${0,18 \cdot 169} = 30,7$
%         $30,7 + 85 = 115,7$ – nastavíme
% lavice: ${2000 \over 11000} = 0,18$
%         ${0,18 \cdot 169} = 30,7$
%         $30,7 + 85 = 115,7$ – nastavíme
% Pro přímé:
% tabule: ${1500 \over 5500} = 0,27$
%         ${0,27 \cdot 169} = 46,1$
%         $46,1 + 85 = 131,1$ – nastavíme
% lavice: ${3000 \over 11000} = 0,27$
%         ${0,27 \cdot 169} = 46,1$
%         $46,1 + 85 = 131,1$ – nastavíme

\medskip
\noindent {\sbf Scéna Denní světlo:}

Simulace scény Denní světlo ukázala, že správně orientovaná okna umožňují optimální využití denního
světla v učebně. V případě potřeby je denní světlo doplňováno umělým osvětlením tak, aby byla zajištěna požadovaná
intenzita osvětlení (300--500 luxů) v~závislosti na počasí.

% Barevná teplota denního světla se mění v závislosti na ročním období a počasí,
% ale v průměru se pohybuje kolem 5000~K, čímž podporuje soustředění a produktivitu studentů.


Z poměru světelného toku (tab.~\ref[param_svetlo])
a světelného toku pro maximální intenzitu (tab.~\ref[nastav_parametry])
plyne, že světelný tok nepřímých svítidel
je u dveří nastaven na 27~{\pcent }, uprostřed na 18~{\pcent } a u okna na 9~{\pcent }
maximálního světelného toku
a světelný tok přímých svítidel
je u dveří nastaven na 14~{\pcent }, uprostřed na 9~{\pcent } a u okna na 5~{\pcent }
maximálního světelného toku.


% {\sbf dveře:}
% tabule: ${1500 \over 5500} = 0,27$
%         ${0,27 \cdot 169} = 46,1$
%         $46,1 + 85 = 131,1$ – nastavíme
% lavice: ${3000 \ over 11000} = 0,27$
%         ${0,27 \cdot 169} = 46,1$
%         $46,1 + 85 = 131,1$ – nastavíme
% {\sbf střed:}

% tabule: ${1000 \over 5500} = 0,18$
%         ${0,18 \cdot 169} = 30,7$
%         $30,7 + 85 = 114,7$ – nastavíme
% lavice: ${2000 \over 11000} = 0,18$
%         ${0,18 \cdot 169} = 30,7$
%         $30,7 + 85 = 114,7$ – nastavíme
% {\sbf okno:}

% tabule: ${500 \over 5500} = 0,09$
%         ${0,09 * 169} = 15,4$
%         $15,4 + 85 = 100,4$ – nastavíme
% lavice: ${1000 \over 11000} = 0,09$
%         ${0,09 \cdot 169} = 15,4$
%         $15,4 + 85 = 100,4$ – nastavíme

\medskip
\noindent {\sbf Shrnutí:}

Simulace osvětlení učebny v programu DIALux pro scény Výklad, Písemka a Denní světlo ukázaly, že navržené osvětlení
splňuje všechny platné normy a předpisy, a~zajišťuje optimální vizuální komfort a vhodné podmínky pro studenty.
Osvětlení je rovnoměrné, bez oslnění a~tmavých míst, jeho barevná teplota podporuje soustředění a~produktivitu.
Využití denního světla snižuje spotřebu energie a~zvyšuje celkovou pohodu studentů.


% Návrh osvětlení učebny v DIALuxu umožňuje vytvořit optimální osvětlené prostředí pro studenty s ohledem na jejich
% vizuální pohodlí, soustředění a produktivitu. Důležité je zvážit specifické potřeby pro každou scénu
% (Výklad, Písemka, Denní světlo) a vybrat vhodné typy osvětlení a parametry. Díky pečlivému plánování
% a simulaci osvětlení v DIALuxu lze dosáhnout vynikajících výsledků a vytvořit učebnu, která podporuje učení a well-being studentů.

% Další tipy:

% Při navrhování osvětlení učebny je důležité dodržovat platné normy a předpisy.
% Doporučuje se zapojit do procesu návrhu osvětlení pedagogy a studenty, aby se zohlednily jejich specifické potřeby a preference.
% Je důležité pravidelně kontrolovat a udržovat osvětlovací systém, aby se zajistila jeho optimální funkce.


% Uvidim jestli dam shrnuti do tabulky???
% \medskip
% \midinsert \clabel[vysl_sim]{Výsledky ze simulace}
% \ctable{lccc}{
% %\hfil Spektrum & Vlnová délka \crl \tskip 4pt
% {\sbf Typ scény}  &    {\sbf Max} [lx] & {\sbf Min} [lx]  &   {\sbf Rovnoměrnost} [-]   \cr
% Kombinované       &     796       &   350       &       0.4         \cr
% Přímé             &     505       &   233       &       0.5         \cr
% Nepřímé           &     291       &   117       &       0.4         \cr
% }
% \caption/t Výsledky ze simulace
% \endinsert