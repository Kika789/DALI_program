\sec Čtvrté měření

Čtvrté měření proběhlo 22.4.2024 v čase 17:00–18:30. Byla provedena jasová analýza
pro různé světelné scény, abychom zmapovali jasové podmínky učebny biologie a také abychom zjistili čitelnost na interaktivní tabuli.
Měření bylo prováděno pomocí jasového analyzátoru \glref{LDA} (Luminance Distribution Analyzer) \cite[jas_analyzator],
který je navržen k hodnocení jasu za použití metadat digitální fotografie. Pro tento proces byl použit fotoaparát
značky {\sbf Nikon model D7200}, s CMOS snímačem o rozměrech 23,5 mm $\times$ 15,6 mm. Zaostření bylo nastaveno v rozmezí od 1 do 3,
citlivost ISO byla stanovena na 100 a clona na hodnotu 4. Výška čočky objektivu byla nastavena na 1,2 m,
což odpovídá normové výšce očí osoby v sedící pozici.

\noindent Měření proběhlo ve dvou měřících bodech. V bodu 2 a 9 (viz půdorys \ref[pudorys]).
\begitems
* {\sbf Bod 9} – žák sedí v zadní lavici ve středu a má dominantní pohled na tabuli. Žák v~tomto bodě může být vystaven oslnění z umělých světelných zdrojů a nebo světelnými odrazy od leklých povrchů lavic a židlí.
* {\sbf Bod 2} – žák sedí ve třetí lavici u dvěří. Žák při pohledu na tabuli může být vystaven oslnění přirozeného světla, které proniká z prvního okna a odlesky z lesklých povrchů lavic.
\enditems

Pro každý kontrolní bod bylo pořízeno devět snímků na složení \glref{HDR} obrazu. Snímky
byly zachyceny s rychlostí závěrky {\sbf 0,5s; 1/10s; 1/50s; 1/250s; 1/1000s; 1/2000s; 1/4000s;
1/8000s}. Objektiv byl zvolený fish eye bez filtru.
Byly nastaveny tři světelné scény, kde byly zapnutá světelná zařízení na maximální
intenzitu - kombinované, přímé, nepřímé (řazeny z leva do prava na grafech níže).
Naměřená data byla následně zpracována v programu LumiDISP \cite[jas_analyzator] z Vysokého učení technického v Brně.

\medskip
\medskip \clabel[jasovka0]{Měření pomocí jasového analyzátoru}
\picw=8cm \cinspic 03_Obrazky/jasovka.jpg
\caption/f Ukázka nastavení měření pomocí jasového analyzátoru
\medskip

\secc Grafy

\medskip
\medskip \clabel[jasovka]{Jasová analýza}
\picw=14cm \cinspic 04_Grafy/04_mereni/bod9a.png
\caption/f Jasová analýza bodu 9 - zatažená rolety
\medskip

\medskip
\medskip \clabel[jasovka2]{Jasová analýza 2}
\picw=14cm \cinspic 04_Grafy/04_mereni/bod2b.png
\caption/f Jasová analýza bodu 2 - zatažené rolety
\medskip
\medskip

\medskip
\medskip \clabel[jasovka3]{Jasová analýza 2b}
\picw=14cm \cinspic 04_Grafy/04_mereni/bod2a.png
\caption/f Jasová analýza bodu 2b - vytažené rolety
\medskip
\medskip

% \mnote{\inoval{\vbox{hodnocení}}}

\secc Zhodnocení grafů

Výsledky s měření hodnotíme z hlediska oslnění žáků a~svítivosti tabule. Pro dobrou viditelnost budeme uvažovat
hodnoty jasu 300 -- 500~cd/m$^2$, tomu odpovídá přibližně světle červená a~až oranžová barva
v~obrázcích \ref[jasovka] a~\ref[jasovka2].
Barvy zelená a~modrá, jas 0 -- 50~cd/m$^2$m znamenají příliš slabý jas a~nemohou způsobit oslnění.
Barva žlutá, jas nad 1000~cd/m$^2$ může znamenat oslnění.

\medskip \noindent {\sbf Graf (\ref[jasovka]) - Měření v bodě 9 – zatažené rolety}

Byly změřeny tři světelné scény se zataženými roletami u oken,
abychom eliminovali dopad denního světla.  Největší kontrasty jsou pozorovány u grafu přímého světla,
což jsme očekávali, protože svítidla  disponují nižším výkonem něž u nepřímého osvětlení
a nerozptylují se od okolních ploch. Kontrasty na tabuli jsou optimální a u všech grafů podobné.
Největší odrazy jsou pozorovány od lesklých lavic. Na stropě vidíme, jak je krásně světlý od osvětlení
nepřímým světlem a rovnoměrně rozložen. Odráží se do prostoru a vytváří homogenní prostředí.

Rozložením jasů v bodě 9 lze hodnotit kladně.Grafické znázornění ukazuje relativně
rovnoměrné rozložení intenzity osvětlení po celé ploše učebny.

\medskip \noindent {\sbf Graf (\ref[jasovka2]) - Měření v bodě 2 – zatažené rolety}

Měření v bodě 2 změřeny tři světelné scény se zataženými roletami u oken,
aby byl eliminován dopad denního světla. Z grafů vidíme, že výsledky jsou velmi podobné,
jako v bodě 9, což naznačuje správné rozložení jasů v celém prostoru.

\medskip \noindent {\sbf Graf (\ref[jasovka3]) - Měření v bodě 2 – vytažené rolety}

Měření v bodě 2 s roztaženými roletami u oken, bylo provedeno z důvodu získání informací o jasech
s vlivem přirozeného světla. Učebna je více oslňována i navzdory
tomuto nárůstu je osvětlení stále v přijatelném rozmezí a interaktivní tabule je stále čitelná.

\medskip \noindent {\sbf Shrnutí:}

Grafy z měření v bodech 9 a 2 s různými světelnými scénami naznačují, že rozložení jasů v~učebně je optimální.
Svítidla zajišťují rovnoměrné osvětlení s minimalizovanými kontrasty a odrazy.
Přirozené světlo dopadající z oken má pozitivní vliv na osvětlení v bodě 2, aniž by narušovalo studijní prostředí.






% Z~grafů vidíme, že v~bodě 9 nejsou žácí oslňování přirozeným světlem a~současně je jas tabule dostatečný.

% V~bodě 2 je jas tabule dostatečný, ale žáci mohou být oslňování přirozeným světlem.

% Byla provedena I jasova analyza v ucebna, abychom meli prestavu o~odleskach na tabuli a~viditelnosti na interaktivni tabuli.

%     • Analyza byla provedena na bodu 2 a~9 viz obrazek pudorysu - 120 cm, na urovni oka
%     \medskip
%     • tyto body byly zvoleny z duvodu ze jsou nejkritictejsi - denni svetlo + odraz svetel na tabuli\
%     \medskip
%     • byl pouzit jasovy analyzator sestaveny jednim profesorem - fotka , nazev, jmeno profesora
%     \medskip
%     • bylo porizovano 9 snimku (jasova mapa) v casech : 1s(vykresli se vice tmava mista); 0,5s; 1/10s; 1/50s; 1/250s; 1/1000s; 1/2000s; 1/4000s; 1/8000s
%     \medskip
%     • 35 mm + fish eye - bez filtru
%     \medskip
%     • zpracovani vysledku probehlo v programu LumiDISP - od profesoru z VUT

%     • jasova analyza – jas na tabuli, na lavicich, muzeme spocitat UGR – oslneni – mel by byt odbry u toho osvetleni, mozna u primeho to bude silnejsi
%     \medskip
%     • celkovy jas v plose
%     \medskip
%     • prime je tmavsi
%     \medskip
%     • klidne porovnat bod 2 – zatazene – denni svetlo, jak se lisi
%     \medskip
%     • dat tam veci, co jen chci – a prijdou mi zajimave
%     \medskip
%     • klidne popsat pomoci oka
%     \medskip
%     • vetsi kontrasty jsou u primeho osvetleni nez u neprimeho osvetleni
%     \medskip
%     • neprime osvetleni ma svetly strop a tim padem cele to rozlozeni svetla v mistnosti je homogenejsi – v obou pripadech – u bodu 9 a 2
%     \medskip
%     • prime je slabsi – je doplnkove, zakladni je neprime osvetleni
%     \medskip
%     • u primeho je tmavsi strop, tak je videt velky kontrast mezi svitidlem a stropem za nim
%     \medskip
%     • u kombinovaneho je kontrast mensi a take mensi riziko oslneni
%     \medskip
%     • bod 9 – dominantni pohled
%     \medskip
%     • bod 2
%     • rozlozeni jasu je homogeni, moc se ty jasy nelisi
%     \medskip
%     • celni stena ma tmabou barvu se ukazuje u neprime osvetleni jako tmavsi a mensi jas, je otazka jestli je to vhodne
%     \medskip
%     • jasy jsou prijemne, ten rozsah neni tak veliky


















