\sec Čtvrté měření

Čtvrté měření proběhlo 22.4.2024 - 17:00 - 18:30.
Byla provedena jasová analýza pro různé světelné scény, abychom zjistili kontrasty na tabuli.
Meření proběhlo pomocí jasového analyzátoru LDA (Luminance Distribution Analyzer). Je určený pro hodnocení
jasu pomocí metadat digitální fotografie.
\medskip
K~měření byl použit fotoaparát značky Nikon typ D7200 se snímačem CMOS, 23,5~mm $\times$ 15,6~mm.
Zaostření bylo nastaveno mezi 1 a~3 a~míra citlivosti ISO na 100 a~clona na 4. Výška čočky
objektivu byla zvolena na 1,2~m, což odpovídá normové výšce oka sedícího člověka.
\medskip
Měření proběhlo ve ve dvou měřících bodech. V~bodu 2 a~9 (viz půdorys).
\medskip
Bod 2 - Při pohledu na tabuli může být student vystaven oslnění přirozeného světla, které proniká z~prvního
okna a~odlesky z~lesklých povrchů lavic.
\medskip
Bod 9 -  Pokud sedí student v~zadní lavici u~zdi, může se vystavovat oslnění z~umělých světelných zdrojů nebo
odrazů světla z~přirozených zdrojů na lavicích a~židlích.
\medskip
Pro každý kontrolní bod bylo pořízeno devět snímků na složení HDR obrazu. Snímky
byly zachyceny s~rychlostí závěrky 0,5s; 1/10s; 1/50s; 1/250s; 1/1000s; 1/2000s; 1/4000s; 1/8000s. Objektiv byl
zvolený fish eye bez filtru.

Naměřená data byla následně zpracována v~programu LumiDISP.

\medskip
\medskip \clabel[jasovka]{Jasová analýza}
\picw=14cm \cinspic 04_Grafy/04_mereni/bod9a.png
\caption/f Jasová analýza bodu 9
\medskip

\medskip
\medskip \clabel[jasovka2]{Jasová analýza 2}
\picw=14cm \cinspic 04_Grafy/04_mereni/bod2b.png
\caption/f Jasová analýza bodu 2
\medskip
\medskip

% \mnote{\inoval{\vbox{hodnocení}}}

Výsledky hodnotíme z hlediska oslnění žáků a~svítivosti tabule. Pro dobrou viditelnost budeme uvažovat
hodnoty jasu 300 -- 500~cd/m$^2$, tomu odpovídá přibližně světle červená a~až oranžová barva
v~obrázcích \ref[jasovka] a~\ref[jasovka2].
Barvy zelená a~modrá, jas 0 -- 50~cd/m$^2$m znamenají příliš slabý jas a~nemohou způsobit oslnění.
Barva žlutá, jas nad 1000~cd/m$^2$ může znamenat oslnění.


Z~grafů vidíme, že v~bodě 9 nejsou žácí oslňování přirozeným světlem a~současně je jas tabule dostatečný.

V~bodě 2 je jas tabule dostatečný, ale žáci mohou být oslňování přirozeným světlem.

% Byla provedena I jasova analyza v ucebna, abychom meli prestavu o odleskach na tabuli a~viditelnosti na interaktivni tabuli.

%     • Analyza byla provedena na bodu 2 a~9 viz obrazek pudorysu - 120 cm, na urovni oka
%     \medskip
%     • tyto body byly zvoleny z duvodu ze jsou nejkritictejsi - denni svetlo + odraz svetel na tabuli\
%     \medskip
%     • byl pouzit jasovy analyzator sestaveny jednim profesorem - fotka , nazev, jmeno profesora
%     \medskip
%     • bylo porizovano 9 snimku (jasova mapa) v casech : 1s(vykresli se vice tmava mista); 0,5s; 1/10s; 1/50s; 1/250s; 1/1000s; 1/2000s; 1/4000s; 1/8000s
%     \medskip
%     • 35 mm + fish eye - bez filtru
%     \medskip
%     • zpracovani vysledku probehlo v programu LumiDISP - od profesoru z VUT




















