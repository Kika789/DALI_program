\sec Čtvrté měření

Čtvrté měření proběhlo 22.4.2024 v čase 17:00--18:30.
Byla provedena jasová analýza pro různé světelné scény, abychom zjistili kontrasty na tabuli.
Meření proběhlo pomocí jasového analyzátoru \glref{LDA} (Luminance Distribution Analyzer). Je určený pro hodnocení
jasu pomocí metadat digitální fotografie.

K~měření byl použit fotoaparát značky {\sbf Nikon typ D7200} se snímačem CMOS, 23,5~mm $\times$ 15,6~mm.
Zaostření bylo nastaveno mezi 1 a~3 a~míra citlivosti ISO na 100 a~clona na 4. Výška čočky
objektivu byla zvolena na 1,2~m, což odpovídá normové výšce oka sedícího člověka.

Měření proběhlo ve ve dvou měřících bodech. V~bodu 2 a~9 (viz půdorys).

Bod 2 - Při pohledu na tabuli může být student vystaven oslnění přirozeného světla, které proniká z~prvního
okna a~odlesky z~lesklých povrchů lavic.

Bod 9 -  Pokud sedí student v~zadní lavici u~zdi, může se vystavovat oslnění z~umělých světelných zdrojů nebo
odrazů světla z~přirozených zdrojů na lavicích a~židlích.

Pro každý kontrolní bod bylo pořízeno devět snímků na složení \glref{HDR} obrazu. Snímky
byly zachyceny s~rychlostí závěrky 0,5s; 1/10s; 1/50s; 1/250s; 1/1000s; 1/2000s; 1/4000s; 1/8000s. Objektiv byl
zvolený fish eye bez filtru.

Byly nastaveny tři světelné scény, kde byly zapnuty světelná zařízení na maximální intenzitě - kombinované, přímé a~nepřímé.

Naměřená data byla následně zpracována v~programu {\sbf LumiDISP}.

\medskip
\medskip \clabel[jasovka]{Jasová analýza}
\picw=14cm \cinspic 04_Grafy/04_mereni/bod9a.png
\caption/f Jasová analýza bodu 9
\medskip

\medskip
\medskip \clabel[jasovka2]{Jasová analýza 2}
\picw=14cm \cinspic 04_Grafy/04_mereni/bod2b.png
\caption/f Jasová analýza bodu 2
\medskip
\medskip

\medskip
\medskip \clabel[jasovka3]{Jasová analýza 2b}
\picw=14cm \cinspic 04_Grafy/04_mereni/bod2a.png
\caption/f Jasová analýza bodu 2b
\medskip
\medskip

% \mnote{\inoval{\vbox{hodnocení}}}

% Zhodnocení grafů

Výsledky hodnotíme z hlediska oslnění žáků a~svítivosti tabule. Pro dobrou viditelnost budeme uvažovat
hodnoty jasu 300 -- 500~cd/m$^2$, tomu odpovídá přibližně světle červená a~až oranžová barva
v~obrázcích \ref[jasovka] a~\ref[jasovka2].
Barvy zelená a~modrá, jas 0 -- 50~cd/m$^2$m znamenají příliš slabý jas a~nemohou způsobit oslnění.
Barva žlutá, jas nad 1000~cd/m$^2$ může znamenat oslnění.


Z~grafů vidíme, že v~bodě 9 nejsou žácí oslňování přirozeným světlem a~současně je jas tabule dostatečný.

V~bodě 2 je jas tabule dostatečný, ale žáci mohou být oslňování přirozeným světlem.

Byla provedena I jasova analyza v ucebna, abychom meli prestavu o~odleskach na tabuli a~viditelnosti na interaktivni tabuli.

    • Analyza byla provedena na bodu 2 a~9 viz obrazek pudorysu - 120 cm, na urovni oka
    \medskip
    • tyto body byly zvoleny z duvodu ze jsou nejkritictejsi - denni svetlo + odraz svetel na tabuli\
    \medskip
    • byl pouzit jasovy analyzator sestaveny jednim profesorem - fotka , nazev, jmeno profesora
    \medskip
    • bylo porizovano 9 snimku (jasova mapa) v casech : 1s(vykresli se vice tmava mista); 0,5s; 1/10s; 1/50s; 1/250s; 1/1000s; 1/2000s; 1/4000s; 1/8000s
    \medskip
    • 35 mm + fish eye - bez filtru
    \medskip
    • zpracovani vysledku probehlo v programu LumiDISP - od profesoru z VUT

    • jasova analyza – jas na tabuli, na lavicich, muzeme spocitat UGR – oslneni – mel by byt odbry u toho osvetleni, mozna u primeho to bude silnejsi
    \medskip
    • celkovy jas v plose
    \medskip
    • prime je tmavsi
    \medskip
    • klidne porovnat bod 2 – zatazene – denni svetlo, jak se lisi
    \medskip
    • dat tam veci, co jen chci – a prijdou mi zajimave
    \medskip
    • klidne popsat pomoci oka
    \medskip
    • vetsi kontrasty jsou u primeho osvetleni nez u neprimeho osvetleni
    \medskip
    • neprime osvetleni ma svetly strop a tim padem cele to rozlozeni svetla v mistnosti je homogenejsi – v obou pripadech – u bodu 9 a 2
    \medskip
    • prime je slabsi – je doplnkove, zakladni je neprime osvetleni
    \medskip
    • u primeho je tmavsi strop, tak je videt velky kontrast mezi svitidlem a stropem za nim
    \medskip
    • u kombinovaneho je kontrast mensi a take mensi riziko oslneni
    \medskip
    • bod 9 – dominantni pohled
    \medskip
    • bod 2 
    • rozlozeni jasu je homogeni, moc se ty jasy nelisi
    \medskip
    • celni stena ma tmabou barvu se ukazuje u neprime osvetleni jako tmavsi a mensi jas, je otazka jestli je to vhodne
    \medskip
    • jasy jsou prijemne, ten rozsah neni tak veliky


















