\chap Fyziologie vidění

Zrakový vjem je výsledným projevem zrakového systému člověka, který zajišťuje příjem, přenos a~zpracování
informace přenášené světlem v~komplexu nervových podráždění.

\medskip\noindent
Zrakový systém člověka se skládá ze tří částí \cite[denni_osvetleni]:
\begitems
    * {\sbf oko:} periferní část,
    * {\sbf zrakový nerv:} spojovací část,
    * {\sbf zraková centra v~mozku:} centrální část.
\enditems

Tyto funkce mají podstatný význam pro posouzení světelné pohody či nepohody, protože osvětlení  ovlivňuje
zrakové vnímání, ale i~další funkce organismu, včetně jeho psychiky. Proto je nutnou podmínkou znalost základní
anatomie a~fyziologie zrakového systému.

\medskip \clabel[oko]{Řez lidským okem}
\picw=10cm \cinspic 03_Obrazky/oko2.png
% \vbox{\caption/f Řez okem }
% \vbox{\centerline{(\url{https://orgpad.com/o/ACe7hVNypGPI31bVzVWI5I})}}
\caption/f Řez lidským okem \cite[soustava_smyslova]
\medskip


Tvar oka (viz obr.~\ref[oko]) je přibližně kulový, pro snadný pohyb v~očnici. Průměrná velikost oční bulvy u~dospělého člověka je 24 mm.

\medskip\noindent
Stěna oka v~zadní části  je tvořena třemi základnímu vrstvami:
\begitems
    * {\sbf bělima:} nosná vrstva,
    * {\sbf cévnatka:} vrstva výstelky, která mimo jiné zajišťuje výživu sítnice,
    * {\sbf sítnice:} obsahuje dva typy světločivných buněk.
\enditems

Mezi světločivné buňky patří tyčinky (v počtu asi $130\cdot10^6$), čípky (asi $7\cdot10^6$) a~\glref{ipRGC} buňky, kde
nejvíce z~nich jsou uloženy ve žluté skvrně (optická osa oka) a~žádný z~nich nenalezneme ve slepé skvrně \cite[denni_osvetleni].

Množství světla vstupujícího do oka je korigováno průměrem zornice, která je součástí duhovky.
Oční čočka promítá obraz pozorovaného předmětu na sítnici, která se skládá z~deseti vrstev,
kde v~poslední vrstvě jsou světločivné buňky. Tyčinky slouží k~vidění za šera -- {\sbf skotopické vidění},
jsou velmi citlivé i~při velice nízkých světelných podnětech, ale nejsou schopny rozlišit barevnost
světla, avšak pomocí nich jsme schopni s~vysokým rozlišením vnímat různé jasy v~zorném poli od bílé,
přes různé odstíny až po černou. Čípky slouží pro vidění ve dne -- {\sbf fotopické vidění},
kdy je zapotřebí dostatečně silné osvětlení. Čípky jsou schopny velice dobře rozlišovat barvy světla,
ale pouze omezeně registrovat rozdílnost jasů. Přechodem mezi oběma viděním je vidění {\sbf mezopické},
kdy se současně uplatňují tyčinky a~čípky \cite[svetelna_technika].

Podstatou vidění je fotochemický děj, odehrávající se na sítnici za pomoci sítnicových pigmentů, které se
rozkládají účinkem světla. Světločivné buňky nejsou tedy drážděny přímo světlem, ale chemickým procesorem
v~závislosti na koncentraci sítnicových pigmentů.

\medskip\noindent
K~subjektivnímu hodnocení zrakové pohody slouží zrakový orgán, mezi kritéria patří:
\begitems
    * akomodace,
    * adaptace,
    * fototropický reflex,
    * vlastnosti zorného pole,
    * rychlost vnímání,
    * rozlišovací schopnost,
    * oslnění,
    * spektrální citlivost zraku.
\enditems



\medskip\noindent
{\sbf Akomodace}

Akomodace je přizpůsobení oka při zaostřování na různě vzdálené předměty.
Při zaostření na dálku se napnou oční svaly, oční čočka se zploští a~zornice se rozšíří, při zaostření
do blízka jsou svaly povolené, oční čočka je širší a~zornice se zúží.
K~akomodaci dochází samovolně, jakmile se zadíváme na nějaký předmět, ale můžeme ji vyvolat částečně i~vědomě,
např. když schválně rozostříme. Ve vyšším věku se schopnost akomodace snižuje\cite[svetelna_technika].

\medskip
Akomodační rozsah:

\label[akomodacni_rozsah]
$$A_R  = {1\over r_1}+{1\over r_2} \eqmark $$

\thistable{\tabskip=0pt plus1fil minus1fil}
\table to 9cm{llll}{
kde   & $A_R$ &-& akomodační rozsah [D] (dioptrie), \cr
      & $r_1$ &-& je vzdálenost blízkého bodu [m], \cr
      & $r_1$ &-& je vzdálenost vzdáleného bodu [m]. \cr
}

\medskip\noindent
{\sbf Adaptace}

Adaptace je přizpůsobení zraku na různé jasy a~hladiny osvětlenosti. Zdravé oko je schopno pojmout $2\cdot10^{-9}$~lx
a~k~vidění dochází v~rozmezí od 0,25 do 100 000~lx.

Při adaptaci zraku na vyšší jas se zmenšuje citlivost světločivných buněk a~trvá 1~minutu s~dozníváním 10~minut.
Adaptace na nižší jas trvá delší dobu, a~to po dobu 20 minut se zvyšuje citlivost čípků a~tyčinek a~pak dalších 40~minut
se ještě zvyšuje citlivost tyčinek\cite[svetelna_technika].

Na rozlišovací schopnost zraku má vliv adaptační jas. Je to jas, na který je zrak v~konkretním prostředí a~konkrétním
čase adaptován. Jsme schopni rozlišit plochy s~jasem 1:3 v~prostředí s~malým jasem a~v~prostředí s~vysokým
jasem jsme schopni rozlišit 1:1,01, proto je zapotřebí ho zohlednit při návrhu osvětlení.

Pozor si musíme dát u~negativního jevu readaptace, což znamená časté střídaní rozdílných jasů v~čase kratším, než
umožňuje adaptační mechanismus sítnice. Může vzniknout při nerovnoměrném osvětlení interiéru a~narušit tak zrakovou pohodu\cite[denni_osvetleni].

\medskip\noindent
{\sbf Fototropický reflex}

U~tohoto jevu se oči automaticky obracejí k~místu v~zorném poli s~nejvyšším jasem nebo s~nejvyšším kontrastem jasů.
Při nuceném překonávání dochází ke zrakové únavě\cite[denni_osvetleni].


\medskip\noindent
{\sbf Zorné pole}

Tímto pojmem je myšlena část prostoru, kterou může pozorovatel sledovat bez pohybu hlavy a~oka.

\medskip \clabel[zorne pole]{Zorné pole}
\picw=14cm \cinspic 03_Obrazky/zorne_pole2.png
% \vbox{\caption/f Zorné pole}
% \vbox{\centerline{(\url{https://www.powerwiki.cz/attach/A5M15ES1/A5M15ES1-02-Zrak.pdf})}}
\caption/f Zorné pole \cite[osluneni_budov]
\medskip

Přesně člověk vidí v~uhlovém rozsahu 8° ve vodorovném směru a~6° ve svislém směru. Největší ostrost vidění by měla
být v~rozsahu 1.5°. Zorné pole (viz obr.~\ref[zorne pole]) se zmenšuje se zmenšujícím jasem.

Pro posouzení zrakové náročnosti je kritický detail, který se umisťuje do centra zorného pole \cite[denni_osvetleni].

\medskip\noindent
{\sbf Rychlost vnímání}

Oči se neustále pohybují a~na sítnici se obraz mění rychlostí přibližně pětkrát za sekundu. Schopnost vnímání se zvyšuje s~jasem
pozorovaného detailu a~s rostoucím kontrastem mezi detaily a~pozadím. Toto je důležité pozorovat u~prostor, kde
dochází k~rychlým pohybům jako je sportovní hala a~průmyslové stavby\cite[svetelna_technika].

\medskip\noindent
{\sbf Rozlišovací schopnost zraku}

Schopnost rozlišení předmětů pozorovatelem v~zorném poli je zapotřebí aby předměty měly dostatečně rozdílné jasy
nebo barvy \cite[denni_osvetleni]. Možnost pozorování detailu na stejnobarevném pozadí je závislá na kontrastu jasů k[-]

$$k={{|L_a-L_b|}\over L_b} \eqmark $$

% kde  $L_a$ je jas kritického detailu [$cd \cdot m^{-2}$]

% $L_b$ je jas bezprostředního okolí [$cd \cdot m^{-2}$]

\thistable{\tabskip=0pt plus1fil minus1fil}
\table to 9cm{llll}{
kde
      & $L_a$ &-& je jas kritického detailu [$cd \cdot m^{-2}$] \cr
      & $L_b$ &-& je jas bezprostředního okolí [$cd \cdot m^{-2}$] \cr
}

\medskip\noindent
{\sbf Oslnění}

Jedná se o~nežádoucí stav zraku, který ruší nebo zhoršuje zrakovou pohodu, případně znemožňuje vidění.
Příčinou bývá příliš velký jas nebo jeho nevhodné rozložení v~zorném poli\cite[denni_osvetleni].

\medskip\noindent
Podle stupně rozeznáváme:
\begitems
* {\sbf rušivé:} pozorovatel si neuvědomuje příčinu oslnění,
* {\sbf omezující:} vidění se stává namáhavým,
* {\sbf oslepující:} je znemožněno vidění.
\enditems

\medskip\noindent
{\sbf Spektrální citlivost zraku}

Lidské oko není citlivé na všechny barvy stejně. Nejvyšší citlivost máme na žluté světlo. Citlivost na barvu
světla se liší při vidění za šera - skotopické vidění a~vidění ve dne - fotopické vidění a~může se lišit
u~jednotlivých osob.
Z~těchto důvodů byla přijata dohoda Mezinárodní komisí pro osvětlenost \glref{CIE} o~spektrální
citlivosti zraku => {\sbf Normální fotometrický pozorovatel} - jedná se o~osobu v~populaci s~průměrnou
spektrální citlivostí zraku. Při fotopickém vidění je citlivost normálního a~fotometrického pozorovatele
největší pro světlo základní vlnové délky $λ_m$ = 555 nm (viz obr.~\ref[ucinnost]). Od této vlnové délky na obě strany spektra
citlivost klesá, kdy při vlnových délkách kratších než 380 nm a~delších než 770 nm je téměř nulová\cite[denni_osvetleni]\cite[svetelna_technika]\cite[svetloaosvetlovani].

\medskip

\medskip \clabel[ucinnost]{Poměrná světelná účinnost monochromatického osvětlení}
\picw=10cm \cinspic 03_Obrazky/citlivost2.png
\caption/f Poměrná světelná účinnost monochromatického osvětlení \cite[citlivost_oka]

\medskip

\noindent {\sbf Nezrakové vnímání světla }

% tri typy - tycinky, cipky a ipRGC bunky (neobrazove vnimani  svetla). ve starsi literature nejsou, protoze identifikovany v 2001.
% Tahle cast vam tam chybi a je dulezita, protoze kvuli temto bunkam se vyviji prave nove LED zdroje s azurovou spektr, slozkou. Pokud to v praci nikde nemate, mela byste doplnit. Zdroj informaci v cestine najdete napr. v me disertaci, ktera je na DSpace.

% Doporucuji zde v poznamce k sitnici jen zminit ipRGC bunky, a na konec kapitoly jeste vlozit stat o nezrakovem vnimani svetla.
Tento termín je důležité zmínit, protože nevizuální vnímání světla nevytváří vizuální obraz v~mozku,
ale poskytuje informace o~přítomnosti nebo nepřítomnosti světla v~prostředí \cite[circadian_vision]. Původně se předpokládalo, že za
tyto procesy mohou tyčinky a~čípky, ale nyní víme, že tomu tak není, protože v~roce 1998 byl objeven melanopsin,
protein citlivý na světlo, což vedlo k~dalším vědeckým výzkumům. V~roce 2001 určil George C. Brainard a~jeho tým vrchol
spektrální citlivosti pro neobrazové vnímání na hodnotu mezi 446–477 nm, odpovídající melanopsinu.
V~roce 2002 identifikoval Samer Hattar a~jeho tým nový fotoreceptor na sítnici – gangliové buňky obsahující melanopsin,
nyní známé jako intrinsically photosensitive Retinal Ganglion Cells (\glref{ipRGC}). Díky tomuto objevu se dnes vyvíjejí nové LED
zdroje s~azurovou spektrální složkou\cite[melanopsin2]\cite[melanopsin3].



