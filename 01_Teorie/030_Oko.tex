\chap Fyziologie vidění

Zrakový vjem je výsledným projevem zrakového systému člověka, který zajišťuje příjem, přenos a zpracování
informace přenášené světlem v~komplex nervových podráždění.

\medskip
Zrakový systém člověka se skládá ze tří částí:
\begitems
    * oko - periferní
    * zrakový nerv - spojovací
    * zraková centra v~mozku - centrální
\enditems

Tyto funkce mají podstatný význam pro posouzení světelné pohody či nepohody, protože osvětlení  ovlivňuje
zrakové vnímání , ale i další funkce organismu, včetně jeho psychiky. Proto je nutnou podmínkou znalost základní
anatomie a fyziologie zrakového systému.

\medskip \clabel[oko]{Řez okem}
\picw=10cm \cinspic 03_Obrazky/oko2.png
\vbox{\caption/f Řez okem }
\vbox{\centerline{(převzato z: \url{https://orgpad.com/o/ACe7hVNypGPI31bVzVWI5I})}}
\medskip


Tvar oka je přibližně kulový, pro snadný pohyb v~očnici. Průměrná velikost oční bulvy u~dospělého člověka je 24 mm.

\medskip
Stěna oka v~zadní části  je tvořena třemi základnímu vrstvami:
\begitems
    * bělima - nosná vrstva
    * cévnatka - funkce výživy
    * sítnice - obsahuje dva typy světločivných buněk
\enditems

Mezi světločivné buňky patří tyčinky (v počtu asi $130\cdot10^6$) a čípky (asi $7\cdot10^6$), kde
nejvíce z~nich jsou uloženy ve žluté skvrně (optická osa oka) a žádný z~nich nenalezneme ve slepé skvrně.

Množství světla vstupujícího do oka je korigováno průměrem zornice, která je součástí duhovky.
Oční čočka promítá obraz pozorovaného předmětu na sítnici, která se skládá z~deseti vrstev,
kde v~poslední vrstvě jsou světločivné buňky. Tyčinky slouží k~vidění za šera - {\sbf skotopické vidění},
jsou velmi citlivé i při velice nízkých světelných podnětech, ale nejsou schopny rozlišit barevnost
světla, avšak pomocí nich jsme schopni s~vysokým rozlišením vnímat různé jasy v~zorném poli od bílé,
přes různé odstíny až po černou. Čípky slouží pro vidění ve dne - {\sbf fotopické vidění},
kdy je zapotřebí dostatečně silné osvětlení. Čípky jsou schopny velice dobře rozlišovat barvy světla,
ale pouze omezeně registrovat rozdílnost jasů. Přechodem mezi oběma viděním je vidění {\sbf mezopické},
kdy se současně uplatňují tyčinky a čípky.

Podstatou vidění je fotochemický děj, odehrávající se na sítnici za pomoci sítnicových pigmentů, které se
rozkládají účinkem světla. Světločivné buňky nejsou tedy drážděny přímo světlem, ale chemickým procesorem
v~závislosti na koncentraci sítnicových pigmentů.

\medskip
K~subjektivnímu hodnocení zrakové pohody slouží zrakový orgán, mezi kritéria patří:
\begitems
    * akomodace
    * adaptace
    * fototropický reflex
    * vlastnosti zorného pole
    * rychlost vnímání
    * rozlišovací schopnost
    * oslnění
    * spektrální citlivost zraku
\enditems



\medskip\noindent
{\bf Akomodace}

Akomodace je přizpůsobení oka při zaostřování na různě vzdálené předměty.
Při zaostření na dálku se napnou oční svaly, oční čočka se zploští a zornice se rozšíří, při zaostření
do blízka jsou svaly povolené, oční čočka je širší a zornice se zúží.
K~akomodaci dochází samovolně, jakmile se zadíváme na nějaký předmět, ale můžeme ji vyvolat částečně i vědomě,
např. když schválně rozostříme. Ve vyšším věku se schopnost akomodace snižuje.

\medskip
Akomodační rozsah:

\label[akomodacni_rozsah]
$$A_R [D] = {1\over r_1}+{1\over r_2} \eqmark $$

\thistable{\tabskip=0pt plus1fil minus1fil}
\table to 8cm{llll}{
, kde
      & $r_1$ &-& je vzdálenost blízkého bodu \cr
      & $r_1$ &-& je vzdálenost vzdáleného bodu  \cr
}

\medskip
{\bf Adaptace}

Adaptace je přizpůsobení zraku na různé jasy a hladiny osvětlenosti. Zdravé oko je schopno pojmout $2\cdot10^{-9}$ lx a
k~vidění dochází v~rozmezí od 0,25 do 100 000 lx.

Při adaptaci zraku na vyšší jas se zmenšuje citlivost světločivných buněk a trvá 1 minutu s~dozníváním 10 minut.
Adaptace na nižší jas trvá delší dobu, a to po dobu 20 minut se zvyšuje citlivost čípků a tyčinek a pak dalších 40 minut
se ještě zvyšuje citlivost tyčinek.

Na rozlišovací schopnost zraku má vliv adaptační jas. Je to jas, na který je zrak v~konkretním prostředí a konkrétním
čase adaptován. Jsme schopni rozlišit plochy s~jasem 1:3 v~prostředí s~malým jasem a v~prostředí s~vysokým
jasem jsme schopni rozlišit 1:1,01, proto je zapotřebí ho zohlednit při návrhu osvětlení.

Pozor si musíme dát u~negativního jevu readaptace,což znamená časté střídaní rozdílných jasů v~čase kratším, než
umožňuje adaptační mechanismus sítnice. Může vzniknout při nerovnoměrném osvětlení interiéru a narušit tak zrakovou pohodu.

\medskip
{\bf Fototropický refelex}

U~tohoto jevu se oči automaticky obracejí k~místu v~zorném poli s~nejvyšším jasem nebo s~nejvyšším kontrastem jasů.
Při nuceném překonávání dochází ke zrakové únavě.
z~knížky zdroj \rfc{ozdrojovat}

\medskip
{\bf Zorné pole}

Tímto pojmem je myšlena část prostoru, kterou může pozorovatel sledovat bez pohybu hlavy a oka.

\medskip \clabel[zorne pole]{Zorné pole}
\picw=12cm \cinspic 03_Obrazky/zorne_pole.png
\vbox{\caption/f Zorné pole}
\vbox{\centerline{(https://www.powerwiki.cz/attach/A5M15ES1/A5M15ES1-02-Zrak.pdf)}}
\medskip


Přesně člověk vidí v~uhlovém rozsahu 8° ve vodorovném směru a ve svislém směru 6°. Největší ostrost vidění by měla
být v~rozsahu 1.5°. Zorné pole se zmenšuje se zmenšujícím jasem.

Pro posouzení kritické náročnosti je kritický detail, který se umisťuje do centra zorného pole.


\medskip
{\bf Rychlost vnímání}

Oči se neustále pohybují a obraz na sítnici se mění rychlostí cca 5 obrázků za sekundu. Rychlost vnímání roste s~jasem
pozorovaného detailu a dále se zvyšováním kontrastu jasu detailu a pozadí. Toto je důležité pozorovat u~prostor, kde
dochází k~rychlým pohybům jako je sportovní hala a průmyslové stavby.


\medskip
{\bf Rozlišovací schopnost zraku}

Schopnost rozlišení předmětů pozorovatelem v~zorném poli je zapotřebí aby předměty měly dostatečně rozdílné jasy
nebo barvy. Možnost pozorování detailu na stejnobarevném pozadí je závislá na kontrastu jasů k[-]
$$k={|L_a-L_b|}\over L_b$$ [-]
kde  $L_a$ je jas kritického detailu [$cd \cdot m^{-2}$]

$L_b$ je jas bezprostředního okolí [$cd \cdot m^{-2}$]



\medskip
{\bf Oslnění}

Jedná se o~nežádoucí stav zraku, který ruší nebo zhoršuje zrakovou pohodu, případně znemožňuje vidění.
Příčinou bývá příliš velký jas nebo jeho nevhodné rozložení v~zorném poli.

\medskip
Podle stupně rozeznáváme:
\begitems
* rušivé - pozorovatel si neuvědomuje příčinu oslnění
* omezující - vidění se stává namáhavým
* oslepující - je znemožněno vidění
\enditems

\medskip
{\bf Spektrální citlivost zraku}

Lidské oko není citlivé na všechny barvy stejně. Nejvyšší citlivost máme na žluté světlo. Citlivost na barvu
světla se liší při vidění za šera - skotopické vidění a vidění ve dne - fotopické vidění a může se lišit
u~jednotlivých osob.
Z~těchto důvodů byla přijata dohoda Mezinárodní komisí pro osvětlenost CIE o~spektrální
citlivosti zraku => {\sbf Normální fotometrický pozorovatel} - jedná se o~osobu v~populaci s~průměrnou
spektrální citlivostí zraku. Při fotopickém vidění je citlivost normálního a fotometrického pozorovatele
největší pro světlo základní vlnové délky $λ_m$ = 555 nm. Od této vlnové délky na obě strany spektra
citlivost klesá, kdy při vlnových délkách kratších než 380 nm a delších než 770 nm je téměř nulová.

\medskip

\medskip \clabel[ucinnost]{Poměrná světelná účinnost monochromatického osvětlení}
\picw=12cm \cinspic 03_Obrazky/citlivost_graf.png
\caption/f Poměrná světelná účinnost monochromatického osvětlení


