
% Zároveň máme vícero způsobů, jak dosáhnout požadovaného osvětlení. Můžeme
% mít klasické celkové osvětlení prostoru, kdy je prostor rovnoměrně osvětlen v celé své
% ploše. V případě, že v určité části je zapotřebí intenzivnějšího osvětlení, se navrhuje
% osvětlení odstupňované. Dalším typem osvětlení je osvětlení místní a~bodové, jehož
% funkcí je zvýšit osvětlenost na určité ploše. Sloučení výše zmíněných typů umělého
% osvětlení se nazývá kombinované osvětlení. Mimo jiné se rozlišuje umělé osvětlení i~na
% základě směru osvětlení. Zde rozlišujeme dva typy osvětlení, a~to přímé a~nepřímé. Přímé
% osvětlení, jak už název napovídá, osvětluje danou plochu přímo, např, běžné osvětlení
% ve školních učebnách. Nepřímé osvětlení osvětluje plochu pomocí odrazu světla např.
% od stropu nebo stěny, čímž eliminujeme oslnění nebo tvorbu stínů. Nicméně je vždy
% důležitý vhodný návrh a~následná volba světelného zdroje tak, aby byly splněny všechny
% požadavky světleného prostředí.

% kdyztak se inspirovat od Krause diplomky - jak zakomponovat směr osvětlení?


\chap Umělé osvětlení

Umělým osvětlením v~uzavřených prostorech se snažíme, co nejvíce přiblížit dennímu světlu, aby se docílilo pracovního komfortu.

% Lidé tráví stále více času v uzavřených prostorech, ať už doma,
% v práci, ve škole, nebo na veřejných místech.
% Proto hraje důležitou roli kvalita osvětlení,
% která ovlivňuje nejen náš komfort, ale i zdraví a produktivitu.

% Umělé osvětlení slouží k zajištění dostatečné viditelnosti v situacích,
% kdy je přirozené denní světlo nedostatečné nebo zcela chybí.

\noindent Umělé osvětlení dělíme do kategorií podle způsobu instalace a funkce~\cite[svetloaosvetlovani]:
\begitems
    * {\sbf Celkové osvětlení:} Rovnoměrně osvětluje celou plochu bez ohledu na specifické zrakové úkoly.
    * {\sbf Odstupňované osvětlení:} Zajišťuje vyšší úroveň osvětlenosti v místech zrakového úkolu a zároveň osvětluje okolní prostory s nižší intenzitou.
    * {\sbf Místní osvětlení:} Slouží k doplňkovému osvětlení specifických oblastí, kde je nutná vysoká úroveň osvětlenosti.
    * {\sbf Nouzové osvětlení:} Aktivuje se v případě výpadku běžného osvětlení a zajišťuje minimální viditelnost pro evakuaci a bezpečnost.
\enditems

% \medskip\noindent
% Umělé osvětlení dělíme dle \cite[svetloaosvetlovani]:

% \begitems
% * {\sbf celkové} -- rovnoměrné osvětlení v~celém uvažovaném prostoru bez ohledu na zvláštní místní požadavky;
% * {\sbf odstupňované} -- v~místech kde přesně známe rozmístění pracovišť, svítidla se umisťují tak, aby vytvořila
%     vyšší hladinu osvětlenosti v~místech zrakového úkolu a~zároveň osvětlila přilehlé plochy na odpovídající hladinu osvětlenosti;
% * {\sbf místní} -- osvětlení podle zrakového úkolu, které doplňuje celkové osvětlení a~lze jej samostatně ovládat;
% * {\sbf nouzové}  -- navržené k~použití v~situaci, kdy dojde k~poruše běžného osvětlení.
% \enditems


Kromě typu instalace se umělé osvětlení dělí i podle směru, jakým světlo dopadá na osvětlenou plochu:
\begitems
    * {\sbf Přímé osvětlení:} Světlo dopadá na plochu přímo ze světelného zdroje, čímž se dosahuje vysoké intenzity.
    * {\sbf Nepřímé osvětlení:} Světlo dopadá na plochu odrazem od stropu, stěn nebo jiných povrchů.
        Toto osvětlení působí méně oslnivě a vytváří příjemnější atmosféru.
\enditems

% Kromě jiných faktorů se umělé osvětlení dělí i podle směru,
% jakým světlo dopadá na danou plochu.
% V tomto ohledu rozlišujeme dva základní typy:

% \begitems
% * {\sbf Přímé osvětlení} - dopadá na osvětlovanou plochu přímo ze světelného zdroje.
% * {\sbf Nepřímé osvětlení} - osvětluje plochu odraženým světlem, které se odráží od stropu, stěn nebo jiných povrchů. Díky tomu se minimalizuje oslnění a tvorba stínů.
% \enditems

\sec Světlo

Světlo, které vnímáme zrakem, představuje formu elektromagnetického záření.
Toto záření se šíří prostorem rychlostí světla a~můžeme ho charakterizovat
buď podle jeho vlnové délky, nebo frekvence (viz obr.~\ref[sireni_svetla]).
Čím kratší je vlnová délka, tím vyšší je energie dané vlny.
Naopak, světlo s~delší vlnovou délkou disponuje nižší energií,
ale má schopnost pronikat hlouběji do materiálu.
Tento jev souvisí s~mírou absorpce a~rozptylu světla v~různých tkáních.
Světlo s~delší vlnovou délkou je méně absorbováno a~rozptýleno,
a proto se může šířit hlouběji do tkáně.
Vlnová délka světla se obvykle udává v~nanometrech (nm)~\cite[jak_funguje_svetlo].

\medskip
\vbox{\clabel[sireni_svetla]{Šíření světla}
\picw=10cm \cinspic 03_Obrazky/vlna.png
\caption/f Šíření světla \cite[jak_funguje_svetlo]}
\medskip

% Elektromagnetické spektrum, rozsah světla, zahrnuje různé vlnové délky s~různými frekvencemi a~barvami od ultrafialového
% až po infračervené světlo (obr. \ref[barevne_spektrum]).
% Toto spektrum obsahuje jak viditelné světlo, které je vnímáno lidským okem, tak i~neviditelné světlo, jako je blízké
% infračervené záření (NIR). Vnímání barev je závislé na vlnové délce světla a~způsobu, jakým je interpretuje lidský
% zrak a~mozek. Viditelné světlo tvoří jen malou část celkového elektromagnetického spektra.


Viditelné světlo tvoří pouhou část celého elektromagnetického spektra.
Rozkládá se v rozmezí vlnových délek přibližně od 380 nm (fialová) do 780 nm (červená)
(obr.~\ref[barevne_spektrum] a~tab.~\ref[spektrum])
a umožňuje nám vnímat svět v jeho barevné rozmanitosti.
Vnímání jednotlivých barev je závislé na specifické vlnové délce dopadajícího
světla a~způsobu jejího zpracování lidským zrakem a~mozkem.


% I~když běžné světelné zdroje, jako jsou žárovky, zářivky nebo sluneční záření, vnímáme jako bílé světlo,
% ve skutečnosti se skládají z~různých barevných světel s~různými vlnovými délkami. Bílé světlo může být vytvořeno
% mícháním světel různých barev, a~to i~v~kombinaci menšího počtu barev. Například barevný obraz v~televizi
% vzniká kombinací tří barevných světel - červeného, zeleného a~modrého.

I když běžné světelné zdroje, jako jsou žárovky, zářivky nebo sluneční záření,
vnímáme jako bílé světlo, ve skutečnosti se jedná o~komplexní směs barevných světel
s~různými vlnovými délkami.
Vnímání bílé barvy je důsledkem lidského zraku, který integruje
informace z celého spektra viditelného světla.

Bílé světlo lze uměle vytvořit smícháním světel různých barev.
Toho principu se využívá například v televizích, kde se obraz skládá
z~pixelů tvořených kombinací červeného, zeleného a~modrého světla (RGB systém).
Tyto primární barvy se v lidském oku smíchají a~vnímáme je jako bílou barvu.

Tento princip míchání barev se nazývá aditivní syntéza. Kromě ní existuje
i subtraktivní syntéza, která využívá princip odčítání světla.
Například barevné filtry pohlcují specifické vlnové délky světla
a propouštějí jen zbylé barvy, čímž se mění výsledná barva procházejícího světla~\cite[jak_funguje_svetlo].


\medskip \clabel[barevne_spektrum]{Barevné spektrum}
\picw=10cm \cinspic 03_Obrazky/spektrum.png
\caption/f Barevné spektrum \cite[vnimanibarev]
\medskip


\midinsert \clabel[spektrum]{Berevné spektrum}
\ctable{lc}{
{\sbf Spektrum}                 & {\sbf Vlnová délka} \crl \tskip 4pt
Ultrafialové (UV světlo) & 100 -- 400 \cr
Modré světlo             & 380 -- 500 \cr
Zelené světlo            & 520 -- 560 \cr
Žluté světlo             & 570 -- 590 \cr
Červené světlo            & okolo 650 \cr
Blízké infračervené (NIR) světlo & 700 -- 1000 \cr
}
\caption/t Popis barevného spektra
\endinsert

\sec Základní veličiny

\rfc{Asi nejak uvest, ze se jedna o veliciny, ktere se pouzivaji pri hodnoceni osvetleni}

% \medskip
% Základní veličinou je {\sbf intenzita osvětlení E}:

% $$ E={I \over r^2} \cos α $$

% \medskip
% , kde pro bodový zdroj o~svítivosti I [cd] s paprsky dopadajícími pod úhlem $\alpha$ [°] k~normále plochy ve vzdálenosti r [m].

% \medskip \clabel[logo]{Osvětlení}
% \picw=5cm \cinspic 02_Obrazky/osvetleni.png
% \caption/f Dopad světla
\begitems
* {\sbf Osvětlenost (intenzita osvětlení) [E]} : 1 lx (lux) -- Veličina udává,
jak je určitá plocha osvětlována (obr.~\ref[lcx]), tedy podíl
\enditems

% $$ E={I \over r^2} \cos α \quad , \eqmark $$
$$ E={{{\rm d} \it Φ} \over {\rm d} A}  \quad , \eqmark $$

\medskip
kde dΦ [lm] je světelný tok dopadající na plochu dA [m$^2$].

% r [m] je vzdálenost od zdroje a $\alpha$ [°] je úhel mezi paprsky dopadajícími na plochu a normálou k ploše.

% kde pro bodový zdroj o~svítivosti I [cd] s paprsky dopadajícími pod úhlem $\alpha$ [°]
% k~normále plochy ve vzdálenosti r [m].

% \medskip \clabel[logo]{Osvětlení}
% \picw=5cm \cinspic 03_Obrazky/osvetleni.png
% \caption/f Dopad světla

\begitems
* {\sbf Světelný tok [Φ]} : 1 lm (lumen) -- Světelný tok udává, kolik světla celkem vyzáří zdroj do všech směrů (obr.~\ref[lcx]).
    Jde o~světelný výkon, který je posuzován z~hlediska lidského oka.
*{\sbf Svítivost [I]} : 1 cd (kandela) -- Veličina udává, kolik světelného toku Φ vyzáří světelný zdroj nebo svítidlo
    do prostorového úhlu Ω v~určitém směru (obr.~\ref[lcx]).
\enditems

\medskip \clabel[lcx]{Vztah mezi lm, cd a lx}
\picw=8cm \cinspic 03_Obrazky/lumen_candela_lux2.png
\caption/f Vztah mezi lumenem, kandelou a~luxem \cite[cp_znamenaji_pojmy]
\medskip

\begitems
* {\sbf Jas [L]} : 1 cd m$^{-2}$ (kandela na metr čtvereční)
    -- Jas je měřítkem pro vjem jasnosti svítícího nebo osvětlovaného povrchu.
* {\sbf Měrný světelný výkon [η]} : 1 lm W$^{-1}$ (lumen na watt) --
    Udává s~jakou účinností je ve zdroji světla elektřina přeměňována na světlo, t.j. kolik
    lm světelného toku se získá z~1 W elektrického příkonu.
* {\sbf Teplota chromatičnosti [$\bf{T_c}$]} : 1 K~(kelvin) -- Teplotou chromatičnosti zdroje je označována ekvivalentní
    teplota tzv. černého zářiče (Planckova), při které je spektrální složení záření těchto dvou zdrojů blízké.
* {\sbf Index barevného podání [Ra]} : 1 (bezrozměrná veličina) --
    Index barevného podání je veličina, která udává, jak věrně zdroj světla podává barvy.
    Hodnota indexu se pohybuje v~rozmezí 0 až 100, kde 100 znamená, že zdroj světla podává barvy věrně
    jako při osvětlení referenčním zdrojem.
    % Každý světelný zdroj by měl podávat svým světelným tokem barvy okolí věrohodně, jak je známe u~přirozeného
    % světla nebo od světla žárovek.
* {\sbf Index oslnění \glref{UGRL} (-)} : 1 (bezrozměrná veličina) --
    Index oslnění, logaritmická veličina vyjadřující míru oslnění v rozsahu
    od 5 (nejmenší) do 40 (nejvyšší), kvantifikuje nepříjemné vnímání jasových
    rozdílů v zorném poli lidského oka. Tyto rozdíly, pokud přesahují mez
    adaptability zraku, vedou k oslnění, které omezují funkci zraku a~narušují zrakovou pohodu.

    % Index oslnění je logaritmická veličina, která udává míru oslnění v rozsahu přibližně 5 až 40,
    % kde 5 je nejmenší míra oslnění a 40 největší.
    % Pokud se vyskytují v~zorném poli lidského oka příliš velké
    % jasové rozdíly, které výrazně překračují mez adaptability zraku, vzniká oslnění.
    % Tím je omezena činnost zrakového ústrojí a tak narušena zraková pohoda.
\enditems

\sec Umělé osvětlení ve vzdělávacích prostorech

% Normy zabývající se touto problematikou:
Normy zabývající se legislativními požadavky a~dalšími doporučeními na osvětlení školních učeben
z~hlediska technických a světelně​ hygienických požadavků:


\begitems
* ČSN EN 17037 (730582) Denní osvětlení budov.
* ČSN 730580-1:2007 Denní osvětlení budov. Část 1: Základní požadavky (Změna Z3 z~08/2019).
* ČSN 730580-3:1994 Denní osvětlení budov. Část 3: Denní osvětlení škol (Změna Z3 z~08/2019)
ČSN EN 12464-1 (360450) Světlo a~osvětlení - Osvětlení pracovních prostorů - Část 1: Vnitřní pracovní prostory, 2022.
* ČSN EN 12665 (36 0001) Světlo a~osvětlení - Základní termíny a~kritéria pro stanovení požadavků na osvětlení.
* ČSN 36 0011-1 Měření osvětlení vnitřních prostorů - Část 1: Základní ustanovení.
* ČSN 36 0011-3 Měření osvětlení vnitřních prostorů - Část 3: Měření umělého osvětlení.
* ČSN 36 0020 Sdružené osvětlení - Základní požadavky.
\enditems

\noindent {\sbf Norma ČSN EN 12464-1 Světlo a osvětlení - Osvětlení pracovišť - Část 1: Vnitřní pracoviště}

\noindent
Norma stanovuje požadavky na osvětlení pro vnitřní pracoviště z~hlediska zrakové pohody a~zrakového výkonu osob s~normálním nebo korigovaným zrakem.

\medskip\noindent {\sbf Světelné prostředí}

\noindent Světelné prostředí dbá na tyto základní lidské potřeby:
\begitems
* zraková pohoda - aby se pracovníci cítili dobře a~přispívalo to k~produktivní a~kvalitní práci;
* zrakový výkon - zda jsou pracovníci schopni pracovat dlouho a~vydávat dobrý výkon při obtížných podmínkách;
* bezpečnost.
\enditems

\medskip\noindent {\sbf Rozložení jasu}

\noindent Nesprávné rozložení jasu ovlivňuje zrakovou pohodu, proto je nutné vyloučit:
\begitems
* příliš velké jasy a kontrasty jasu, které mohou způsobovat oslnění;
* příliš velké změny jasy, které mohou způsobovat únavu;
* příliš malé jasy, které vytváří monotónní prostředí.
\enditems

\noindent Abychom docílili vyváženého rozložení jasu, musí se zohlednit jasy všech povrchů,
          které jsou určeny jejich činiteli odrazu a osvětlenosti.

\medskip\noindent {\sbf Činitele odrazu povrchů}

\medskip\noindent Doporučené hodnoty odrazu povrchů:
\begitems
* strop: 0,7 až 0,9;
* stěny: 0,5 až 0,8;
* podlaha: 0,2 až 0,6;
* hlavní předměty (nábytek, strojní vybavení apod.) má být v~rozmezí: 0,2 až 0,7.
\enditems

% {\sbf Osvětlenost povrchů}

% osvětlenost stěn a stropů spolu s činiteli odrazu povrchů přispívají k jasům a jsou ukazateli vnímané jasnosti místnosti.

\medskip\noindent {\sbf Osvětlenost}

Osvětlenost je důležitým parametrem při posuzování zrakové pohody a~výkonu.
Místa, která se osvětlují jsou místa zrakových úkolů a~činností,
bezprostřední okolí a~pozadí, stěny, strop a~předměty v~prostoru. (viz obr.~\ref[ukol])

\medskip \clabel[ukol]{Oblasti zrakového úkonu}
\picw=8cm \cinspic 03_Obrazky/ukol.png
\caption/f Oblasti zrakového úkonu
\medskip

Ve školských zařízení místem zrakového úkolu je většinou sešit nebo učebnice studenta, jeho
bezprostředním okolím je lavice a pozadím je podlaha při činnosti psaní a čtení. U výkladu se zapojením tabule
je zrakovým úkolem tabule, bezprostředním okolím jsou zbylé části tabule a pozadím jsou okolní stěny.


\medskip\noindent
Doporučená řada osvětlenosti v luxech podle EN 12665:
{ 5 -- 7,5 -- 10 -- 15 -- 20 -- 30 -- 50 -- 75 -- 100 -- 150 -- 200 -- 300 --
500 -- 750 -- 1000 -- 1 500 -- 2 000 -- 3 000 -- 5 000.}


Tyto hodnoty jsou pouze doporučené. V případě, že se předpokládá, že student
bude na určitém místě plnit zrakový úkol delší dobu,
než je obvyklé, doporučuje se zvýšit požadavek na osvětlení o jeden stupeň. Naopak, pokud student tráví
na daných místech kratší dobu
nebo pokud jsou kritické detaily, na které se student zaměřuje, rozměrné (např. tělocvična),
případně pokud detaily disponují vysokým kontrastem, je možné požadavek na osvětlení o stupeň snížit.

% {\sbf Osvětlenosti místa zrakového úkolu }
% \begitems \style o
% * Psychologická hlediska, zraková a celková pohoda
% * Požadavky dle různých činností pro zabezpečení zrakové pohody
% * Zraková ergonomie
% * Praktické zkušenosti z historie a z výzkumů
% * Provozní bezpečnost
% * Hospodárnost
% \enditems

\midinsert \clabel[vztah]{Vztah mezi osvětlenostmi}
\ctable{|c|c|}
{ \crl
Osvětlenost místa zrakového & Osvětlenost bezprostředního \cr
  úkolu (lx)                &         okolí úkolu (lx)  \crl
\tskip 4pt
    $\geq$ 750    &    500           \cr
    500           &    300           \cr
    300           &    520 -- 560    \cr
    200           &    570 -- 590    \cr
    $\leq$150     &    stejná        \crl
}
\caption/t Vztah mezi osvětlenostmi bezprostředního okolí a osvětlenostmi místa zrakového úkolu
\endinsert

\begitems
* Osvětlení v bezprostředním okolí zrakového úkolu nebo místa činnosti by mělo odpovídat osvětlení daného
  úkolu či místa a zajišťovat rovnoměrné rozložení světla v~zorném poli~(tab.~\ref[vztah]).
  Šířka osvětleného pásu v~okolí zrakového úkolu by měla dosahovat minimálně 0,5 metru.
* Světlo v~bezprostředním okolí zrakového úkolu může být slabší než světlo nad samotným úkolem,
  ale nesmí klesnout pod hodnoty uvedené v tabulce.
\enditems



\noindent {\sbf Rovnoměrnost osvětlení}

Pro rovnoměrnost osvětlení je definován následující vztah:

$$ U = { E_{min} \over E_{m}} \quad , \eqmark $$

kde $E_{min}$ je minimální hodnota udržované osvětlenosti [lx],
$E_{m}$ je střední hodnota udržované osvětlenosti [lx].

Posuzuje se bez přítomnosti denního světla  a platí jen pro elektrické osvětlení.

\begitems
* Rovnoměrnost osvětlenosti bezprostředního okolí musí  být $U_o \ge 0,40$.
* Rovnoměrnost osvětlenosti pozadí, stěn a stropu musí být $U_o \ge 0,10$.
\enditems


\noindent {\sbf Norma ČSN 73 0580-3 stanovuje požadavky na denní osvětlení škol. Školské stavby jsou děleny na:}
\begitems
* předškolní zařízení,
* základní školy,
* střední školy.
\enditems

V učebnách je zásadní volba osvětlovacích soustav, které jsou vybrány s ohledem na efektivitu a~pohodlí žáků.
Svítidla jsou umístěna tak, aby minimalizovala oslnění odrazem a~přímým oslnění.
Proto se vyhýbáme umístění svítidel kolmo nad lavicemi
a~preferujeme jejich podélné umístění, jelikož svítidla mají v~příčné rovině obvykle vyšší jas než v~podélné.

Srovnávací rovina se v učebnách středních škol umísťuje ve výšce 850 mm nad podlahou, pro předškolní zařízení 450 mm
a v tělocvičnách na úrovni podlahy.
Zároveň se vylučuje pruh o~šířce 500 mm od stěn, aby se minimalizovaly odrazy.

Pracovní plochy jsou navrženy s rozptylnou, nelesklou úpravou, s doporučenými hodnotami činitele odrazu světla mezi 0,3 a~0,45.
Prostor lavic a~stolu učitele je považován za místo zrakového úkolu,
a~proto je průměrná udržovaná osvětlenost 300 lx s rovnoměrností 0,7.

Zvláštní pozornost je věnována osvětlení tabule, která musí být dostatečně osvětlena pro snadné sledování.
Při přechodu z tabule na lavice (a zpět) dochází ke změně pohledu a~oko se musí přizpůsobit
různým pozorovacím vzdálenostem, jasu a~kontrastu.

Vnitřní prostory s obrazovkami nebo displeji jsou navrženy s ohledem na denní osvětlení a~vzájemný vztah mezi obrazovkami
a osvětlovacími otvory je pečlivě zohledněn. Aby nedocházelo k~rušivým odrazům nebo oslnění, obrazovky jsou ideálně umístěny tak,
aby denní světlo přicházelo převážně ze strany nebo shora.











%%%%%%%%%%%%%%%%%%%%%%%%%%%%%%%%%%%%%%%% zatim zaramovat
% Je nezbytné dbát na správné osvětlení ve školních prostorech, neboť děti v předškolním a~školním věku jsou ve fázi vývoje,
% a~proto je důležité zajistit jim dostatečné denní světlo.
% K vyhodnocení dostupnosti denního osvětlení se obvykle používá model zatažené oblohy v zimě, který nezávisí na světových stranách.
% Za takových podmínek je slunce skryto za mraky a~obloha působí jako celkový zdroj světla.
% Jas oblohy se mění v závislosti na výšce nad horizontem, a~proto jsou příslušné požadavky relevantní pro všechny místnosti bez ohledu na orientaci.
% \medskip
% Zraková pohoda je stav, který splňuje hygienické standardy a~závisí na intenzitě a~kvalitě denního osvětlení. Při hodnocení příspěvku denního světla se zohledňuje dostupnost denního světla na daném místě a~charakteristiky prostoru, jako jsou vnější překážky, průchodnost zasklením, tloušťka stěn a~střech, vnitřní členění, odrazový koeficient povrchů, vnitřní vybavení apod.

% \secc Hygienické požadavky na stavby - denní osvětlení škol:
% Ověřit!!!! Normy ??? :
% \begitems
% * ČSN EN 17037 (08/2019)
% * ČSN 730580-1:2007 Denní osvětlení budov. Část 1:Základní požadavky (Změna Z3 z 08/2019)
% * ČSN 730580-3:1994 Denní osvětlení budov. Část 3:Denní osvětlení škol (Změna Z3 z 08/2019)
% \medskip
% {\sbf Vyhlášky:}
% \medskip
% *Vyhláška 465/2016 Sb. kterou se mění vyhláška č. 410/2005 Sb., o~hygienických požadavcích na prostory a~provoz zařízení a~provozoven pro výchovu a~vzdělávání dětí a~mladistvých, ve znění vyhlášky č. 343/2009 Sb.
% (předškolní a~školní - vyjma škol vysokých)

% *Tyto veškeré požadavky jsou závazné zákonem o~územním plánování a~stavebním řádu (z.č. 183/2006 Sb.) a~prováděcí vyhláškou č.268/2009 Sb. O technických požadavcích na stavby.
% \enditems

% \secc Technické a~normové požadavky:   ------ nevim jestli zminovat

% V nových a~rekonstruovaných objektech školských zařízení musí vyhovovat dennímu osvětlení dle normových hodnot vnitřní prostory s trvalým pobytem lidí a~prostory, ve kterých se uživatelé střídají v krátkodobém pobytu, ale celková doba pobytu v nich má trvalý charakter.
% Trvalý pobyt (čl.3.1.3.ČSN 730580-1):
% \medskip
% Trvalý pobyt je pobyt lidí ve vnitřním prostoru nebo v jeho funkčně vymezené části, který trvá v průběhu jednoho dne (za denního světla) déle než 4 hodiny a~opakuje se při trvalém užívání budovy více než jednou týdně.
% \medskip
% Do prostor s trvalým pobytem lidí jsou ve školách například zařazeny učebny kmenové i~víceúčelové, pracovny, posluchárny, studovny, kabinety, pracovny vyučujících, kanceláře, sborovny atd. Za prostor s vyhovujícím denním světlem se považuje prostor, v němž je dosaženo hodnoty cílové osvětlenosti na části srovnávací roviny uvnitř prostoru nejméně po polovinu doby s denním světlem.
% \medskip
% V prostorech se svislými nebo šikmými osvětlovacími otvory musí být na srovnávací rovině zároveň splněna hodnota minimální cílové osvětlenosti.
% \medskip
% Srovnávací rovina se umisťuje do výšky 850 mm nad podlahou, pokud není uvedeno jinak. Při hodnocení lze z důvodů eliminace singularit malou část srovnávací roviny vynechat. Z oblasti sítě hodnotících bodů uvnitř prostoru se má vyloučit pruh o~šířce 500 mm od stěn, pokud není uvedeno jinak.
% \medskip
% Hodnoty cílových osvětleností, minimálních cílových osvětleností a~části srovnávací roviny jsou uvedeny v ČSN EN 17037 tab. A1
% \medskip
% {\sbf Další normy}
% \begitems
%     * ČSN EN 12464-1 (360450) Světlo a~osvětlení - Osvětlení pracovních prostorů - Část 1: Vnitřní pracovní prostory, 2022
%     * ČSN EN IEC 62386 Digitální adresovatelné rozhraní pro osvětlení, 2018
%       a~další (Alexandr Sizov)
%     * ČSN EN 12665 (36 0001) Světlo a~osvětlení - Základní termíny a~kritéria pro stanovení požadavků na osvětlení
%     * ČSN 36 0011-1 Měření osvětlení vnitřních prostorů - Část 1: Základní ustanovení
%     * ČSN 36 0011-3 Měření osvětlení vnitřních prostorů - Část 3: Měření umělého osvětlení
%     * ČSN 36 0020 Sdružené osvětlení - Základní požadavky
%     * ČSN EN 12464-1:2004 (36 0450) Světlo a~osvětlení - Osvětlení pracovních prostorů - Část 1: Vnitřní pracovní prostor
%     * ČSN 73 0580-1 Denní osvětlení budov - Část 1: Základní požadavky
% \enditems

% {\sbf Vnitřní povrchy:}

% Povrchy pracovních ploch se navrhují vždy s rozptylnou, nelesklou úpravou. Doporučuje se, aby měli hodnoty činitele odrazu světla v mezích 0,3 až 0,45.
% \medskip
% Tabule v učebnách a~posluchárnách se navrhují s činitelem odrazu světla nejméně 0,1 a~se snadno čistitelným rozptýleným povrchem, který si tyto vlastnosti udržuje i~po dlouhodobém používání. Toto pravidlo se nevztahuje na tabule, na které se nepíše křídou.
% \medskip
% Ve vnitřních prostorech, vyžadující soustředěnou práci, se používá chladnějších, klidných barevných odstínů.
% \medskip
% Vnitřní prostory s obrazovkami:
% \medskip
% Ve vnitřních prostorech s pravidelně používanými obrazovkami nebo přístroji s displeji se navrhuje denní osvětlení a~vzájemný vztah obrazovek a~displejů k~osvětlovacím otvorům tak, aby:
% \medskip
% \begitems
%     * nevznikly rušivé obrazy světla na obrazovkách nebo displejích zrcadlením osvětlovacích otvorů
%     * nebyla úroveň denního osvětlení povrchu obrazovek nebo displejů tak velká, že by mohla narušovat jejich viditelnost
%     * nevzniklo oslnění velkým jasem osvětlovacích otvorů v blízkosti obvyklého směru pohledu na obrazovku
% \enditems
% \medskip
% Nejvhodnější umístění obrazovek je takové, aby denní světlo přicházelo převážně ze strany nebo seshora.
% \medskip
% Vnitřní prostory s obrazovkami a~displeji se navrhují s plynulou regulací denního osvětlení (nikoliv stupňovitou).
% \medskip
% {\sbf DOPLNIT}
% \begitems
% * Světelné zdroje, svítidla a~osvětlovací soustavy
% * Osvětlení tabule
% \enditems
% \medskip
% {\sbf Zdroje:}
% \begitems
% * https://elektro.tzb-info.cz/osvetleni/21513-denni-osvetleni-ve-skolach-dle-csn-en-17037
% * http://www.odbornecasopisy.cz/svetlo/casopis/tema/osvetlovani-ve-skolach--15781
% * https://elektro.tzb-info.cz/osvetleni/9397-zdrave-svetlo-skoly
% \enditems





% ---------------------------------------------------------------------------------- a~co tohle?

% \medskip
% \begitems
%     *Fyzikální podstata světla
%     * Měření světla
%     *     Úvod do měření osvětlení a~používaných jednotek.
%     *     Vysvětlení α-opických luxů a~jejich významu pro měření osvětlení v souladu s lidským vnímáním.
%     *     Přehled váhových křivek, jako je CIE 1931 a~CIE 1978, a~jejich využití při váhování spektrální citlivosti lidského oka.
%     * Výpočet lux z naměřeného spektra
%     *     Principy a~metody výpočtu osvětlení luxem z naměřeného spektra světla.
%     *     Kosinová korekce a~její vliv na přesnost měření osvětlení.
%        * Vysvětlení důležitosti kosinové korekce a~Lambertova zákona.
% Lambertův povrch - Povrch tělesa, který působí dokonalý rozptyl světla. Jinými slovy povrch, který odráží světlo do všech směrů stejně.
% Lambertův zákon - ???
%     * Kosinová korekce
%     * Lambertův zákon a~Lambertův povrch
%     *     Vysvětlení Lambertova zákona, který popisuje rozptyl světla na matných površích.
%     *     Pochopení Lambertova povrchu a~jeho využití při modelování chování světla v různých prostředích.
%     *     Kosinová korekce: Vysvětlení, jak kosinová korekce zajišťuje rovnoměrnější měření osvětlení pod různými úhly.
%     % * Přehled jednotek používaných k~měření světla:
%     % *     Candela: Definice a~význam v měření světelné intenzity.
%     % *     Lumen: Vysvětlení a~význam při měření světelného toku.
%     % *     Lux: Význam v měření osvětlení na určité ploše.
% \enditems

