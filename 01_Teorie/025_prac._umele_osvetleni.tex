\chap Umělé osvětlení

Umělým osvětlením v~uzavřených prostorech se snažíme, co nejvíce přiblížit dennímu světlu, aby se docílilo pracovního komfortu.

\medskip\noindent
Umělé osvětlení dělíme na:

% Zároveň máme vícero způsobů, jak dosáhnout požadovaného osvětlení. Můžeme
% mít klasické celkové osvětlení prostoru, kdy je prostor rovnoměrně osvětlen v celé své
% ploše. V případě, že v určité části je zapotřebí intenzivnějšího osvětlení, se navrhuje
% osvětlení odstupňované. Dalším typem osvětlení je osvětlení místní a~bodové, jehož
% funkcí je zvýšit osvětlenost na určité ploše. Sloučení výše zmíněných typů umělého
% osvětlení se nazývá kombinované osvětlení. Mimo jiné se rozlišuje umělé osvětlení i~na
% základě směru osvětlení. Zde rozlišujeme dva typy osvětlení, a~to přímé a~nepřímé. Přímé
% osvětlení, jak už název napovídá, osvětluje danou plochu přímo, např, běžné osvětlení
% ve školních učebnách. Nepřímé osvětlení osvětluje plochu pomocí odrazu světla např.
% od stropu nebo stěny, čímž eliminujeme oslnění nebo tvorbu stínů. Nicméně je vždy
% důležitý vhodný návrh a~následná volba světelného zdroje tak, aby byly splněny všechny
% požadavky světleného prostředí.

% kdyztak se inspirovat od Krause diplomky - jak zakomponovat směr osvětlení?

\begitems
* {\bf celkové} -- rovnoměrné osvětlení v~celém uvažovaném prostoru bez ohledu na zvláštní místní požadavky;
* {\bf odstupňované} -- v~místech kde přesně známe rozmístění pracovišť, svítidla se umisťují tak, aby vytvořila
    vyšší hladinu osvětlenosti v~místech zrakového úkolu a~zároveň osvětlila přilehlé plochy na odpovídající hladinu osvětlenosti;
* {\bf místní} -- osvětlení podle zrakového úkolu, které doplňuje celkové osvětlení a~lze jej samostatně ovládat;
* {\bf nouzové}  -- navržené k~použití v~situaci, kdy dojde k~poruše běžného osvětlení.
\enditems
\sec Světlo

Světlo, které vnímáme zrakem, představuje formu elektromagnetického záření.
Toto záření se šíří prostorem rychlostí světla a~můžeme ho charakterizovat
buď podle jeho vlnové délky, nebo frekvence (viz obr. \ref[sireni_svetla]).
% Světlo je formou elektromagnetického záření, které můžeme charakterizovat podle jeho vlnové délky
% nebo frekvence (obr. \ref[sireni_svetla]).
Čím kratší je vlnová délka, tím vyšší je energie dané vlny.
Naopak, světlo s delší vlnovou délkou disponuje nižší energií,
ale má schopnost pronikat hlouběji do materiálu.
Tento jev souvisí s mírou absorpce a~rozptylu světla v různých tkáních.
Světlo s delší vlnovou délkou je méně absorbováno a~rozptýleno,
a proto se může šířit hlouběji do tkáně.
% Čím nižší je vlnová délka, tím vyšší je obsažená energie vlny.
% Naopak, světlo s~delší vlnovou délkou má nižší energii,
% ale může pronikat do materiálu hlouběji. Tento jev je důsledkem různé absorpce a~rozptylu světla různými složkami
% tkáně, což umožňuje světlu s~delší vlnovou délkou pronikat dovnitř až na hlubší úrovně.
Vlnové délky světla se obvykle měří v~nanometrech (nm).

\medskip\clabel[sireni_svetla]{Šíření světla}
\picw=10cm \cinspic 03_Obrazky/vlna.png
\caption/f Šíření světla
\medskip

% Elektromagnetické spektrum, rozsah světla, zahrnuje různé vlnové délky s~různými frekvencemi a~barvami od ultrafialového
% až po infračervené světlo (obr. \ref[barevne_spektrum]).
% Toto spektrum obsahuje jak viditelné světlo, které je vnímáno lidským okem, tak i~neviditelné světlo, jako je blízké
% infračervené záření (NIR). Vnímání barev je závislé na vlnové délce světla a~způsobu, jakým je interpretuje lidský
% zrak a~mozek. Viditelné světlo tvoří jen malou část celkového elektromagnetického spektra.


Viditelné světlo tvoří pouhou část celého elektromagnetického spektra.
Rozkládá se v rozmezí vlnových délek přibližně od 380 nm (fialová) do 780 nm (červená)
(obr.~\ref[barevne_spektrum] a~tab.~\ref[spektrum])
a umožňuje nám vnímat svět v jeho barevné rozmanitosti.
Vnímání jednotlivých barev je závislé na specifické vlnové délce dopadajícího
světla a~způsobu jejího zpracování lidským zrakem a~mozkem.


% I~když běžné světelné zdroje, jako jsou žárovky, zářivky nebo sluneční záření, vnímáme jako bílé světlo,
% ve skutečnosti se skládají z~různých barevných světel s~různými vlnovými délkami. Bílé světlo může být vytvořeno
% mícháním světel různých barev, a~to i~v~kombinaci menšího počtu barev. Například barevný obraz v~televizi
% vzniká kombinací tří barevných světel - červeného, zeleného a~modrého.

I když běžné světelné zdroje, jako jsou žárovky, zářivky nebo sluneční záření,
vnímáme jako bílé světlo, ve skutečnosti se jedná o~komplexní směs barevných světel
s různými vlnovými délkami.
Vnímání bílé barvy je důsledkem lidského zraku, který integruje
informace z celého spektra viditelného světla.

Bílé světlo lze uměle vytvořit smícháním světel různých barev.
Toho principu se využívá například v televizích, kde se obraz skládá
z pixelů tvořených kombinací červeného, zeleného a~modrého světla (RGB systém).
Tyto primární barvy se v lidském oku smíchají a~vnímáme je jako bílou barvu.

Tento princip míchání barev se nazývá aditivní syntéza. Kromě ní existuje
i subtraktivní syntéza, která využívá princip odčítání světla.
Například barevné filtry pohlcují specifické vlnové délky světla
a propouštějí jen zbylé barvy, čímž se mění výsledná barva procházejícího světla.


\medskip \clabel[barevne_spektrum]{Barevné spektrum}
\picw=10cm \cinspic 03_Obrazky/spektrum.png
\caption/f Barevné spektrum
\medskip

\midinsert \clabel[spektrum]{Berevné spektrum}
\ctable{lrrrrr}{
\hfil Spektrum & Vlnová délka \crl \tskip 4pt
Ultrafialové (UV světlo) & 100 -- 400 \cr
Modré světlo             & 380 -- 500 \cr
Zelené světlo            & 520 -- 560 \cr
Žluté světlo             & 570 -- 590 \cr
Črvené světlo            & okolo 650 \cr
Blízké infračervené (NIR) světlo & 700 -- 1000 \cr
}
\caption/t Popis barevného spektra
\endinsert

\sec Základní veličiny

% \medskip
% Základní veličinou je {\sbf intenzita osvětlení E}:

% $$ E={I \over r^2} \cos α $$

% \medskip
% , kde pro bodový zdroj o~svítivosti I [cd] s paprsky dopadajícími pod úhlem $\alpha$ [°] k~normále plochy ve vzdálenosti r [m].

% \medskip \clabel[logo]{Osvětlení}
% \picw=5cm \cinspic 02_Obrazky/osvetleni.png
% \caption/f Dopad světla
\begitems
* {\sbf Osvětlenost (intenzita osvětlení) [E]} : 1 lx (lux) - Veličina udává, jak je určitá plocha osvětlována,
    t.j. kolik lm světelného toku dopadá na 1 $m^2$.
\enditems

$$ E={I \over r^2} \cos α $$

\medskip
, kde pro bodový zdroj o~svítivosti I [cd] s paprsky dopadajícími pod úhlem $\alpha$ [°] k~normále plochy ve vzdálenosti r [m].

% \medskip \clabel[logo]{Osvětlení}
% \picw=5cm \cinspic 03_Obrazky/osvetleni.png
% \caption/f Dopad světla

\begitems
* {\sbf Světelný tok [Φ]} : 1 lm (lumen) - Světelný tok udává, kolik světla celkem vyzáří zdroj do všech směrů.
    Jde o~světelný výkon, který je posuzován z~hlediska lidského oka.
*{\sbf Svítivost [I]} : 1 cd (kandela) - Veličina udává, kolik světelného toku Φ vyzáří světelný zdroj nebo svítidlo
    do prostorového úhlu Ω v~určitém směru.
\enditems

\medskip \clabel[lcx]{lumencadelalux}
\picw=8cm \cinspic 03_Obrazky/lumen_candela_lux2.png
\caption/f Vztah mezi lumenem, kandelou a~luxem
\medskip

\begitems
* {\sbf Jas [L]} : 1 cd m$^{-2}$ (kandela na metr čtvereční)
    - je měřítkem pro vjem světlosti svítícího nebo osvětlovaného povrchu.
* {\sbf Měrný světelný výkon [η]} : 1 lm W$^{-1}$ (lumen na watt) -
    Udává s~jakou účinností je ve zdroji světla elektřina přeměňována na světlo, t.j. kolik
    lm světelného toku se získá z~1 W elektrického příkonu.
* {\sbf Teplota chromatičnosti [Tc]} : 1 K~(kelvin) - Teplotou chromatičnosti zdroje je označována ekvivalentní
    teplota tzv. černého zářiče (Planckova), při které je spektrální složení záření těchto dvou zdrojů blízké.
* {\sbf Index barevného podání [Ra]} : 1 (bezrozměrná veličina) -
    Každý světelný zdroj by měl podávat svým světelným tokem barvy okolí věrohodně, jak je známe u~přirozeného
    světla nebo od světla žárovek.
* {\sbf index oslnění UGRL (-)} : - Pokud se vyskytují v~zorném poli lidského oka příliš velké jasy nebo jejich
    rozdíly, které výrazně překračují mez adaptability zraku, vzniká oslnění. Tím je omezena činnost zrakového ústrojí a
    tak narušena zraková pohoda.
\enditems

\sec Umělé osvětlení ve vzdělávacích prostorech

Normy zabývající se touto problematikou:
\begitems \style o
* ČSN EN 17037 (730582) Denní osvětlení budov.
* ČSN 730580-1:2007 Denní osvětlení budov. Část 1:Základní požadavky (Změna Z3 z 08/2019).
* ČSN 730580-3:1994 Denní osvětlení budov. Část 3:Denní osvětlení škol (Změna Z3 z 08/2019)
ČSN EN 12464-1 (360450) Světlo a~osvětlení - Osvětlení pracovních prostorů - Část 1: Vnitřní pracovní prostory, 2022.
* ČSN EN 12665 (36 0001) Světlo a~osvětlení - Základní termíny a~kritéria pro stanovení požadavků na osvětlení.
* ČSN 36 0011-1 Měření osvětlení vnitřních prostorů - Část 1: Základní ustanovení.
* ČSN 36 0011-3 Měření osvětlení vnitřních prostorů - Část 3: Měření umělého osvětlení.
* ČSN 36 0020 Sdružené osvětlení - Základní požadavky.
\enditems

\noindent
Norma ČSN EN 12464-1 Světlo a osvětlení - Osvětlení pracovišť - Část 1: Vnitřní pracoviště

Norma stanovuje požadavky na osvětlení pro vnitřní pracoviště z hlediska zrakové pohody a zrakového výkonu osob s normálním nebo korigovaným zrakem.

{\bf Světelné prostředí}

Světelné prostředí dbá na tyto základní lidské potřeby:

\begitems \style o
* zraková pohoda - aby se pracovníci cítili dobře a přispívalo to k produktivní a kvalitní práci;
* zrakový výkon - zda jsou pracovníci schopni pracovat dlouho a vydávat dobrý výkon při obtížných podmínkách 
* bezpečnost
\enditems

{\bf Rozložení jasu}
Nesprávné rozložení jasu ovlivňuje zrakovou pohodu, proto je nutné vyloučit:
\begitems \style o
* příliš velké jasy a kontrasty jasu, které mohou způsobovat oslnění;
* příliš velké změny jasy, které mohou způsobovat únavu; 
* příliš malé jasy, které vytváří monotónní prostředí.
\enditems

Abychom docílili vyváženého rozložení jasu, musí se zohlednit jasy všech povrchů, které jsou určeny jejich činiteli odrazu a osvětlenosti.

{\bf Činitele odrazu povrchů:}
Doporučené hodnoty:
\begitems \style o
* strop: 0,7 až 0,9
* stěny: 0,5 až 0,8
* podlaha: 0,2 až 0,6
* hlavní předměty (nábytek, strojní vybavení apod.) má být v rozmezí: 0,2 až 0,7
\enditems

% {\bf Osvětlenost povrchů}

% osvětlenost stěn a stropů spolu s činiteli odrazu povrchů přispívají k jasům a jsou ukazateli vnímané jasnosti místnosti.

{\bf Osvětlenost}

Osvětlenost je důležitým parametrem při posuzování zrakové pohody a výkonu. Místa, která se osvětlují jsou místa zrakových úkolů a činností,
bezprostřední okolí a pozadí, stěny, strop a předměty v prostoru. (viz obr.)

Ve školských zařízení se většinou jedná při psaní a čtení, že místem zrakového úkolu je sešit studenta, jeho
bezprostředním okolím je lavice a pozadím je podlaha. Při výkladu učitele a zapojením tabule do
výuky je zrakovým úkolem četba. Část tabule je tedy jeho místem úkolu, bezprostředním okolím jsou
zbylé části tabule % a doplněk do 60 0 pohledu studenta ( Obr. 20 ). Zbytek do 90 0 od osy pohledu je pozadím.

Doporučená řada osvětlenosti podle EN 12665:

5 - 7,5 - 10 - 15 - 20 - 30 - 50 - 75 - 100 - 150 - 200 - 300 - 500 - 750 - 1000 - 1 500 - 2 000 - 3 000 - 5 000.

% {\bf Osvětlenosti místa zrakového úkolu }
% \begitems \style o
% * Psychologická hlediska, zraková a celková pohoda
% * Požadavky dle různých činností pro zabezpečení zrakové pohody
% * Zraková ergonomie
% * Praktické zkušenosti z historie a z výzkumů
% * Provozní bezpečnost
% * Hospodárnost
% \enditems

Přidat tabulku s osvětlením zrakového úkolu a bezprostředního okolí: % prepsat sbymi slovy
\begitems \style o
* Osvětlenost bezprostředního okolí musí být vztažená k osvětlenosti místa zrakového úkolu nebo místa činnosti a má poskytovat
vyvážené rozložení jasů v zorném poli, Bezprostřední okolí má tvořit pás o šířce alespoň 0,5 m
kolem místa zrakového úkolu v zorném poli.
* Osvětlenost bezprostředního okolí může být menší než osvětlenost místa zrakového úkolu,
avšak nesmí být menší než hodnoty v uvedené tabulce.
\enditems

\midinsert \clabel[vztah]{Vztah mezi osvětlenostmi}
\ctable{lrrrrr}{
\hfil Osvětlenost místa zrakového
úkolu E úkol (lx) & Osvětlenost bezprostředního okolí úkolu (lx) \crl \tskip 4pt
>= 750 & 500 \cr
500    & 300 \cr
300    & 520 -- 560 \cr
200    & 570 -- 590 \cr
<=150  & stejné \cr
}
\caption/t Vztah mezi osvětlenostmi
\endinsert

{\bf Rovnoměrnost osvětlení}

Posuzuje se bez přítomnosti denního světla  a platí jen pro elektrické osvětlení.


\begitems \style o
* Rovnoměrnost osvětlenosti bezprostředního okolí musí  být Uo >= 0,40.
* Rovnoměrnost osvětlenosti pozadí, stěn a stropu musí být Uo >= 0,10.
\enditems

% {\bf Osvětlení pracovních míst se zobrazovacími jednotkami (DSE)}

Přidat tabulky

% \midinsert \clabel[vztah]{Vztah mezi osvětlenostmi}
% \ctable{lrrrrr}{
% \hfil Osvětlenost místa zrakového
% úkolu E úkol (lx) & Osvětlenost bezprostředního okolí úkolu (lx) \crl \tskip 4pt
% >= 750 & 500 \cr
% 500    & 300 \cr
% 300    & 520 -- 560 \cr
% 200    & 570 -- 590 \cr
% <=150  & stejné \cr
% }
% \caption/t Vztah mezi osvětlenostmi
% \endinsert


A pak vypsat z normy pro školní učebny


\noindent
Norma ČSN 73 0580-3 stanovuje požadavky na denní osvětlení škol. Školské stavby jsou děleny na:
\begitems
* předškolní zařízení,
* základní školy,
* střední školy.
\enditems

V učebnách je zásadní volba osvětlovacích soustav, které jsou vybrány s ohledem na efektivitu a~pohodlí žáků.
Svítidla jsou umístěna tak, aby minimalizovala oslnění odrazem a~přímým oslnění. Proto se vyhýbáme umístění svítidel kolmo nad lavicemi
a preferujeme jejich podélné umístění, jelikož svítidla mají v příčné rovině obvykle vyšší jas než v podélné.

Srovnávací rovina se v učebnách středních škol umísťuje ve výšce 850 mm nad podlahou, pro předškolní zařízení 450 mm
a v tělocvičnách na úrovni podlahy.
Zároveň se vylučuje pruh o~šířce 500 mm od stěn, aby se minimalizovaly odrazy.

Pracovní plochy jsou navrženy s rozptylnou, nelesklou úpravou, s doporučenými hodnotami činitele odrazu světla mezi 0,3 a~0,45.
Prostor lavic a~stolu učitele je považován za místo zrakového úkolu,
a~proto je průměrná udržovaná osvětlenost 300 lx s rovnoměrností 0,7.

Zvláštní pozornost je věnována osvětlení tabule, která musí být dostatečně osvětlena pro snadné sledování.
Při přechodu z tabule na lavice (a zpět) dochází ke změně pohledu a~oko se musí přizpůsobit
různým pozorovacím vzdálenostem, jasu a~kontrastu.

Vnitřní prostory s obrazovkami nebo displeji jsou navrženy s ohledem na denní osvětlení a~vzájemný vztah mezi obrazovkami
a osvětlovacími otvory je pečlivě zohledněn. Aby nedocházelo k~rušivým odrazům nebo oslnění, obrazovky jsou ideálně umístěny tak,
aby denní světlo přicházelo převážně ze strany nebo shora.











%%%%%%%%%%%%%%%%%%%%%%%%%%%%%%%%%%%%%%%% zatim zaramovat
% Je nezbytné dbát na správné osvětlení ve školních prostorech, neboť děti v předškolním a~školním věku jsou ve fázi vývoje,
% a~proto je důležité zajistit jim dostatečné denní světlo.
% K vyhodnocení dostupnosti denního osvětlení se obvykle používá model zatažené oblohy v zimě, který nezávisí na světových stranách.
% Za takových podmínek je slunce skryto za mraky a~obloha působí jako celkový zdroj světla.
% Jas oblohy se mění v závislosti na výšce nad horizontem, a~proto jsou příslušné požadavky relevantní pro všechny místnosti bez ohledu na orientaci.
% \medskip
% Zraková pohoda je stav, který splňuje hygienické standardy a~závisí na intenzitě a~kvalitě denního osvětlení. Při hodnocení příspěvku denního světla se zohledňuje dostupnost denního světla na daném místě a~charakteristiky prostoru, jako jsou vnější překážky, průchodnost zasklením, tloušťka stěn a~střech, vnitřní členění, odrazový koeficient povrchů, vnitřní vybavení apod.

% \secc Hygienické požadavky na stavby - denní osvětlení škol:
% Ověřit!!!! Normy ??? :
% \begitems
% * ČSN EN 17037 (08/2019)
% * ČSN 730580-1:2007 Denní osvětlení budov. Část 1:Základní požadavky (Změna Z3 z 08/2019)
% * ČSN 730580-3:1994 Denní osvětlení budov. Část 3:Denní osvětlení škol (Změna Z3 z 08/2019)
% \medskip
% {\sbf Vyhlášky:}
% \medskip
% *Vyhláška 465/2016 Sb. kterou se mění vyhláška č. 410/2005 Sb., o~hygienických požadavcích na prostory a~provoz zařízení a~provozoven pro výchovu a~vzdělávání dětí a~mladistvých, ve znění vyhlášky č. 343/2009 Sb.
% (předškolní a~školní - vyjma škol vysokých)

% *Tyto veškeré požadavky jsou závazné zákonem o~územním plánování a~stavebním řádu (z.č. 183/2006 Sb.) a~prováděcí vyhláškou č.268/2009 Sb. O technických požadavcích na stavby.
% \enditems

% \secc Technické a~normové požadavky:   ------ nevim jestli zminovat

% V nových a~rekonstruovaných objektech školských zařízení musí vyhovovat dennímu osvětlení dle normových hodnot vnitřní prostory s trvalým pobytem lidí a~prostory, ve kterých se uživatelé střídají v krátkodobém pobytu, ale celková doba pobytu v nich má trvalý charakter.
% Trvalý pobyt (čl.3.1.3.ČSN 730580-1):
% \medskip
% Trvalý pobyt je pobyt lidí ve vnitřním prostoru nebo v jeho funkčně vymezené části, který trvá v průběhu jednoho dne (za denního světla) déle než 4 hodiny a~opakuje se při trvalém užívání budovy více než jednou týdně.
% \medskip
% Do prostor s trvalým pobytem lidí jsou ve školách například zařazeny učebny kmenové i~víceúčelové, pracovny, posluchárny, studovny, kabinety, pracovny vyučujících, kanceláře, sborovny atd. Za prostor s vyhovujícím denním světlem se považuje prostor, v němž je dosaženo hodnoty cílové osvětlenosti na části srovnávací roviny uvnitř prostoru nejméně po polovinu doby s denním světlem.
% \medskip
% V prostorech se svislými nebo šikmými osvětlovacími otvory musí být na srovnávací rovině zároveň splněna hodnota minimální cílové osvětlenosti.
% \medskip
% Srovnávací rovina se umisťuje do výšky 850 mm nad podlahou, pokud není uvedeno jinak. Při hodnocení lze z důvodů eliminace singularit malou část srovnávací roviny vynechat. Z oblasti sítě hodnotících bodů uvnitř prostoru se má vyloučit pruh o~šířce 500 mm od stěn, pokud není uvedeno jinak.
% \medskip
% Hodnoty cílových osvětleností, minimálních cílových osvětleností a~části srovnávací roviny jsou uvedeny v ČSN EN 17037 tab. A1
% \medskip
% {\sbf Další normy}
% \begitems
%     * ČSN EN 12464-1 (360450) Světlo a~osvětlení - Osvětlení pracovních prostorů - Část 1: Vnitřní pracovní prostory, 2022
%     * ČSN EN IEC 62386 Digitální adresovatelné rozhraní pro osvětlení, 2018
%       a~další (Alexandr Sizov)
%     * ČSN EN 12665 (36 0001) Světlo a~osvětlení - Základní termíny a~kritéria pro stanovení požadavků na osvětlení
%     * ČSN 36 0011-1 Měření osvětlení vnitřních prostorů - Část 1: Základní ustanovení
%     * ČSN 36 0011-3 Měření osvětlení vnitřních prostorů - Část 3: Měření umělého osvětlení
%     * ČSN 36 0020 Sdružené osvětlení - Základní požadavky
%     * ČSN EN 12464-1:2004 (36 0450) Světlo a~osvětlení - Osvětlení pracovních prostorů - Část 1: Vnitřní pracovní prostor
%     * ČSN 73 0580-1 Denní osvětlení budov - Část 1: Základní požadavky
% \enditems

% {\sbf Vnitřní povrchy:}

% Povrchy pracovních ploch se navrhují vždy s rozptylnou, nelesklou úpravou. Doporučuje se, aby měli hodnoty činitele odrazu světla v mezích 0,3 až 0,45.
% \medskip
% Tabule v učebnách a~posluchárnách se navrhují s činitelem odrazu světla nejméně 0,1 a~se snadno čistitelným rozptýleným povrchem, který si tyto vlastnosti udržuje i~po dlouhodobém používání. Toto pravidlo se nevztahuje na tabule, na které se nepíše křídou.
% \medskip
% Ve vnitřních prostorech, vyžadující soustředěnou práci, se používá chladnějších, klidných barevných odstínů.
% \medskip
% Vnitřní prostory s obrazovkami:
% \medskip
% Ve vnitřních prostorech s pravidelně používanými obrazovkami nebo přístroji s displeji se navrhuje denní osvětlení a~vzájemný vztah obrazovek a~displejů k~osvětlovacím otvorům tak, aby:
% \medskip
% \begitems
%     * nevznikly rušivé obrazy světla na obrazovkách nebo displejích zrcadlením osvětlovacích otvorů
%     * nebyla úroveň denního osvětlení povrchu obrazovek nebo displejů tak velká, že by mohla narušovat jejich viditelnost
%     * nevzniklo oslnění velkým jasem osvětlovacích otvorů v blízkosti obvyklého směru pohledu na obrazovku
% \enditems
% \medskip
% Nejvhodnější umístění obrazovek je takové, aby denní světlo přicházelo převážně ze strany nebo seshora.
% \medskip
% Vnitřní prostory s obrazovkami a~displeji se navrhují s plynulou regulací denního osvětlení (nikoliv stupňovitou).
% \medskip
% {\bf DOPLNIT}
% \begitems
% * Světelné zdroje, svítidla a~osvětlovací soustavy
% * Osvětlení tabule
% \enditems
% \medskip
% {\sbf Zdroje:}
% \begitems
% * https://elektro.tzb-info.cz/osvetleni/21513-denni-osvetleni-ve-skolach-dle-csn-en-17037
% * http://www.odbornecasopisy.cz/svetlo/casopis/tema/osvetlovani-ve-skolach--15781
% * https://elektro.tzb-info.cz/osvetleni/9397-zdrave-svetlo-skoly
% \enditems





% ---------------------------------------------------------------------------------- a~co tohle?

% \medskip
% \begitems
%     *Fyzikální podstata světla
%     * Měření světla
%     *     Úvod do měření osvětlení a~používaných jednotek.
%     *     Vysvětlení α-opických luxů a~jejich významu pro měření osvětlení v souladu s lidským vnímáním.
%     *     Přehled váhových křivek, jako je CIE 1931 a~CIE 1978, a~jejich využití při váhování spektrální citlivosti lidského oka.
%     * Výpočet lux z naměřeného spektra
%     *     Principy a~metody výpočtu osvětlení luxem z naměřeného spektra světla.
%     *     Kosinová korekce a~její vliv na přesnost měření osvětlení.
%        * Vysvětlení důležitosti kosinové korekce a~Lambertova zákona.
% Lambertův povrch - Povrch tělesa, který působí dokonalý rozptyl světla. Jinými slovy povrch, který odráží světlo do všech směrů stejně.
% Lambertův zákon - ???
%     * Kosinová korekce
%     * Lambertův zákon a~Lambertův povrch
%     *     Vysvětlení Lambertova zákona, který popisuje rozptyl světla na matných površích.
%     *     Pochopení Lambertova povrchu a~jeho využití při modelování chování světla v různých prostředích.
%     *     Kosinová korekce: Vysvětlení, jak kosinová korekce zajišťuje rovnoměrnější měření osvětlení pod různými úhly.
%     % * Přehled jednotek používaných k~měření světla:
%     % *     Candela: Definice a~význam v měření světelné intenzity.
%     % *     Lumen: Vysvětlení a~význam při měření světelného toku.
%     % *     Lux: Význam v měření osvětlení na určité ploše.
% \enditems

