\chap Úvod

V dnešní době se vlivem inteligentních technologií mění i~naše prostředí, ve kterem žijeme a pracujeme.
Inteligentní osvětlení se stává běžnou součástí moderních interiérů a~odborné studie poukazují na jeho schopnost
podpořit schopnost koncentrace, zvýšit produktivitu a celkově přispět ke zdraví.

Vzhledem k~tomu, že se většinu času pohybujeme ve vnitřních prostorech,
stává se klíčovým faktorem nastavení správného umělého osvětlení.

Navrhnout osvětlení správně je velkou výzvou. Záleží na mnoha faktorech, jako je
venkovní počasí, velikost a~orientace oken, barva stěn a~nábytku, počet lidí v místnosti,
celková velikost prostoru atd. Také je zapotřebí myslet na volbu vhodného svítidla, zejména v~nočních hodinách.
Vystavovat se modrému světlu po západu Slunce je pro nás nepřirozený jevem.

Snaha je tedy zvolit parametry tak, abychom se nejlépe přizpůsobili dennímu osvětlení, protože je tomu tak
přizpůsobené naše lidské oko, vychází to z historie, kde lidé strávili mnohem více času venku.

Tato práce se zabývá vlivem umělého osvětlení na lidské zdraví a~produktivitu,
s~důrazem na výukové prostředí.
Cílem je analyzovat aktuální světelné podmínky ve vybrané učebně
na Gymnáziu Libeňský zámek a~na základě zjištěných dat navrhnout
optimální osvětlení pro různé výukové situace, které budou ovládány uživatelsky komfortním ovládáním.

Práce se zaměřuje na technické a~světelně hygienické požadavky,
zrakový komfort a~biologické potřeby studentů.
Součástí je i~návrh implementace sběrnicového systému
pro snadnou regulaci osvětlení.

\medskip \medskip \medskip

Práce ve druhé kapitole popisuje různé typy umělého
osvětlení a~základní principy světla, včetně jeho vlnových délek a~charakteristik.
Také se zaměřuje na normy a~požadavky na osvětlení ve vzdělávacích
prostorech a~důležitost správného osvětlení pracovních ploch pro optimální vzdělávací prostředí.

Třetí kapitola podrobně popisuje fyziologii vidění, zahrnující
fungování lidského zrakového systému a~jeho reakci na světlo.
Zrakový systém se skládá z oka, zrakového nervu a~zrakových center v mozku,
které jsou klíčové pro posouzení světelné pohody a~vnímání prostředí.
Fyziologie vidění zahrnuje akomodaci, adaptaci na různé jasové úrovně,
fototropický reflex a~spektrální citlivost zraku.

Čtvrtá kapitola pojednává o~přenosovém protokolu \glref{DALI}, který
využívá Manchesterovského kódování s~přenosovou rychlosti 1200 bps.
Komunikace po sběrnici je asynchronní.
Protokol umožňuje komplexní zprávu všech světelných funkcí, zahrnující komunikaci
s tlačítky a dalšími zařízeními připojenými na síť, jako jsou senzory.

V praktické části diplomové práce byla provedena analýza výukové místnosti na
gymnáziu u Libeňského zámku s~cílem optimalizovat osvětlení pro výuku biologie.
Na základě získaných dat byl navržen nový ovládací panel pro osvětlení,
zjednodušující ovládání a~zlepšující uživatelský komfort.

V kapitole 6 diplomové práce byla provedena analýza umělého osvětlení ve školní učebně
pomocí spektrometru GL Spectis 1.0 Touch.
Měření zahrnovala horizontální i~vertikální rovinu a~poskytla
informace o~osvětlenosti v různých světelných scénách.
Jasová analýza čtvrtého měření ukázala kontrasty v učebně při různých světelných podmínkách.

V kapitole 7 byla provedena simulace osvětlení učebny v programu DIALux,
zahrnující vytvoření 3D modelu, volbu světelných prvků a~nastavení parametrů
světel pro scény výkladu, písemky a~denního světla.
Výsledky simulace poskytly informace o~osvětlenosti na lavicích a~stropě pro jednotlivé scény.

Kapitola 8 se zaměřuje na návrh optimalizovaného světelného ovládání pro učebnu,
které redukuje počet tlačítek na 4 a~přidává 2 tlačítka pro regulaci jasu.
Tento návrh přináší snížení složitosti a~intuitivnější ovládání.
Implementace zahrnuje vytvoření elektrického schématu, plošného spoje, a~využití
řídící jednotky s~mikrokontrolérem \glref{ESP32} WROOM, spolu s~programováním scén pro různé účely.


% Přechodem na inteligentní domácnost se pojí i~inteligentní osvětlení, které má pomoci
% v~lepší koncentraci, produktivitě a~efektivní práci. Toto osvětlení má pozitivní dopady
% na náš zdravotní stav, zrakový komfort, správný vývoj a~také může zabránit negativním
% účinkům, jako je deprese. Z toho důvodu si někteří lidé pořizují speciální biolampy, aby
% měli pocit, že jsou venku na sluníčku. Vzhledem k~tomu, že v dnešní době se většinu času
% pohybujeme ve vnitřních prostorech, stává se klíčovým faktorem nastavení správného
% umělého osvětlení.

% Nestavení správného osvětlení je velkou výzvou. Záleží na mnoha faktorech, jako je
% venkovní počasí, velikost a~orientace oken, barva stěn a~nábytku, počet lidí v místnosti,
% celková velikost prostoru atd. Také je zapotřebí vybrat správný typ osvětlovacích prvků,
% které také závisí na specifické činnosti vykonávané v místnosti. Snaha je nastavit parametry
% tak, abychom se nejlépe přizpůsobili dennímu osvětlení, protože je tomu tak
% přizpůsobené naše lidské oko, vychází to z historie, kde lidé strávili mnohem více času
% venku.

% Svícení ve večerních a~nočních hodinách narušuje biorytmus lidského těla. Ovlivňuje
% tvorbu hormonu melatoninu, který je zodpovědný za spouštění regeneračních procesů
% v těle a~očišťování organismu od toxických látek. Dále melatonin podporuje funkci
% imunitního systému a~chrání organismus před nežádoucími chemickými reakcemi.

% Umělé světlo nemá špatný vliv jen na lidské zdraví, ale ovlivňuje i~zvířata a~hmyz.
% U nichž totiž dochází k~narušení přirozeného biorytmu a~k narušení orientace v noční
% krajině. Světelné zdroje přitahují hmyz a~často se kolem nich pohybují tak dlouho, až
% zahynou vyčerpáním.

% V mé diplomové práci se zaměřuji na umělé osvětlení ve výukové místnosti na gymnáziu
% Libeňský zámek. Cílem je specifikovat legislativní požadavky a~další doporučení
% na osvětlení školních učeben z hlediska technických a~světelně hygienických požadavků.
% Provést analýzu aktuálních světelných podmínek ve sledované učebně. S~pomocí měření
% osvětlenosti na referenčních rovinách a~následně provést jasovou a~spektrální analýzu.

% Z naměřených výsledků navrhnout různé světelné scény odpovídající různým výukovým
% situacím s~důrazem na zrakový komfort a
% biologické potřeby. Dále navrhnout implementaci sběrnicového systému pro realizaci
% navržených světelných scén a~změřit nové parametry osvětlení a~porovnat s~původními
% podmínkami. Na závěr popsat přínosy a~limity zvoleného řešení.

%%%%%%%%%%%%%%%%%%%%%%%%%%%%%%%%%

% Přechodem na inteligentní domácnost se pojí i~inteligentní osvětlení, které má pomoci v~lepší koncentraci,
% produktivitě a~efektivní práci. Toto osvětlení má pozitivní dopady
% na náš zdravotní stav. Vzhledem k~tomu, že v~dnešní době se většinu času pohybujeme ve vnitřních prostorech,
% stává se klíčovým faktorem nastavení správného umělého osvětlení.
% Nastavit správné paramtry osvětlení je velkou výzvou. Záleží na mnoha faktorech, jako je venkovní počasí,
% velikost a~orientace oken, barva stěn a~nábytku, počet lidí v~místnosti,
% celková velikost prostoru atd. Snaha je nastavit parametry tak, abychom se nejlépe přizpůsobili
% dennímu osvětlení, protože je tomu tak přizpůsobené naše lidské oko, vychází to z~historie,
% kde lidé strávili mnohem více času venku. V~dnešní době denní světlo přestalo ovlivňovat délku
% dne a~lidé mohou své aktivity plánovat nezávisla na denní době a~světle. Nicméně se ukázalo, že lidské tělo je stále
% biologicky naladěno na původní, přirozený přísun světla, a~cyklus dne a~noci. Z~tohoto
% hlediska může přísun umělého osvětlení v~nočním čase porušit biologický rytmus
% člověka.

%
% Také je zapotřebí myslet na volbu vhodného svítidla, zejména v~nočních hodinách. Vystavovat se modrému
% světlu po západu Slunce je pro nás nepřirozený jevem.

%
% V~mé diplomové práci se zaměřuji na osvětlení v~učebně biologie na gymnáziu u~Libeňského zámku.
% V~učebně je istalováno moderní integrativní osvětlení s~přímým LED osvětlením a
% plnospektrálním nepřímým osvětlením, přičemž oba typy osvětlení mají vysoký činitel podání barev.
% Nepřímé osvětlení bylo zvoleno z~toho důvodu, abychom se přiblížilu chování světlu
% v~přírodě, kde přímé světlo v~podstatě neexistuje. Uživatelé si moderní osvětlení velmi chválí,
% nicméně byla opmenuta jedna z~důležitých věcí a~to ovládání takového osvětlení.
% Aktuálně je takové osvtělení ovládádno 12 tlačítky a~způsobuje to velké potíže.
% Proto je mým cílem zredukovat tento počet tlačítek na 3 -- 4 tlačítka nasatvit vhodné scény podle biologického
% a~hygienického komfortu.
%%%%%%%%%%%%%%%%%%%%%%%%%%%%------------------------------

% Nicméně bychom neměli volbu svítidel podceňovat a~rovněž by nemělo docházet
% k~jejich nadužívání zejména v nočních hodinách, jelikož se na světlo váže řada
% biologických rytmů v našem těle. Zejména expozice modrému světlu po západu Slunce
% je pro nás nepřirozeným jevem. Nejedná se však zdaleka jen o~dopad na biologické rytmy
% člověka. Tím, že nyní je v trendu osvětlování venkovních prostorů, historických objektů
% a~jiných turistických cílů, výrazně stoupá míra světelného znečištění nejen v České
% republice, ale i~globálně. Toto zbytečné nadužívání umělého osvětlení ovlivňuje přilehlé
% 13ekosystémy a~v nich vyskytující se zvířata a~hmyz, kteří jsou stejně jako lidé zvyklí na
% pravidelný cyklus dne a~noci, nikoliv na nepřetržitou expozici bílému světlu, které je
% typické pro denní světlo. Konkrétní vliv umělého osvětlení bude rozebrán v následující
% kapitole. [10] [11]


% zrakový komfort, správný vývoj a~také může zabránit negativním
% účinkům, jako je deprese. Z toho důvodu si někteří lidé pořizují speciální biolampy, aby
% měli pocit, že jsou venku na sluníčku. Vzhledem k~tomu, že v~dnešní době se většinu času
% pohybujeme ve vnitřních prostorech, stává se klíčovým faktorem nastavení správného
% umělého osvětlení.

%%%%%%%%%%%%%%%%%%%%%%%%%--------------------------------------------------------------------------------------------------------------------
%
% Nestavení správného osvětlení je velkou výzvou. Záleží na mnoha faktorech, jako je venkovní počasí, velikost a~orientace oken,
% barva stěn a~nábytku, počet lidí v~místnosti, celková velikost prostoru atd. Také je zapotřebí vybrat správný typ osvětlovacích prvků,
% které také závisí na specifické činnosti vykonávané v~místnosti. Snaha je nastavit parametry tak, abychom se nejlépe přizpůsobili
% dennímu osvětlení, protože je tomu tak přizpůsobené naše lidské oko, vychází to z~historie, kde lidé strávili mnohem více času venku.

%
% Svícení ve večerních a~nočních hodinách narušuje biorytmus lidského těla. Ovlivňuje tvorbu hormonu melatoninu, který je
% zodpovědný za spouštění regeneračních procesů v~těle a~očišťování organismu od toxických látek. Dále melatonin podporuje
% funkci imunitního systému a~chrání organismus před nežádoucími chemickými reakcemi.

%
% Umělé světlo nemá špatný vliv jen na lidské zdraví, ale ovlivňuje i~zvířata a~hmyz. U~nichž totiž dochází k~narušení
% přirozeného biorytmu a~k~narušení orientace v~noční krajině. Světelné zdroje přitahují hmyz a~často se kolem nich
% pohybují tak dlouho, až zahynou vyčerpáním.

% %
% % V mé diplomové práci se zaměřuji na umělé osvětlení ve výukové místnosti na gymnáziu Libeňský zámek. Cílem je specifikovat
% legislativní požadavky a~další doporučení na osvětlení školních učeben z hlediska technických a~světelně hygienických požadavků.
% Provést analýzu aktuálních světelných podmínek ve sledované učebně.
% S~pomocí měření osvětlenosti na referenčních rovinách a~následnou jasovou a~spektrální analýzou popsat možnosti
% osvětlovacího systému. Z naměřených výsledků navrhnout různé světelné scény odpovídající různým výukovým
% situacím s~důrazem na zrakový komfort a~biologické potřeby.
% Dále navrhnout implementaci sběrnicového systému pro realizaci navržených světelných scén a~změřit nové parametry
% osvětlení a~porovnat s~původními podmínkami. Na závěr popsat přínosy a~limity zvoleného řešení.

%
% V~mé diplomové práci se zaměřuji na návrh ovládacího panelu a~světelných scén v~učebně biologie na
% gymnázium Libeňský zámek. Původní ovládací panel s~12 tlačítky v~učebně způsoboval uživatelům problémy při
% nastavování osvětlení. Cílem mé práce bylo redukovat počet tlačítek na 3, kterými se budou zapínat scény
% podle biologického komfortu a~technických a~hygienických požadavků stanovených legislativou.

%
% V~teoretické části se zabývám popisem umělého osvětlení, fungováním lidského oka a~také popisem DALI protokolu.

%
% V~praktické části jsem provedla analýzu původního stavu učebny, zahrnující analýzu umělého osvětlení a
% rozložení jasu. Dále jsem popsala původní ovládací panel učebny a~navrhla nový panel obsahující dané scény.
% Také jsem zde analyzovala nově navržené scény. Na závěr jsem hodnotila nový stav světelných scén a~porovnala ho s~původním návrhem.


% proc to tema resim? Je to vubec dobre? Ma to smysl?
% Dale rozepsat co resim v diplomove praci - klidne rozepsat o~kazde kapitole, co tam resim

%%%%%%%%%%%%%%%%%%%%%%%%-----------------------------------------------------------------------------------
% Přechodem na inteligentní domácnost se pojí i~inteligentní osvětlení, které má za cíl zlepšit naši koncentraci, produktivitu a~efektivitu práce.
% Toto osvětlení má pozitivní vliv na náš zdravotní stav, což je v dnešní době, kdy trávíme většinu času ve vnitřních prostorech, klíčové.
% Nastavení správných parametrů osvětlení je však náročnou úlohou, která závisí na řadě faktorů, jako je venkovní počasí,
%  dispozice místnosti, barva stěn, nebo dokonce počet osob v prostoru.

% %
% Moderní technologie umožňují osvětlení přizpůsobit dennímu cyklu, což je základním principem našeho biologického rytmu,
% vycházejícího z historického zvyku trávit více času venku. Přestože dnes můžeme své aktivity plánovat nezávisle na denním světle,
% naše tělo stále reaguje na přirozený cyklus dne a~noci.

% %
% Nesprávné osvětlení ve večerních a~nočních hodinách může narušit tento biologický rytmus, ovlivnit tvorbu hormonu melatoninu,
% který má klíčovou roli v regeneraci a~ochraně našeho těla. Stejně tak umělé světlo negativně ovlivňuje zvířata a~hmyz,
% narušuje jejich přirozený biorytmus a~orientaci v noční krajině.

% %
% Má diplomová práce se zaměřuje na problematiku osvětlení v učebně biologie na gymnáziu u Libeňského zámku.
% S~využitím moderních technologií se snažím optimalizovat ovládací panel osvětlení tak, aby reflektoval biologický a~hygienický komfort uživatelů.
% Redukce počtu ovládacích prvků na 3 -- 4 tlačítka a~implementace scén podle specifických požadavků je hlavním cílem práce.

% %
% Teoretická část se zabývá popisem umělého osvětlení, funkčností lidského oka a~vysvětlením protokolu DALI.
% V praktické části jsem provedla analýzu stávajícího stavu osvětlení a~navrhla nové scény a~ovládací panel,
% které lépe odpovídají požadavkům uživatelů a~technickým standardům. Hodnocení nového stavu světelných scén pak
% ukazuje efektivní vylepšení v porovnání s~původním stavem.

