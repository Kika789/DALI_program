\chap Úvod
Přechodem na inteligentní domácnost se pojí i inteligentní osvětlení, které má pomoci v~lepší koncentraci, produktivitě a efektivní práci. Toto osvětlení má pozitivní dopady
na náš zdravotní stav, zrakový komfort, správný vývoj a také může zabránit negativním
účinkům, jako je deprese. Z toho důvodu si někteří lidé pořizují speciální biolampy, aby
měli pocit, že jsou venku na sluníčku. Vzhledem k tomu, že v~dnešní době se většinu času
pohybujeme ve vnitřních prostorech, stává se klíčovým faktorem nastavení správného
umělého osvětlení.

\medskip
Nestavení správného osvětlení je velkou výzvou. Záleží na mnoha faktorech, jako je venkovní počasí, velikost a orientace oken,
barva stěn a nábytku, počet lidí v místnosti, celková velikost prostoru atd. Také je zapotřebí vybrat správný typ osvětlovacích prvků,
které také závisí na specifické činnosti vykonávané v místnosti. Snaha je nastavit parametry tak, abychom se nejlépe přizpůsobili
dennímu osvětlení, protože je tomu tak přizpůsobené naše lidské oko, vychází to z historie, kde lidé strávili mnohem více času venku.

\medskip
Svícení ve večerních a nočních hodinách narušuje biorytmus lidského těla. Ovlivňuje tvorbu hormonu melatoninu, který je
zodpovědný za spouštění regeneračních procesů v těle a očišťování organismu od toxických látek. Dále melatonin podporuje
funkci imunitního systému a chrání organismus před nežádoucími chemickými reakcemi.

\medskip
Umělé světlo nemá špatný vliv jen na lidské zdraví, ale ovlivňuje i zvířata a hmyz. U nichž totiž dochází k narušení
přirozeného biorytmu a k narušení orientace v noční krajině. Světelné zdroje přitahují hmyz a často se kolem nich
pohybují tak dlouho, až zahynou vyčerpáním.

% \medskip
% V mé diplomové práci se zaměřuji na umělé osvětlení ve výukové místnosti na gymnáziu Libeňský zámek. Cílem je specifikovat legislativní požadavky a další doporučení na osvětlení školních učeben z hlediska technických a světelně hygienických požadavků. Provést analýzu aktuálních světelných podmínek ve sledované učebně. S pomocí měření osvětlenosti na referenčních rovinách a následnou jasovou a spektrální analýzou popsat možnosti osvětlovacího systému. Z naměřených výsledků navrhnout různé světelné scény odpovídající různým výukovým situacím s důrazem na zrakový komfort a biologické potřeby. Dále navrhnout implementaci sběrnicového systému pro realizaci navržených světelných scén a změřit nové parametry osvětlení a porovnat s původními podmínkami. Na závěr popsat přínosy a limity zvoleného řešení.

\medskip
V mé diplomové práci se zaměřuji na návrh ovládacího panelu a světelných scén v~učebně biologie na
gymnázium Libeňský zámek. Původní ovládací panel s 12 tlačítky v učebně způsoboval uživatelům problémy při
nastavování osvětlení. Cílem mé práce bylo redukovat počet tlačítek na 3, kterými se budou zapínat scény
podle biologického komfortu a technických a hygienických požadavků stanovených legislativou.

\medskip
V teoretické části se zabývám popisem umělého osvětlení, fungováním lidského oka a také popisem DALI protokolu.

\medskip
V praktické části jsem provedla analýzu původního stavu učebny, zahrnující analýzu umělého osvětlení a
rozložení jasu. Dále jsem popsala původní ovládací panel učebny a navrhla nový panel obsahující dané scény.
Také jsem zde analyzovala nově navržené scény. Na závěr jsem zhodnotila nový stav světelných scén a porovnala ho s původním návrhem.


% proc to tema resim? Je to vubec dobre? Ma to smysl?
% Dale rozepsat co resim v diplomove praci - klidne rozepsat o kazde kapitole, co tam resim 



