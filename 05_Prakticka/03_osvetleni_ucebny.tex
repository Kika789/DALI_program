\chap Všeobecné požadavky na umělé osvětlení

Navázal bych na polemiku týkající se výšky srovnávací roviny. V~[5] je pro předškoláky uvedeno stejné číslo
jako pro denní osvětlení, tedy 45 cm nad podlahou. Výhrady mám stejné. Rozumněji je již vznesen obecný
požadavek na výšku srovnávací roviny, kde se píše, že je to výška stejná jako převládající výška lavic.
Uvedené platí pro horizontální rovinu.

\medskip
Zde se hodí připomenutí, že se často špatně volí srovnávací rovina. To se týká nejen její výšky,
ale i~orientace. Srovnávací rovina je tam, kde je předmět zájmu pozorovatele. Je to papír na stolní
desce, sešit na lavici, svislá školní tabule, podlaha na chodbě, v~herně nebo v~tělocvičně.
Srovnávací rovina může být nejen vodorovná či svislá, ale obecně šikmá, například (dnes snad již jen historické)
 lavice se šikmou pracovní deskou.
\medskip
Ještě jedno je nutné mít na paměti. To je velikost této roviny. Za místo zrakového úkolu je považován prostor
s~lavicemi nebo stůl učitele a~za blízké okolí zrakového úkolu prostor místnosti sloužící výuce.
To znamená, že předepsaná průměrná udržovaná osvětlenost 300 lx musí být splněna na lavicích s~rovnoměrností 0,7
(podle novelizace, která vejde v~platnost v~nejbližší době /možná již v~okamžiku vydání tohoto čísla platí/ to bude 0,6).
 To znamená, že minimální osvětlenost kdekoliv na této ploše smí být 210 lx (nově 180).
 Ve zbytku učebny je blízké okolí, tam je požadována osvětlenost o~stupeň nižší, tedy 200 lx.
 S~rovnoměrností 0,5 (nově 0,4). Z~toho plyne, že v~učebně může být minimální osvětlenost 100 lx, resp. 80 lx.
 Otázce rovnoměrnosti osvětlení jsem se věnoval podrobněji proto, že existují pracovníci kontrolních orgánů,
 kteří šmahem požadují dosažení udržované osvětlenosti 300 lx v~celé učebně s~rovnoměrností 0,7.
 Takový požadavek je nelegitimní, neohleduplný ke kapse investora a~nakonec i~nepřímo k~životnímu prostředí.
 Zneužití pravomoci se tomu říká.
\medskip
A~když už, ještě něco o~hygienicích. Dalším parametrem osvětlovacích soustav je míra oslnění.
Ta se vyjadřuje prostřednictvím indexu rušivého oslnění (UGR). Norma předepisuje jeho maximální velikost.
Nemyslící hygienik na nepřekročitelnosti tohoto čísla trvá. Je to naprostý nesmysl.
Index rušivého oslnění je jen číslo, které kvantifikuje subjektivní pocit nepohody nebo snížené
schopnosti vidět podrobnosti. Ovšem každý člověk vnímá oslnění jinak. Někomu vadí již nepatrně
jasnější ploška v~zorném poli, někomu nevadí ani významně vyšší jas. Na základě subjektivního
pozorování byl sestaven vzorec, který se snaží složitou situaci vyjádřit jednoduchým vztahem.
Nechci zde uvádět vzorce a~výpočty, tak jen tolik, že ve vztahu je UGR vyjádřeno jako osminásobek
nějakého logaritmu. Proč tam není 8,1 nebo 7,9? Ve vlastním logaritmu se zase nalézá 1/4.
Proč ne 1/4,1? Proč ne 1/3,9? Protože není důvod, jde o~snahu nějakým způsobem kvantifikovat
nekvantifikovatelné, takže zaokrouhlení na celá čísla stačí. Kdyby byly ve vzorci pro výpočet
UGR čísla o~desetinu jiná, výsledné hodnoty by se lišili od normovaných o~pět a~více procent. T
rvat tedy na tom, že UGR = 19,1 (místo 19,0) je nevyhovující, je projev nekompetentnosti úředníka
(je to odchylka půl procenta od nominálu). Kde je nejistota vstupních dat, nejistota výpočtu?
Ostatně zcela obdobně lze mluvit i~o~jiných parametrech osvětlení. Vždy je nutné přistupovat
k~úloze návrhu osvětlení zodpovědně, a~rozhodně s~rozumem.

\sec Světelné zdroje, svítidla a~osvětlovací soustavy

Pro volbu světelného zdroje platí stejná pravidla jako v~kancelářských prostorech. V~drtivé v
ětšině se budou používat zářivky. Na místech s~občasným pobytem, například ve skladech, to mohou být
i žárovky, V~případě, že je to ekonomicky výhodné, tak zde mohou být i~kompaktní zářivky či LED.
Obr. 6
Obr. 6 Ve velkých posluchárnách se nelze vyhnout zakázaným zónám; také někdy nelze okna umístit
do bočních stěn a~svítidla umístit v~řadách s~ní rovnoběžných
Obr. 3
Obr. 3 Příklad svítidla pro osvětlování učeben
\medskip
Pro svítidla ve školách platí, že musejí být bezpečná. A~samozřejmě i~odolná. Musejí přežít houbové
a jiné bitvy během přestávek bez pedagogického dozoru. Pochopitelně musejí být energeticky úsporná.
To znamená nejen s~vysokou účinností, ale především s~vysokým činitelem využití, který je zajištěn
dobrou distribucí světelného toku. Vhodná jsou svítidla s~dostatečnou nepřímou složkou, která zaručí
dobré prostorové vidění. V~neposlední řadě by měla být svítidla nenáročná na údržbu.
Také vzhled svítidla je důležitý, protože ovlivňuje estetické cítění dětí.
\medskip
V~klasické učebně se pro celkové osvětlení volí prakticky vždy odstupňovaná osvětlovací soustava.
Svítidla musejí být umístěna tak, aby světlo dopadalo na místo zrakového úkolu ze správné strany
a aby byla na maximální míru omezena možnost oslnění odrazem. Všechny tyto požadavky se splní tím,
že se svítidla umístí těsně vedle levého okraje lavic a~jejich podélná osa je rovnoběžná s~uličkami.
Velice často je vidět naprosto chybné rozmístění svítidel, kdy jsou řady svítidel kolmé na uličky.
V~takovém případě zcela jistě dochází k~oslnění odrazem i~k~oslněním přímým, neboť v~příčné rovině mají
svítidla ve většině případů vyšší jas než v~rovině podélné. Rovněž směr dopadu světla je chybný
(zepředu místo ze strany). A~konečně - nedochází ani ke shodě směru umělého a~denního světla (v~případě sdruženého osvětlení).
Osvětlení tabule
Obr. 5
Obr. 5 Příklad svítidla pro osvětlování tabulí
\medskip
Zvláštní pozornost si zaslouží osvětlení tabule. Ta musí být osvětlena tak, aby ji bylo možné snadno
a s~co nejmenší námahou sledovat. Při změně pohledu z~tabule na lavici (a naopak) dochází ke změně směru
pohledu, oko se musí přizpůsobit jiné pozorovací vzdálenosti, změně jasu i~kontrastu. U~oblíbených a~stále
velmi hojných černých, ale i~tmavě zelených tabulí se střídá se změnou pohledu pozitivní kontrast
na tabuli za negativní v~knize či sešitu.
\medskip
Papír v~knize či sešitu ležícím na lavici má jas běžně i~100 cd.m−2. Aby nedocházelo k~nadměrnému
namáhání zraku při změně pohledu z~lavice na tabuli, tabule nesmí mít jas menší, než je přibližně
třetina jasu papíru. Pro černou tabuli by vertikální osvětlenost musela být kolem 1000 lx.
Pro tabuli tmavě zelenou již postačuje osvětlenost asi 600 lx. Uvedených hodnot lze dosáhnout
volbou vhodných svítidel a~světelných zdrojů. Změnu kontrastu (pozitiv - negativ) však
neovlivní žádné svítidlo. Řešením je světle šedá tabule a~černé omyvatelné fixy.
Světle šedá tabule bude mít kromě negativního kontrastu i~vyšší jas při stejně intenzivním osvětlení.
 Potom se zraková náročnost omezí pouze na přizpůsobení zraku na různou vzdálenost.
 Tomu již nelze zabránit - i~když není technicky nemožné nahradit tabuli, učebnice, sešity počítačem.
 A~není to ani příliš vzdálená budoucnost.
\medskip
V~ČSN EN 12 464-1 je pro osvětlení tabule požadováno pouhých 500vluxů. To vylučuje černou i~tmavě zelenou tabuli.
 Opět o~důvod víc pro volbu šedé tabule.
Obr. 4
Obr. 4 Čára svítivosti asymetrického svítidla pro osvětlování tabulí (obr. 5)
\medskip
Pro osvětlení školních tabulí jsou vhodná svítidla s~asymetrickým reflektorem, který zaručí,
že je světlo směrováno především ve směru k~tabuli. Současně je tak zajištěno dokonalé clonění
ze strany žáků v~lavicích.
\medskip
Zde je namístě připomenout, že je nevhodné kombinovat ve třídě světelné zdroje s~různým barevným podáním
(to ostatně platí obecně, nejen v~učebnách). Přesto jsou stále k~vidění někdejší „svítidla pro osvětlování tabulí“.
 To byla žárovková svítidla s~nekvalitním reflektorem. Nezajišťovala ani dostatečnou osvětlenost
 ani rovnoměrnost a~způsobovala zmíněnou nežádoucí nejednotnost barevného podání.
\medskip
Také použití běžných zářivkových svítidel není vhodné. Ani tehdy, když se pomocí nestejně
dlouhých závěsů natočí směrem k~tabuli. Ani pak se nedosáhne dostatečně intenzivního a~rovnoměrného osvětlení.
Obr. 7
Obr. 7 Netradiční učebna - netradiční rozmístění, tradičně tmavá tabule
\medskip
Není snad třeba připomínat, že svítidla pro osvětlení tabulí musejí být umístěna tak, aby se
na tabuli neodrážel jejich obraz směrem do očí posluchačů. Platí zde podobně zakázaná pásma,
jak to bylo uvedeno v~textu o~osvětlování kanceláří.
\medskip
Rozmístění svítidel tak, jak bylo popsáno, však nelze chápat jako dogma. Platí pro klasické
uspořádání učebny. Od něho se však v~poslední době ustupuje (aby se po čase k~němu zase vrátilo?),
a tak je zapotřebí osvětlit rovnoměrně celou místnost (obr. 7).
\medskip
Ale i~potom je nutné některé zásady dodržet. Je to orientace svítidel vůči oknům i~k~převážnému směru
pohledu žáků. Ten (směr pohledu žáků) by neměl být do oken, jako tomu je na obr. 7 (okna jsou ve stěně,
která není na obrázku vidět - směrem vlevo). Ostatně na obrázku lze vidět i~další prohřešek - tmavé tabule.