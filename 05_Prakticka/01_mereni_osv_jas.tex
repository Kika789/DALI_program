\chap Měření osvětlenosti a jasu

\sec Měřicí přístroje

Pro měření střední kulové (polokulové) osvětlenosti se používá luxmetr s~kulovým (polokulovým) nástavcem nebo luxmetr
s~fotonkou umísťovanou na boky elementární krychle. Pro měření střední válcové osvětlenosti se používá luxmetr
s~válcovým nástavcem nebo luxmetr s~fotonkou umísťenou na svislých bocích elementární krychle.
Pro měření světelného vektoru se používá speciální měřicí přístroj podle návodu výrobce nebo luxmetr s~fotonkou
umísťovanou taktéž na bocích elementární krychle. Pro měření teploty chromatičnosti se používá speciální
měřicí přístroj (spektrofotometr) podle návodu výrobce. Pro měření napájecího napětí se používají kalibrované
digitální (analogové) nebo registračních voltmetry.

\sec Výběr kontrolních bodu

Při výběru kontrolních bodů se postupuje podle článku 4.4 ČSN 36 0011-1:2006. Při volbě polohy kontrolních bodů
v~pravidelné síti na srovnávací vodorovné rovině pro měření osvětlenosti se bere v~úvahu rozmístění svítidel
se snahou vystihnout místa s~největší a nejmenší osvětleností (místa pod svítidly a mezi nimi), případně
se respektuje rozmístění pracovních míst (např. ve školní učebně).
\medskip
Při měření osvětlenosti pracovní plochy nebo jiné plochy (místo zrakového úkolu), u~které je nezbytné zjistit
rozložení osvětlenosti, průměrnou osvětlenost a rovnoměrnost osvětlení, se měřicí body rozmístí v~pravidelné
síti podle článku 4.4.4 ČSN 36 0011-1:2006. Osvětlenost se měří co nejblíže povrchu plochy, nejvýše 0,05 m nad ním.
\medskip
Měření osvětlenosti blízkého okolí místa zrakového úkolu se provádí v~rovnoměrné síti kontrolních bodů
podle 4.4.4 ČSN 36 0011-1:2006 umístěné kolem místa zrakového úkolu alespoň v~pásu šíře 0,5 m, uvnitř
zorného pole a ve stejné srovnávací rovině, kde se zrakový úkol nachází. Při úzkém pásu lze použít
řadu rovnoměrně rozmístěných bodů. Pokud nelze jednoznačně určit zorné pole, provede se měření pro celé okolí.
Střední kulová (krychlová) osvětlenost, střední válcová osvětlenost, světelný vektor a jasy pro hodnocení
oslnění, se měří v~kontrolních bodech v~polovině délky stěn vnitřního prostoru a to ve vzdálenosti 1 m
od povrchu stěny, ve výšce 1,5 m nad podlahou.
\medskip
Měření těchto veličin, měření střední polokulové i střední válcové osvětlenosti, se provádí i v~dalších
kontrolních bodech důležitých pro zrakové činnosti v~daném vnitřním prostoru. Výška těchto bodů nad
podlahou se volí 1,2 m nebo 1,5 m (průměrná výška oka sedící nebo stojící osoby).

\sec Postup při měření

Při všech měřeních umělého osvětlení se zařízení pro regulaci denního osvětlení uvedou do stavu
obvyklého pro daný druh umělého osvětlení (např. zatažené záclony, závěsy, rolety, žaluzie, zavřené okenice atd.).
Při měření umělého osvětlení se vyloučí vliv denního osvětlení jedním z~těchto způsobů:
\medskip
\begitems
    * měří se v~době bez denního světla (večer, v~noci);
    * měří se během dne při zatemnění (při zakrytých osvětlovacích otvorech).
\enditems
\medskip
Pro měření umělého osvětlení musí být dodržena minimální doba předběžného stárnutí světelných zdrojů,
žárovky musí svítit celkem nejméně 10 hodin, výbojové zdroje (včetně zářivek) nejméně 100 hodin.
\medskip
Před začátkem měření se zapne umělé osvětlení s~takovým předstihem, aby se světelný tok stabilizoval.
Za stabilizovaný se považuje tehdy, kdy měřená hodnota osvětlení při měření s~odstupem několika minut
třikrát po sobě, nevykazuje systematické změny při stanovení doby stabilizace se berou v~úvahu i údaje výrobce.
U~výbojových zdrojů se považuje za minimální dobu stabilizace světelného toku 20 minut, pň uzavřených
svítidlech může být tato doba ještě delší (ustálení provozní teploty).
\medskip
Při přesném a provozním měření umělého osvětlení se měří faktory ovlivňující osvětlení, zejména teplota
vzduchu a napájecí napětí světelného obvodu (v~některých zařízeních je možné získat spolehlivé údaje od
provozovatele). Při provozním měření se může použít běžných teploměrů a voltmetrů. V~tomto případě se
doporučuje použít přesných registračních přístrojů nebo přístrojů umožňujících stanovení průměrné hodnoty
v~měřeném intervalu. Při měření napájecího napětí se doporučuje snímat jeho velikost na svorkách rozváděče,
pro napájení měřené osvětlovací soustavy v~běžném provozu. V~případě, že je pokles napájecího napětí větší,
než stanoví ČSN 33 0120, nelze přesná ani provozní měření osvětlení spolehlivě provádět.
\medskip
Při posouzení kombinovaného umělého osvětlení se měří osvětlenost nejprve při kombinovaném osvětlení
(celkovém i místním), pak pouze pň celkovém a zaznamenají se obě hodnoty.
Střední kulová (polokulová) osvětlenost se měří tak, aby nástavec nebyl ze žádné strany zastíněn
(například měřící osobou). Není-li k~dispozici kulový (polokulový) nástavec, může se měřit
orientačně tak, že se změří luxmetrem osvětlenost v~daném měřicím bodě na stěnách fiktivní krychle,
a to zanedbatelných rozměrů v~pravoúhlém systému souřadnic. Ze změřených šesti (pěti) hodnot osvětlenosti
se vypočte průměrná hodnota. Strany krychle musí být rovnoběžné se stěnami měřeného prostoru.
Analogicky se měří střední polokulová osvětlenost.

\medskip
Střední válcová osvětlenost se měří tak, aby nástavec nebyl zastíněn zejména z~bočních směrů.
Není-li k~dispozici válcový nástavec, stanoví se válcová osvětlenost přibližně na základě měření
vertikální osvětlenosti v~měřicím bodě ve čtyřech svislých rovinách (v~pravoúhlém vnitřním prostoru
rovnoběžných se stěnami) vzájemně kolmých, proložených kontrolním bodem. Hodnota válcové osvětlenosti
se stanoví jako aritmetický průměr změřených čtyř osvětleností. Strany krychle musí být rovnoběžné
se stěnami měřeného prostoru. Přesnějších výsledků se touto metodou docílí při větším počtu svislých
rovin (8 nebo 12). Analogicky se měří střední poloválcová osvětlenost.

\medskip
Světelný vektor ε se měří speciálním přístrojem podle údajů výrobce. Není-li takový přístroj k~dispozici,
změří se luxmetrem osvětlenost v~daném kontrolním bodě na stěnách fiktivní krychle zanedbatelných rozměrů,
v~pravoúhlém systému souřadnic, ze změřených osvětleností protilehlých stran krychle se vypočtou složky
světelného vektoru v~pravoúhlém souřadnicovém systému.

\sec Měření jasu ploch

Jasy ploch ve vnitřních prostorech se měří podle článku 4.6.6 a 4.6.7 ČSN 360011-1:2006. Pokud není
možné stanovit jasy svítidel na základě údajů výrobce (jasy svítidla a jeho jednotlivých částí z~různých
směrů, nebo výpočtem průměrného jasu z~údajů o~svítivosti v~daném směru), měří se jasy svítidel v~kontrolních
bodech dle 4.4.13 ČSN 360011-1:2006.
\medskip
Při rozdílných jasech jednotlivých částí světelné aktivních ploch svítidla se stanoví průměrný jas
svítidla jako průměrná hodnota z~jasů jednotlivých částí svítidla s~ohledem na velikost těchto částí.
U~otevřených svítidel se zahrnuje do měření jasů i měření jasu světelného zdroje, pokud je z~daného
kontrolního bodu viditelný.
\medskip
Ve vnitřních prostorech se zrakovými činnostmi, které mohou být nepříznivě ovlivněny odrazem světla
od lesklých nebo pololesklých povrchů v~zorném poli (např. pracoviště s~obrazovkami, pracoviště s~lesklými
nebo pololesklými povrchy pracovní plochy nebo předmětu pozorování) se zjišťují jasy svítidel i v~úhlech,
které jsou pro vytváření odrazů kritické (z~míst na odrážející ploše).

\medskip
Zdroje: https://www.elektroprumysl.cz/merici-technika/mereni-umeleho-osvetleni-vnitrnich-prostor